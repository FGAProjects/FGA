  O principal problema a ser enfrentado todos os dias pelos alunos e servidores da Faculdade do Gama, que utilizam carro para se deslocarem até a faculdade, é o da incerteza sobre a integridade dos mesmos ao final do dia. A segurança do campus possui muitas falhas, como a falta de cercamento, guaritas, rondas de funcionários contratados para a segurança e da própria polícia.

	Portanto, o projeto do SUM visa potencializar a segurança, já existente no campus, com o uso de balões que farão o monitoramento 24 horas de toda extensão do estacionamento do campus. Para isso, será necessário desenvolver a estrutura do balão, o sistema de transmissão de dados entre eles, pois visa-se projetar mais de um balão para que se possa cobrir toda a área do estacionamento, o tratamento de imagens e modelo de padrões de risco e sua alimentação. Todos os tópicos a seguir apresentam a solução de cada item elencado.

\section{Proposta da Solução/Funcionamento do Sistema} % (fold)
\label{sec:proposta_da_solu_o_funcionamento_do_sistema}
O produto foi idealizado como sendo um sistema capaz de integrar vários balões à área a ser monitorada, que será o estacionamento da faculdade UnB Gama.

Foram definidos cinco pontos estratégicos de localização dos balões cativos, que podem ser vistos na figura \ref{img:localizacao-baloes}, para que não possuam pontos cegos.


\begin{figure}[H]
	\centering
	\caption{Localização dos balões no campus FGA}
	\includegraphics[width=0.7\textwidth]{figuras/localizacao-baloes}
	\label{img:localizacao-baloes}
\end{figure}


Para o melhor entendimento do funcionamento geral do Sistema Unificado de Monitoramento pelo leitor, foi feito, com o uso do programa \emph{CoreIDRAW X7}, uma ilustração simplificada (figura \ref{img:ilus-fun-sum}) da disposição de alguns balões, estes presos no terraço do prédio da faculdade, e a estação de solo, localizada ao lado do prédio. Todos os elementos do projeto foram enumerados  para que, em seguida, o leitor possa acompanhar o funcionamento geral e a disposição da solução específica no presente relatório.

\begin{figure}[H]
\centering
\caption{Ilustração do funcionamento do SUM}
\includegraphics[width=\textwidth]{figuras/ilus-fun-sum}
\label{img:ilus-fun-sum}
\end{figure}

\begin{enumerate}

\item O balão cativo  SP (Super Pressão)  consistirá de uma bexiga (figura \ref{img:payload}), carga útil, cabeamento de sustentação (figura \ref{img:sustentacao}) e base de ancoramento.  Este necessitará de três pontos de fixação. Destes três, dois pontos estarão fixos no topo do respectivos prédios e o terceiro ponto será fixado em um poste de sustentação que terá a altura do prédio cerca de 10 metros de altura e estará distante 5 metros perpendicularmente ao mesmo.  A bexiga é o componente capaz de gerar uma sustentação, força que promove a subida do balão, e que será preenchida com um tipo de gás menos denso que o ar, gerando força para cima.

\begin{figure}[htp]
\centering
\caption{Protótipo da bexiga, \textit{payload} e cabos de sustentação feitos no software CATIA.}
\includegraphics[width=0.6\textwidth]{figuras/balaoRenderizado}
\label{img:payload}
\end{figure}

O gás do balão sairá do envelope a uma taxa baixa por semana, de acordo com as características do material do envelope, sendo necessária a reposição de gás a cada 4 meses. Para isso, serão instalados em dois pontos fixos dos prédios motores elétricos e será utilizado o cabo de aço Alma de Fibra.

O desenvolvimento da solução está disposto no Subprojeto de Estrutura e Sistema Aéreo, localizado na seção \ref{sec:subprojeto_de_estrutura_e_sistema_a_reo}.

\item A carga útil ou \textit{payload} (figura \ref{img:sustentacao}), é a unidade formada por uma estrutura portando todos  os equipamentos eletrônicos que promovem o monitoramento desejado. Por exemplo, câmeras, sensores, microcontroladores. Toda a transmissão de dados entre o balão e a estação de solo será feita por meio de ondas de rádio em uma faixa de frequências autorizada pelo órgão nacional competente , a ANATEL. Possuirá também um sistema de estabilização da carga para que as imagens possam ser transmitidas sem interferências.

\begin{figure}[htp]
\centering
\caption{Protótipo da \textit{payload} e cabos de sustentação feitos no software CATIA.}
\includegraphics[width=0.6\textwidth]{figuras/PAYLOADRENDERIZADA}
\label{img:sustentacao}
\end{figure}

O desenvolvimento da solução está disposto no Subprojeto de Eletrônica Embarcada, na seção \ref{sec:subprojeto_da_eletr_nica_embarcada}.

\item A estação de solo possui como função a recepção dos vídeos transmitidos pelo balão, de forma que um operador possa efetuar um diagnóstico do que acontece no estacionamento do campus. Tal estação deve ser capaz de estabelecer a recepção de vídeos de todos os balões do sistema SUM operantes na área de análise.

O sistema presente na estação de solo identifica se  está acontecendo alguma ação suspeita. Ao identificar,  aciona os operadores via rádio  para que eles verifiquem o motivo do alerta feito pela estação (figura \ref{img:processo}).

	\begin{figure}[H]
		\centering
		\caption{Processo 1- Funcionamento SUM. Fonte: Do autor, 2015.}
		\includegraphics[width=\textwidth]{figuras/processo}
		\label{img:processo}
	\end{figure}


	O desenvolvimento da solução está disposto no Subprojeto da Estação de Solo, na seção \ref{sec:subprojeto_da_esta_o_de_solo}.

\item O sistema de alimentação do SUM será OnGrid, ou seja, via rede de distribuição, e sistema de back up com 18 baterias estacionárias de 70 amperes dando 6 horas de autonomia.  O consumo do SUM será de 4712,35 kWh/mês. O desenvolvimento da solução está disposto no Consumo Energético, na seção \ref{sec:consumo_energ_tico}.

\end{enumerate}


\section{Subprojeto de Estrutura e Sistema Aéreo} % (fold)
\label{sec:subprojeto_de_estrutura_e_sistema_a_reo}
O principal desafio do sistema aéreo é manter a payload estável e em uma altitude constante. A câmera deve ser capaz de conseguir identificar a situação de risco sem que ocorra prejuízos nas imagens devido a má alocação do balão ou instabilidade do mesmo.

Para a flutuação será utilizado o princípio de Arquimedes do empuxo. Para fazer o balão flutuar, a força de empuxo deve ser maior que o peso total de toda a estrutura, portanto deve-se escolher os materiais mais leves e resistentes disponíveis no mercado.

Cada balão terá três pontos de fixação, esses pontos se encontram no teto dos prédios da universidade, porém alguns balões necessitarão de postes de fixação ao lado dos prédios. Cada balão terá dois pontos de fixação motorizados para que se possa realizar manutenções, os cabos e as junções devem suportar as tensões envolvidas no processo. Os cabos também devem ser dimensionados para suportar variações na velocidade do vento, que geram força de arrasto.
A seguir serão apresentadas mais detalhadamente  discussões e resultados referentes às questões levantadas acima.

\subsection{O Envelope} % (fold)
\label{sub:o_envelope}

\subsubsection{Gás do Balão}

	Existem duas possibilidades para a escolha do gás do balão cativo, o gás hélio e o gás hidrogênio. A seguir são apresentadas duas tabelas contendo as características físicas dos gases que podem ser escolhidos para o balão. Nas tabelas \ref{tab:caracHelio} e \ref{tab:caracHidro} as características do hélio e do hidrogênio são tiradas da empresa \textit{Gama Gases}.

	\begin{table}[H]
		\centering
		\caption[Características do Hélio]{Características do Hélio\cite{gamagases1}}
		\begin{tabular}{|c|c|}
			\hline
			\rowcolor[HTML]{FFFFFF}
			{\color[HTML]{000000} \textbf{Propriedades}}          & {\color[HTML]{000000} \textbf{Valores Numéricos}}   \\ \hline
			Densidade absoluta, gás a 101.325kPa a 0 ºC           & 0.1785 $Kg/m^3$                                     \\ \hline
			Densidade crítica                                     & 0.5307 $Kg/m^3$                                     \\ \hline
			Densidade relativa, gás a 101.325 kPa a 0 ºC, (ar = 1)& 0.138                                               \\ \hline
			Fator crítico de compressibilidade                    & 0.305                                               \\ \hline
			Fórmula                                               & 4He                                                 \\ \hline
			Massa Molecular                                       & 4.002602                                            \\ \hline
			Pressão crítica                                       & \begin{tabular}[c]{@{}c@{}}229 kPa ; 2.29 bar; 33.2 \\
			psia; 2.261 atm.\end{tabular}                                                                               \\ \hline
			Viscosidade, gás a 101.325 kPa a 26.8 ºC.             & 0.02012 mPa x s; 0.02012 cP.                        \\ \hline
			Volume específico a 21.1 ºC 101.325 kPa               & 6030.4$dm^3$/ kg; 96.6 $ft^3$/ Ib                   \\ \hline
		\end{tabular}
		\label{tab:caracHelio}
	\end{table}


\begin{table}[H]
	\centering
	\caption[Características do Hidrogênio]{Características do Hidrogênio~\cite{gamagases2}}
	\begin{tabular}{|c|c|}
		\hline
		\rowcolor[HTML]{FFFFFF}
		{\color[HTML]{000000} \textbf{Propriedades}}          & {\color[HTML]{000000} \textbf{Valores Numéricos}} \\ \hline
		Densidade absoluta, gás a 101.325kPa a 0ºC            & 0.08235 $Kg/m^3$                                  \\ \hline
		Densidade crítica                                     & 0.0310 $Kg/m^3$                                   \\ \hline
		Densidade relativa, gás a 101.325 kPa a 0ºC, (ar = 1) & 0.0695                                            \\ \hline
		Fator crítico de compressibilidade                    & 0.305                                             \\ \hline
		Fórmula                                               & H2                                                \\ \hline
		Limites de inflamabilidade no ar.                     & 4.0 à 75\% (por volume).                          \\ \hline
		Massa Molecular                                       & 2,01588                                           \\ \hline
		Pressão crítica                                       & \begin{tabular}[c]{@{}c@{}}1297 kPa; 12.97 bar; 188.1 psia;
	                                                                                                            \\
		12.80 atm.\end{tabular}                                                                                   \\ \hline
		Temperatura de auto-ignição.                          & \begin{tabular}[c]{@{}c@{}}844.3 K; 571.2 ºC;     \\
		1060 ºF.\end{tabular}                                                                                     \\ \hline
		Volume específico a 21.1 ºC, 101.325kPa.         			& 11967,4$dm^3$/kg; 191,7$ft^3$/lb.                 \\ \hline
	\end{tabular}
	\label{tab:caracHidro}
\end{table}

	O gás hidrogênio a primeira vista é mais vantajoso pois é mais leve que o hélio, sua densidade relativa ao ar é de 0.0695 enquanto que a do hélio é de 0.138, e apresentam um fator crítico de compressibilidade iguais. Porém o hidrogênio possui a característica de ser inflamável, enquanto que o hélio é conhecido por ser um gás inerte. Tendo em vista a segurança dos usuários do estacionamento e dos funcionários responsáveis pela manutenção do balão, a exposição ao sol e a possíveis, porém improváveis,  descargas elétricas o hélio se mostra a opção mais vantajosa.

	\subsubsection{Material do Envelope}

	Segundo \citeauthoronline{yajima} (\citeyear{yajima}), a maioria dos balões atmosféricos são feitos de filme de polietileno. A espessura dos envelopes dos balões usados pela NASA variam de 7 a 90 micrometros dependendo da altitude, funcionalidade, tempo de atividade e peso da payload e etc. Desta forma, cabe analisar que tipo de polietileno será utilizado para a confecção do envelope.

	Duas opções de polietileno foram analisadas, o Polietileno Linear de Baixa Densidade (PELBD) e o Polietileno de Baixa Densidade (PEBD). De acordo com \citeauthoronline{coutinho} (\citeyear{coutinho}), a temperatura máxima de atuação do PELBD é cerca de 120 ºC, sua massa específica varia numa faixa de 0.92 a 094 $g/cm^3$ e possui uma resistência à tração de 37 Mpa.

	A outra opção, o PEBD, segundo \citeauthoronline{coutinho} (\citeyear{coutinho}), trabalha a uma temperatura máxima de 110 ºC, possui uma massa específica de 0.92 $g/cm^3$ e resistência à tração de 24 Mpa.

	Outro material pesquisado foi o PVC pneumático, que possui densidade de 1.39 $g/cm^3$ \cite{innovainstrucao} e resistência a tração de 40 MPa E trabalha a uma temperatura máxima de 140 $^{\circ}$C \cite{datasheetPvc}. A tabela \ref{tab:tabmaterial} mostra os valores importantes dos materiais aqui citados.

	\begin{table}[htp]
		\centering
		\caption{Características dos materiais}
		\label{tab:tabmaterial}
		\begin{tabular}{ | c | c | c | c |}
			\hline
			\textbf{Material} &\textbf{Densidade} &\textbf{Resistência à tração} &\textbf{Temp. máx. de atuação}\\ \hline
			PELBD & 0.92 a 094 $g/cm^3$ &   37 MPa.&  120 $^{\circ}$C \\ \hline
			PEBD & 0.92 $g/cm^3$ &  24 MPa.&  110 $^{\circ}$C \\ \hline
			PVC & 1.39 $g/cm^3$ &  40 MPa.& 140 $^{\circ}$C \\
			\hline
		\end{tabular}
	\end{table}

	Analisando as duas opções de polietileno, o PELBD é o material mais adequado ao envelope do balão, pois possui melhor resistência mecânica e é leve, portanto, a priori ele seria o material escolhido, mas não foi possível encontra-lo na pesquisa de mercado, desta forma o material utilizado será o PVC pneumático.

% subsection o_envelope (end)

\subsection{Modelo do Balão} % (fold)
\label{sub:modelo_do_bal_o}

\subsubsection{Formato do Balão}
	De acordo com \citeauthoronline{yajima} (\citeyear{yajima}), existem vários formatos para balões atmosféricos, como o balão esférico, balão cilíndrico, balão tetraédrico e balão formato natural. O sistema mais fácil de ser trabalhado é o balão esférico, pois os cálculos de empuxo, volume, gás  apresentam menos complicações. Uma outra característica dos balões esféricos é a possibilidade do uso de uma fita de carga para suspender a payload no equador do envelope, o uso desse artifício tem por objetivo distribuir melhor a tensão na superfície do envelope.

	O esquema do balão pode ser observado na imagem \ref{img:balaoEsferico}.

	\begin{figure}[H]
		\centering
		\caption{Esquema de Balão Esférico com Fita de Carga.}
		\includegraphics[width=0.5\textwidth]{figuras/balaoEsferico}
		\label{img:balaoEsferico}
	\end{figure}

	A fórmula para cálculo da tensão é a seguinte:

	\begin{equacao}
	\caption{Tensão do balão}
		\begin{equation}
			T = \frac{pr}{2}
		\end{equation}
		\label{eqn:calculoTensao}
	\end{equacao}

	Em que:
	\begin{description}
		\item[T] = tensão
		\item[p] = pressão interna do balão
		\item[r] = raio do balão
	\end{description}

\subsubsection{Sistema do Balão}

	Existem dois modelos de balões metereológicos empregados atualmente pela Agência Espacial Ameriacana (NASA), o sistema Zero Pressão (ZP) e o Super-Pressão (SP).

	O sistema ZP recebe esse nome porque a pressão interna do balão é a mesma  pressão do ambiente de atuação do sistema. Na parte de baixo do envelope do balão existe pelo menos um duto que permite a saída do gás dentro do balão para o alívio da pressão interna~\cite{nasa3}, como é exemplificado na figura \ref{img:maiorBalaoZeroPressao}.

	\begin{figure}[htp]
		\centering
		\caption[Maior Balão Zero Pressão Usado pela NASA]{Maior Balão Zero Pressão Usado pela NASA~\cite{nasa1}}
		\includegraphics[width=0.5\textwidth]{figuras/maiorBalaoZeroPressao}
		\label{img:maiorBalaoZeroPressao}
	\end{figure}

	Para esse sistema normalmente é utilizado um material com 20 micrometros de espessura para o envelope. O balão é inflado em terra e é solto na atmosfera até que a força de empuxo seja igual ao seu peso. O duto funciona de tal modo que quando a pressão interna na base do balão excede a pressão da atmosfera exterior, o duto é empurrado para fora e forma uma um cilindro que permite que parte do gás seja expelida. Quando a diferença de pressão é negativa, a pressão atmosférica empurra o duto para dentro, impedindo a entrada de ar. Com a queda de temperatura, o gás esfria e contrai-se, diminuindo o volume e consequentemente o empuxo. Em balões atmosféricos usado pela NASA, para manter a altitude constante durante a noite o balão solta pesos c, como por exemplo areia, durante esse período a quantidade de gás dentro do envelope é a mesma pois o duto se fecha quando o gás se contrai~\cite{eoss}. Porém, no próximo dia, com o aumento da temperatura o gás irá se expandir, aumentando o volume e subindo mais, pois estará mais leve. Para manter uma altitude a mais constante possível, segundo a \citeauthoronline{nasa3} (\citeyear{nasa3}), um balão desse tipo requer uma perda de 6 a 8\% de massa quando a temperatura diminui. O principal problema desse tipo de sistema é a perda de gás pelo duto aberto ao ambiente.

	O sistema SP possui a vantagem de ser completamente vedado, ou seja, não perde gás durante seu período de atividade. Por esse fato o balão tem uma pressão interna maior que a pressão externa e isso implica no aumento da espessura do envelope, normalmente cerca de 10 vezes maior que a espessura de um balão ZP~\cite{yajima}. A figura \ref{img:distribuicaoPressao} mostra a distribuição de pressão de um balão atmosférico da NASA, percebe-se que o equador do envelope é a área de maior concentração de pressão, esse fato exige que a parte de fixação seja reforçada para que o envelope não seja danificado.

	\begin{figure}[htp]
		\centering
		\caption[Distribuição de Pressão de um Balão Super Pressão]{Distribuição de Pressão de um Balão Super Pressão~\cite{nasa2}}
		\includegraphics[width=0.5\textwidth]{figuras/distribuicaoPressao}
		\label{img:distribuicaoPressao}
	\end{figure}

	Em linhas gerais o balão ZP é mais leve, porém possui o problema de perder gás durante sua operação. O balão SP tem a vantagem de não perder gás durante sua operação, porém é mais pesado pois precisa ter um envelope mais grosso. Pelo fato de não haver saída de gás, visando economia de hélio durante a operação, o sistema escolhido será o sistema Super Pressão.



\subsection{Cálculo  do Volume do Balão} % (fold)
\label{sub:c_lculo_do_volume_do_bal_o}

	\subsubsection{Cálculo Preliminar do Volume do Balão}

	O diagrama de corpo livre do balão pode ser observado na imagem \ref{img:corpoLivreBalao}.
	\begin{figure}[H]
		\centering
		\includegraphics[width=0.5\textwidth]{figuras/corpoLivreBalao}
		\caption[Diagrama de Corpo Livre de um Balão]{Diagrama de Corpo Livre de um Balão~\cite{justino}}
		\label{img:corpoLivreBalao}
	\end{figure}

	O volume do balão foi calculado primeiramente desprezando a tensão e o peso dos cabos. 	O volume do gás Hélio foi calculado da seguinte maneira:

	Partindo do principio de que o volume de gás tem que gerar uma força de empuxo maior que a força peso, como mostrado na figura \ref{img:corpoLivreBalao}, para que o balão consiga ficar a uma certa altitude. Os cabos de apoio do balão foram desconsiderados nesse cálculo preliminar. A partir disso considerou-se a equação \eqref{eqn:calcEmpuxo} para o calculo de empuxo.

	\begin{equacao}
	\caption[Fórmula para o cálculo de empuxo]{Fórmula para o cálculo de empuxo~\cite[pp.~63--65]{munson}}
		\begin{equation}
			E = \rho_{ar} \cdot g \cdot V
		\end{equation}
		\label{eqn:calcEmpuxo}
	\end{equacao}

	Em que:
	\begin{description}
		\item[$\boldsymbol{\rho_{ar}}$] = massa especıfica do ar
		\item[$\boldsymbol{g}$] = aceleração da gravidade
		\item[$\boldsymbol{V}$] = volume do fluído deslocado
	\end{description}

	A massa de helio ($M_{h}$) é calculada segundo a equação \eqref{eqn:calcMassaHelio}.

	\begin{equacao}
	\caption{Fórmula para o cálculo da massa do hélio}
		\begin{equation}
			M_{h} = 0.138 \cdot \rho_{ar} \cdot g \cdot V
		\end{equation}
		\label{eqn:calcMassaHelio}
	\end{equacao}

	A massa de material foi calcula pela equação \eqref{eqn:calcMassaMaterial}.

	\begin{equacao}
	\caption{Fórmula para o cálculo da massa do material}
		\begin{equation}
			M_{h} = \rho_{m} \cdot e_{m} \cdot A
		\end{equation}
		\label{eqn:calcMassaMaterial}
	\end{equacao}

	Em que:
	\begin{description}
		\item[$\boldsymbol{A}$] = área do envelope
		\item[$\boldsymbol{\rho_{m}}$] = massa expecífica do material ($940Kg/m^3$)
		\item[$\boldsymbol{e_{m}}$] = espessura do material ($20mu m$)
	\end{description}

	A massa da \emph{payload} foi estimada em cerca de 5kg, e para margem de segurança, os cálculos foram realizados com uma massa de 10kg. A equação \eqref{eqn:calcBalancoForca} foi utilizada para realizar o cálculo do balanço das forças.

	\begin{equacao}
	\caption{Fórmula para o cálculo do balanço das forças}
		\begin{equation}
			E = g \cdot (M_{h} + M_{p} + M_{m})
		\end{equation}
		\label{eqn:calcBalancoForca}
	\end{equacao}

	Substituindo \eqref{eqn:calcEmpuxo}, \eqref{eqn:calcMassaHelio}, \eqref{eqn:calcMassaMaterial} em \eqref{eqn:calcBalancoForca}, temos a equação \eqref{eqn:eqResultante1}.

	\begin{equacao}
	\caption{Equação resultante de \eqref{eqn:calcEmpuxo}, \eqref{eqn:calcMassaHelio} e \eqref{eqn:calcMassaMaterial}}
		\begin{equation}
			\rho_{ar} \cdot g \cdot V = g(M_{h} + M_{p} + M_{m})
		\end{equation}
		\label{eqn:eqResultante1}
	\end{equacao}

	Como trata-se de um balão esférico temos que $V = \frac{1}{3} \cdot 4 \pi \cdot r^2$. Voltando a equação \eqref{eqn:eqResultante1}, isolando e agrupando os termos semelhantes temos:

	\begin{equacao}
	\caption{Equação resultante de \eqref{eqn:eqResultante1}}
		\begin{equation}
			\frac{1}{3} \cdot \rho \cdot 4 \pi \cdot r^3 \cdot (1 - 0.138) - 18.8 \cdot 10^{-3} \cdot 4 \pi \cdot r^2 - 10 = 0
		\end{equation}
		\label{eqn:eqResultante2}
	\end{equacao}

	Substituindo os valores e usando o metodo de Newton~\cite[pp.~174--175]{hoffman} para calcular as raızes de polinômios para resolver a equação \eqref{eqn:eqResultante2}, foi encontrado um valor para o raio de 1.4 metros. Logo o balao teria que ter pelo menos 1.4 metros de raio. A partir desse valor calcula-se o empuxo, o peso do helio, o peso e a quantidade de polietileno que o balao ira precisar. Para o cálculo do empuxo considerou-se ar  = $1.11166 kg/m^3$, que é a densidade do ar a 1000m de altitude. Considerou-se h = 0.138 ar que é a massa específica do hélio. A tabela \ref{tab:forcasVerticaisAtuantes} mostra todos os valores de forças verticais atuantes no balão.

	\begin{table}[htp]
		\centering
		\caption{Forças Verticais Atuantes no Balão}
		\begin{tabular}{|c|c|}
			\hline
			\rowcolor[HTML]{FFFFFF}
			{\color[HTML]{000000} \textbf{Grandezas}} & {\color[HTML]{000000} \textbf{Valores Numéricos (N)}} \\ \hline
			Empuxo                                    & 138.1                                                 \\ \hline
			Peso do Hélio                             & 19.1                                                  \\ \hline
			Peso do material do envelope              & 4.6                                                   \\ \hline
			Peso da Payload                           & 100                                                   \\ \hline
		\end{tabular}
		\label{tab:forcasVerticaisAtuantes}
	\end{table}

	As contas acima foram realizadas para que se tivesse uma ideia inicial do tamanho do balão e da ordem de grandeza das forças associadas.

\subsubsection{Cálculo Real do Volume do Balão}

O  balão irá operar em uma faixa de altura entre 20 e 25 metros em relação ao solo, porém  o mesmo será fixado no alto dos prédios da Universidade de Brasília no Campus do Gama, esses prédios têm uma altura de 10 a 12 metros dependendo do prédio. Desta forma, foram decididos os pontos de fixação e calculado o tamanho dos cabos, e concluiu-se que os cabos deveriam ter no máximo 20 metros de comprimento.

O cabo de aço escolhido tem um diâmetro de 6,4 mm e tem massa aproximada de 146 gramas por metro, um balão precisa de três cabos de sustentação de 20 metros cada. Desta forma o peso devido a massa dos cabos está descrito na equação \eqref{eqn:pesoCabos}.

\begin{equacao}
\caption{Peso dos cabos}
	\begin{equation}
		P_{c} = C \cdot m \cdot g \cdot n_{c} \implies 20 \cdot 0.146 \cdot 9,8 \cdot 3 = 86 N
	\end{equation}
	\label{eqn:pesoCabos}
\end{equacao}

Em que:
\begin{description}
	\item[$\boldsymbol{P_{c}}$] = peso dos cabos
	\item[$\boldsymbol{C}$] = comprimento do cabo
	\item[$\boldsymbol{m}$] = massa por unidade de comprimento
	\item[$\boldsymbol{g}$] = aceleração da gravidade
	\item[$\boldsymbol{n_{c}}$] = número de cabos
\end{description}

O cabo de alimentação energética tem uma massa aproximada de 50 gramas por metro, como esse cabo não pode sofrer esforço mecânico, ele nunca  ficará tensionado e para isso seu comprimento deverá ser maior que o comprimento dos outros cabos. Para o cálculo do peso desse cabo seu comprimento será de 25 metros, procedendo da mesma forma no cálculo dos outros cabos temos: $25 \cdot 0.05 \cdot 9.8 = 13 N$. Portanto o peso total dos cabos será na faixa de 100 N.

Ainda existem peças que não foram levadas em consideração nessa conta como os mosquetões, sistema de estabilização entre outros. Admiti-se que essa carga extra não ultrapasse o valor de 20 N. Por margem de segurança usaremos um envelope com 60 micrometros de espessura.

A tabela \ref{tab:forcasAtuantes} mostra todos os valores de forças verticais atuantes no balão. Para o cálculo do empuxo considerou-se $\rho_{ar}$ = $1.11166 kg/m^3$, que é a densidade do ar a 1000m de altitude. Considerou-se $\rho_{h} = 0.138 \rho_{ar}$, que é a massa específica do hélio. A tabela \ref{tab:forcasAtuantes} apresenta todas as forças atuantes no balão.

\begin{table}[htp]
	\centering
	\caption[Forças Atuantes no Balão]{Forças Atuantes no Balão. Volume de $113.1 m^2$}
	\begin{tabular}{|c|c|}
		\hline
		\rowcolor[HTML]{FFFFFF}
		{\color[HTML]{000000} \textbf{Forças}} & {\color[HTML]{000000} \textbf{Valores Numéricos (N)}} \\ \hline
		Empuxo                                 & 1232.1                                                \\ \hline
		Peso do Hélio                          & 170                                                   \\ \hline
		Peso do material do envelope           & 62.5                                                  \\ \hline
		Peso dos Cabos                         & 100                                                   \\ \hline
		Peso Extra                             & 20                                                    \\ \hline
		Peso da Payload                        & 100                                                   \\ \hline
		Empuxo Líquido                         & 780                                                   \\ \hline
	\end{tabular}
	\label{tab:forcasAtuantes}
\end{table}


\subsection{Forças no Balão} % (fold)
\label{sub:for_as_no_bal_o}

De acordo com \citeauthoronline{yajima} (\citeyear{yajima}), a força \textbf{F} que atua no balão devido ao vento relativo tem duas componentes: A força de arrasto, que atua paralelamente a direção do vento relativo, e a força lateral, que atua perpendicular à direção do vento.

Tais forças são descritas pelas equações \eqref{eqn:forcaArrasto} \eqref{eqn:forcaLateral}.

	\begin{equacao}
	\caption[Força de arrasto]{Força de Arrasto~\cite{yajima}}
		\begin{equation}
			F_{d} = \frac{1}{2} \rho_{ar} \left | v_{w} - v_{b} \right |^{2} C_{d}A{b}
		\end{equation}
		\label{eqn:forcaArrasto}
	\end{equacao}

	\begin{equacao}
	\caption[Força lateral]{Força Lateral~\cite{yajima}}
		\begin{equation}
			F_{y} = \frac{1}{2} \rho_{ar} \left | v_{w} - v_{b} \right |^{2} C_{y}A{b}
		\end{equation}
		\label{eqn:forcaLateral}
	\end{equacao}

	Em que:
	\begin{description}
		\item[$\boldsymbol{F_{d}}$] = Força de Arrasto;
		\item[$\boldsymbol{F_{y}}$] = Força Lateral;
		\item[$\boldsymbol{\rho_{a}}$] = Densidade do Ar;
		\item[$\boldsymbol{C_{d}}$] = Coeficiente de Arrasto;
		\item[$\boldsymbol{C_{y}}$] = coeficiente de força lateral efetiva;
		\item[$\boldsymbol{A_{b}}$] = Area Frontal.
	\end{description}

	Considerando que o balão de monitoramento deve permanecer parado, podemos considerar que a velocidade do balão é nula, e a força de arrasto e força lateral se darão apenas em função da velocidade do vento.

	Para o calcula da força de arrasto e força lateral então é necessário conhecer a velocidade do vento que incidirá sobre o balão. Para determinar a velocidade do vento entramos no Banco de Dados Histórico do Instituto Nacional de Meteorologia~\cite{inmet}. Para termos um valor que nos desse uma margem de segurança analisamos os dados de 01/01/2005 a 01/01/2015. No período avaliado a maior velocidade do vento ocorrida no DF registrada foi de 17 m/s.

	O Coeficiente de Arrasto e o coeficiente de força lateral está diretamente relacionado ao número de Reynolds. O número de Reynolds é um número adimensional usado em mecânica dos fluidos para o cálculo do regime de escoamento de determinado fluido dentro de um tubo ou sobre uma superfície.~\cite{bird}. Segundo \citeauthoronline{yajima} (\citeyear{yajima}), o número de Reynolds para um balão é dado pela equação \eqref{eqn:numeroReynolds}.

	\begin{equacao}
	\caption{Número de Reynolds para balão}
		\begin{equation}
			Re_{b} = \frac{\rho_{ar} D_{b} \mid V_{w} - V_{b} \mid}{\mu_{a}}
		\end{equation}
		\label{eqn:numeroReynolds}
	\end{equacao}

	Em que:
	\begin{description}
		\item[$\boldsymbol{Re_{b}}$] = Número de Reynolds;
		\item[$\boldsymbol{D_{b}}$] = Diâmetro do Balão;
		\item[$\boldsymbol{V_{w}}$] = Velocidade do Vento;
		\item[$\boldsymbol{V_{b}}$] = Velocidade do Balão;
		\item[$\boldsymbol{\mu_{a}}$] = coeficiente de viscosidade do Ar .
	\end{description}

	Considerando que o balão terá um diâmetro de 6m e a Brasília se encontra a uma altitude próxima a 1000m acima do nível do Mar. Temos que:

	\begin{description}
		\item[Densidade do Ar a 1km de altitude:] $\rho_{a}$ = $1,11166 Kg/m^3$ segundo a tabela de atmosfera padrão~\cite{bird};
		\item[Diâmetro do Balão:] 6m;
		\item[Velocidade máxima do vento observada:] 17 m/s;
		\item[Viscosidade do Ar:] 0.01813 mPa.s~\cite{bird}.
	\end{description}

	Logo, obtemos a equação \eqref{eqn:numeroReynoldsBalai}.

	\begin{equacao}
	\caption{Número de Reynolds para balão}
		\begin{equation}
			Re_{e} = \frac{1.11166 kg/m^3 \cdot 6m \cdot 17 m/s}{0.01813 \cdot 10^{-3} Pa.s} \implies Re_{b} = 6.354 \cdot 10^6
		\end{equation}
		\label{eqn:numeroReynoldsBalai}
	\end{equacao}

	A figura \ref{img:coeficienteArrasto} mostra o coeficiente de arrasto do balão em função do número de Reynolds do fluido e do formato do balão. Para o número de Reynolds da ordem de 6 x 106  temos um coeficiente de arrasto de aproximadamente 0.2.

	\begin{figure}[htp]
		\centering
		\caption[Coeficiente de arrasto do balão em função do número de Reynolds do fluido e do formato do balão]{Coeficiente de arrasto do balão em função do número de Reynolds~\cite{myers}}
		\includegraphics[width=0.5\textwidth]{figuras/coeficienteArrasto}
		\label{img:coeficienteArrasto}
	\end{figure}

	Tendo em mão esses valores podemos então calcular a força máxima de arrasto que o balão estará sujeito a partir da \eqref{eqn:forcaArrasto}, obtendo a equação \eqref{eqn:forcaArrastoMax}.

	\begin{equacao}[htp]
	\caption{Força máxima de arrasto que o balão estará sujeito}
		\begin{equation}
			F_{D} = \frac{1}{2} \cdot 1.11166 \frac{Kg}{m^{3}} \cdot \left | 17m/s \right |^{2} \cdot 0.2 \cdot 28.273m^{2} \implies F_{D} = 908.2 N
		\end{equation}
		\label{eqn:forcaArrastoMax}
	\end{equacao}

	O coeficiente de força lateral efetiva para um balão que se encontra estacionário é de 0.18~\cite{ferguson}. Podemos observar que nas duas equações este é o único valor que se altera, logo temos que:

	\begin{equacao}[htp]
	\caption{Força máxima lateral que o balão estará sujeito}
		\begin{equation}
			F_y = 817.38 N
		\end{equation}
		\label{eqn:forcaLateralMax}
	\end{equacao}

	Temos então que sobre o balão atuaram as seguintes forças:

	\begin{description}
		\item[Força de arrasto:] 908.2 N
		\item[Força lateral:] 817.38 N
		\item[Força de empuxo:] 800 N
	\end{description}

	O que dá origem a equação \eqref{eqn:forcaResultante}.

	\begin{equacao}
	\caption{Força resultante}
		\begin{equation}
			F_D = \sqrt{908.2^2 + 817.38^2 + 800^2} \implies F_D = 1460.458 N
		\end{equation}
		\label{eqn:forcaResultante}
	\end{equacao}

\subsection{Posicionamento dos cabos}

	A posição em relação ao balão a qual os cabos estarão fixados possui grande importância, pois o mal posicionamentos deles poderá levar o balão a uma condição de instabilidade, logo prejudicando o monitoramento. O balão será fixado por três cabos, cuja  configuração é mostrada pela figura \ref{img:coordcabos}.

	\begin{figure}[htp]
		\centering
		\caption[Coordenadas do cabo]{Coordenadas do cabo~\cite{beer}}
		\includegraphics[width=0.8\textwidth]{figuras/coorcabo}
		\label{img:coordcabos}
	\end{figure}

	Serão utilizados cinco balões, que serão posicionados de acordo com a seguinte configuração: dois balões estarão em cima da Unidade de Ensino e Docência (UED) e outros dois balões no terraço na Unidade Acadêmica (UAC). Para estes balões, dois cabos estarão fixados no topo do prédio (pontos B e C na figura) e um terceiro cabo será preso numa haste (pontos C) de 10 m de altura que ficará a uma determinada distância dos prédios. O quinto balão ficará posicionado em cima do restaurante universitário (RU), onde dois cabos serão fixados em cima do RU (pontos B e D) e o terceiro cabo será preso a uma haste (ponto C) que possui a altura do restaurante universitário.

	Para que o balão fique em uma posição de equilíbrio o sistema de equações \eqref{eq:sustentacaoCabos}.

\begin{equacao}
	\caption{Sistema de sustentação dos cabos}
	\label{eq:sustentacaoCabos}
	\begin{equation}
		\begin{bmatrix}
			-\frac{X_b}{L_B}T_{AC} & -\frac{Y-Y_p}{L_B}T_{AB} & 0\\
			-\frac{X_c}{L_C}T_{AB} & -\frac{Y-Y_h}{L_C}T_{AC} & \frac{Z_c}{L_C}T_{AC}\\
			0 & -\frac{Y-Y_p}{L_D}T_{AD} & -\frac{Z_d}{L_D}T_{AD}
		\end{bmatrix}
		\begin{bmatrix}
			i\\
			j\\
			k
		\end{bmatrix}
		 =
		\begin{bmatrix}
			F_{dx}\\
			E_L\\
			F_{dz}
		\end{bmatrix}
	\end{equation}
\end{equacao}



	Em que:
	\begin{description}
		\item[$\boldsymbol{L_{B}}$] = Comprimento do cabo do ponto A até o ponto B
		\item[$\boldsymbol{L_{C}}$] = Comprimento do cabo do ponto A até o ponto C
		\item[$\boldsymbol{L_{D}}$] = Comprimento do cabo do ponto A até o ponto D
		\item[$\boldsymbol{E_{L}}$] = Empuxo líquido
		\item[$\boldsymbol{F_{D}}$] = Força de arrasto
		\item[$\boldsymbol{F_{dx}}$] = Componente - x da força aerodinâmica
		\item[$\boldsymbol{F_{dz}}$] = Componente - z da força aerodinâmica
		\item[$\boldsymbol{Y}$] = Altura do balão
		\item[$\boldsymbol{Y_{p}}$] = Altura do prédio
		\item[$\boldsymbol{Y_{h}}$] = Altura da haste
	\end{description}

A melhor abordagem para resolver este sistema é definir aonde estarão fixados os cabos ($x_{b}$, $x_{c}$, $z_{c}$, $z_{d}$) e estabelecer o comprimento de cada cabo ($L_{B}$, $L_{C}$, $L_{D}$), logo as incógnitas serão as forças em cada cabo ($T_{AB}$, $T_{AC}$, $T_{AD}$). Portanto, deve-se escolher as posições dos cabos que forneçam valores de forças fisicamente possíveis. Outro ponto importante é que a força do cabo que é fixado na haste seja bem inferior em relação as forças dos cabos preso aos prédios, logo será necessário o uso de uma haste com simples resistência a tração, ao invés de uma haste com boas propriedades mecânica no então de custo elevado. A disposição dos cabos pode ser observada na tabela \ref{tab:composangcabos}.

\begin{table}[htp]
\centering
\caption{Posição, comprimento e ângulo dos cabos}
\begin{tabular}{l|l|l|l|l|l|l|l|l|l|l|}
\cline{2-11}
 & $X_{b}$ m & $X_{c}$ m & $Z_{c}$ m & $Z_{d}$ m & $L_{b}$ m & $L_{c}$ m & $L_{D}$ m & $\Theta _{B}$ & $\Theta _{C}$ & $\Theta _{D}$ \\ \hline
\multicolumn{1}{|l|}{UED} & 18,0 & 5,0 & 10.0 & 18,0 & 23.5 & 18,7 & 23.5 & 39,8 & 45,0 & 39,8 \\ \hline
\multicolumn{1}{|l|}{UAC} & 18,0 & 5,0 & 10.0 & 18,0 & 23.5 & 18,7 & 23.5 & 39,8 & 45,0 & 39,8 \\ \hline
\multicolumn{1}{|l|}{RU} & 25,0 & 5,0 & 15,0 & 25,0 & 32,7 & 26,3 & 32,7 & 40.3 & 46,4 & 40.3 \\ \hline
\end{tabular}
\label{tab:composangcabos}
\end{table}

	Para as condições de projeto:

	O empuxo líquido, $E_{L}$, foi calculado anteriormente, o seu valor é de 800 N. A forca de arrasto, $F_{D}$ é de aproximadamente 880 N. Portando a força total exercida nos cabos é de aproximadamente 1200 N. Assume-se que a força de arrasto é dividida igualmente entre a componente x e z: $F_{dx}$ = 440 N e $F_{dz}$ = 440 N.

	\begin{itemize}
		\item Altura do balão = 10 m
 		\item Altura do UED = 10 m
 		\item Altura do UAC = 10 m
 		\item Altura do RU = 4 m
 		\item Altura da haste = 10 m
	\end{itemize}

Portanto, para os balões presentes no UED e UAC, o cabo fixado na haste ficará aproximadamente a 12 metros do prédio, e o cabos que estão presos no terraço estão a aproximadamente 18 metros do balão. Para o balão posicionado no RU, o cabo na haste estará a 16 metros do prédio, e os cabos fixados no topo do prédio a 25 metros do balão. As tensões presentes estão dispostas na tabela \ref{table:modTensoes}.

\begin{table}[htp]
\centering
\caption{ Módulo das tensões}
\begin{tabular}{l|l|l|l|}
\cline{2-4}
 & $T_{AB}$ [N] & $T_{AC}$ [N] & $T_{AD} [N]$ \\ \hline
\multicolumn{1}{|l|}{UED} & 588,53 & 45,35 & 604,31 \\ \hline
\multicolumn{1}{|l|}{UAC} & 588,53 & 45,35 & 604,31 \\ \hline
\multicolumn{1}{|l|}{RU} & 585,13 & 42,28 & 606,14 \\ \hline
\end{tabular}
\label{table:modTensoes}
\end{table}

A figura \ref{img:pos-bal} mostra como deve ser posicionado um dos balões, nota-se que dois pontos de sustentação estão no teto do prédio e o terceiro ponto é um poste. Os dois pontos em cima do prédio possuem motores para puxar o balão para baixo e controlar sua altitude.

\begin{figure}[htp]
	\centering
	\caption{Posicionamento de um dos balões}
	\includegraphics[width=0.8\textwidth]{figuras/pos-bal}
	\label{img:pos-bal}
\end{figure}

\subsection{Sustentação do SUM}

O Balão Cativo será fixado em três pontos diferentes (figura \ref{img:catiaPayload}) para que seja garantida sua estabilidade considerando ventos laterais de qualquer direção, adotou-se o modelo de balão de zero pressão de formato esférico preenchido com gás hélio. No caso dos balões de zero pressão há uma perda diária do volume de gás de cerca de 8\%. O volume de gás hélio no balão será de 113m$^{3}$ e, considerando as perdas diárias, estima-se que a cada sete dias haverá a necessidade de  efetuar uma reposição desse gás para garantir o bom funcionamento do monitoramento do estacionamento, onde o empuxo gerado pelo gás seja suficiente para manter uma altitude de 20m.

	Na figura \ref{img:catia1} está o protótipo do SUM. Nota-se que há uma fita de carga que passa no equador do envelope que segura a payload. A figura \ref{img:catiaPayload} mostra como a payload deve ser amarrada ao balão. A figura \ref{img:catia2} mostra o detalhe do trilho na parte superior da payload, com esse trilho a payload pode se mover sem que haja perda no foco da imagem.

	\begin{figure}[htp]
		\centering
		\caption{Protótipo do SUM no Catia.}
		\includegraphics[width=0.6\textwidth]{figuras/catia1}
		\label{img:catia1}
	\end{figure}

	\begin{figure}[htp]
		\centering
		\caption{Protótipo da payload do SUM no Catia}
		\includegraphics[width=0.6\textwidth]{figuras/catiadopayload}
		\label{img:catiaPayload}
	\end{figure}

	\begin{figure}[htp]
		\centering
		\caption{Detalhe do trilho da payload.}
		\includegraphics[width=0.6\textwidth]{figuras/catiadafixacao}
		\label{img:catia2}
	\end{figure}

A fixação será adaptada para cada um dos balões, pois como estarão em lugares distintos e estratégicos estudaremos cada um dos três pontos de fixação. A manutenção do balão será feita em solo, por exemplo: no caso de reposição do gás, na reparação de algum item interno da payload e etc. Para tal iremos utilizar dois motores elétricos para puxar o balão até o solo, os motores estarão fixos no chão onde será definido o local onde o cabo de sustentação será fixado, ou seja, o cabo de sustentação estará sendo regulado pelo motor.

De acordo com cálculos realizados previamente, a força resultante do balão será de aproximadamente de 6.4 KN. A partir destas informações, escolhemos o motor vendido pela empresa Bremen com a capacidade de 7240 Kg, potência de 4500W, 12V de tensão e pesando 52,5 Kg. A figura \ref{img:tabelamotor} mostra as informações referentes aos motores vendidos pela mesma empresa.

\begin{figure}[htp]
	\centering
	\caption[Tabela ilustrativa dos tipos de motores disponíveis]{Tabela ilustrativa dos tipos de motores disponíveis~\cite{bremem}}
	\includegraphics[width=0.8\textwidth]{figuras/tabelamotor}
	\label{img:tabelamotor}
\end{figure}

\begin{figure}[htp]
	\centering
	\caption[Modelo de motor elétrico escolhido]{Modelo de motor elétrico escolhido~\cite{bremem}}
	\includegraphics[width=0.8\textwidth]{figuras/modelodemotoreletrico}
	\label{img:motorescolhido}
\end{figure}

A figura \ref{img:motorescolhido} ilustra como será o motor. Para este motor será efetuada uma simples adaptação no cabo, onde será substituído o cabo de fábrica pelo cabo de aço classe 8x19 – Alma de Fibra com diâmetro de 6,4 mm. A alteração foi necessária em consideração ao peso do cabo, pois o mesmo é mais leve em relação ao cabo de aço comum, e também possui boas propriedades mecânicas como mostrado na figura \ref{img:motorescolhido}.

\begin{figure}[htp]
	\centering
	\caption[Propriedade do cabo Alma de Fibra]{Propriedade do cabo Alma de Fibra~\cite{acrocabo}}
	\includegraphics[width=0.8\textwidth]{figuras/tabelacabo}
	\label{img:motorescolhido}
\end{figure}

\subsection{Simulações na estrutura da payload}

	A estrutura interna da payload é dividida em duas superfícies e barras que conectam essas superfícies, como mostrado na figura \ref{img:payload3}.

	\begin{figure}[htp]
		\centering
		\caption{Estrutura interna da payload}
		\includegraphics[width=0.8\textwidth]{figuras/payload3}
		\label{img:payload3}
	\end{figure}

	Deve-se analisar os pontos de maior deformação e concentração de tensão na estrutura da payload para que os equipamentos eletrônicos não sejam perdidos no meio da operação  do SUM. Simulações feitas no software CATIA V5R19 são agora analisadas.

	A figura \ref{img:maior-tensao} mostra onde a tensão está concentrada na payload, as barras de apoio entre as faces que “tampam” a payload também conectam os cabos ao envelope do balão, desta forma essas barras estão sempre tracionadas.

	\begin{figure}[htp]
		\centering
		\caption{Pontos de maior tensão na payload}
		\includegraphics[width=0.8\textwidth]{figuras/maior-tensao}
		\label{img:maior-tensao}
	\end{figure}

	Os pontos passíveis de ocorrer deformação são mostrados na figura \ref{img:pontos-def}. A superfície em que o envelope é conectado à payload poderá sofrer maior, enquanto que as barras que sofrem tração na figura \ref{img:maior-tensao} podem se romper próximos a essa superfície.

	\begin{figure}[htp]
		\centering
		\caption{Pontos de possível deformação}
		\includegraphics[width=0.8\textwidth]{figuras/pontos-def}
		\label{img:pontos-def}
	\end{figure}

	Portanto, durante as manutenções deve-se verificar se essas áreas ainda estão aptas para o desempenho de suas funções.


\subsection{Mudanças no projeto do balão}

O cronograma previsto para o ponto de controle 3 previa apenas alguns ajustes pontuais na escolha dos materiais e a pesquisa de preço, porém houve um aspecto na apresentação que foi questionado pelos avaliadores com relação a vazão em massa do gás, as suposições feitas para esse cálculo estariam equivocadas. Para solucionar esse problema, foi proposto a utilização de sensores para medir o empuxo líquido do balão, porém para empregar esses sensores teríamos de calibra-los para saber o comportamento do balão.

O que foi discutido entre o subgurpo de estrutura do balão foi a questão de não haver tempo viável para a pesquisa desses sensores e a calibração dos mesmos, sem contar que  as adaptações no projeto poderiam levar a mudança de alguns materiais já pesquisados, e consequentemente, as contas referentes a sustentação do balão no solo também sofreriam modificações. Visando diminuir o impacto dessas mudanças no projeto, foi proposto por membros do grupo que mudássemos o sistema do balão de Zero Pressão para Super Pressão, sem que houvesse mudança na massa total do sistema para não alterar toda a cadeia de cálculos. As próximas seções apresentam como chegamos a conclusão de que a mudança de sistema é viável para o projeto.

A pesquisa de mercado também não encontrou os materiais para o envelope do balão ideais, desta forma houve uma mudança no material do balão do ponto de controle 2 para o ponto de controle 3. Esse tópico será melhor abordado nas próximas seções.

\subsubsection{Novo dimensionamento do balão}

Os cálculos anteriores chegaram aos seguintes valores:
\begin{itemize}
	\item Volume do balão: 113,1 $m^3$
	\item Massa de hélio: 17 kg
\end{itemize}

A mudança desses dois valores acarretaria na mudança da maioria dos cálculos já realizados, especialmente o volume do balão, que afeta a força de empuxo. A mudança no valor da massa de hélio não é um fator tão importante quanto o volume, pois a massa de hélio é pequena se comparada com o resto da estrutura.

Considerando que o PVC pneumático possui resistência a tração de 40 Mpa, o Polietileno Linear de baixa densidade (PELBD) possui uma resistência à tração de 37 MPa, o Polietileno de baixa densidade (PEBD) possui resistência à tração de 24 MPa, e adotando um fator de segurança de 150, pois qualquer dos materiais escolhidos vão trabalhar em condições adversas de temperatura, chega-se a conclusão de que a pressão interna do balão não deve ultrapassar 240 kPa (1.6 atm). Portanto, adotou-se o valor de 1.5 atm como a pressão máxima permitida para o balão.

Usando a equação dos gases ideias,considerando o hélio como um gás ideal e admitindo uma temperatura de 39$^{\circ}$ (312.15 K), que é uma das temperaturas mais altas registradas na cidade do Gama,  pode-se calcular a massa de hélio para sabermos um valor de referência.

\begin{equation}
	PV = nRT \Rightarrow n =  \frac{113100\cdot1.5}{0.082\cdot312.15} \Rightarrow n = 6.63\cdot10^3 mols
	\label{gasideal}
\end{equation}

onde:

\begin{itemize}
	\item \textbf{P} é a pressão,
	\item \textbf{V} é o volume,
	\item \textbf{n} é o número de mols,
	\item \textbf{R} é a constante universal dos gases (0.082 $atm\cdot l/mol\cdot K$), e
	\item \textbf{T} é a temperatura.
\end{itemize}

Como a massa total é o número de mols multiplicado pela massa molar, chega-se ao valor de 26.5 kg de hélio.

Como a alteração do valor da massa do hélio não é muito grande não haverá mudanças com relação ao sistema de sustentação balão.

\subsubsection{O material}

O Balão \textit{Super-Pressão} funciona de tal forma que começará a descender quando sua pressão interna diminuir a um nível menor que a pressão externa. Um balão perde gás naturalmente em decorrência da passagem de moléculas do gás pelo material do envelope. Escolhemos o material do balão como sendo PVC pneumático que apresenta uma permeabilidade de  5$cm^3/ml\cdot24h\cdot atm$   valor que relaciona o volume de  perda do material pelo volume que ele suporta e a pressão em seu interior a cada 24h.

Temos um balão de volume 113.1$m^3$ que suporta uma pressão de 1.5 atm, isso nos dá uma perda de 0.3 atm por dia.

Temos que determinar o tempo que leva para a pressão no interior do balão atingir 1 atm considerando que seu volume permanece constante. Utilizando a equação geral dos gases perfeitos, equação \label{gasideal}, temos que a variação de pressão no balão considerando uma variação de 0,3$m^3$ por dia é de $3.87\cdot10^{-3}$ atm/dia.

A quantidade de dias necessários para se perder 0.5 atm e atingir o nível onde o balão começa a descer é de aproximadamente 129 dias, o que nos da um perda total de 38.7$m^3$ de gás. Ressalta-se que o volume do balão permanece inalterado desde que a pressão interna do balão seja maior que a pressão externa, desta forma o empuxo líquido não varia se a manutenção for feita no período estipulado.

A massa específica do PVC é de 1390$kg/m^3$, o envelope balão terá uma área superficial de 113$m^2$ e uma espessura de 0.23 mm. Desta forma a massa total desse material será de aproximadamente de 36.2 kg.


\subsection{Peso total da estrutura}
Com as mudanças no material e na quantidade de gás o peso total da estrutura também será alterado. As alterações são mostradas na tabela \ref{tab:forcanova}.

\begin{table}[htp]
	\centering
	\caption{Alteração nas forças atuantes na estrutura}
	\begin{tabular}{ | c || c | c |}
		\hline
		\textbf{Força} &\textbf{Anterior} &\textbf{Atual} \\ \hline
		\textbf{Empuxo} & 1232.1 N & 1232.1 N \\ \hline
		\textbf{Peso do Hélio}  & 170 N& 260.7 N \\\hline
		\textbf{Peso do envelope} & 62.5 N&  354.4 N\\ \hline
		\textbf{Peso dos cabos} & 100 N &  100 N \\\hline
		\textbf{Peso Extra} & 20 N &  20 N \\\hline
		\textbf{Peso da \textit{Payload}} & 100 N &  100 N \\\hline
		\textbf{Empuxo Líquido} & 780 N &  397 N \\
		 \hline
	\end{tabular}
	\label{tab:forcanova}
\end{table}

O peso dos cabos, da payload e outros componentes não sofreram alteração. Com a tabela \ref{tab:forcanova} percebe-se que a maior mudança foi no empuxo líquido, mas essa alteração não influencia nos cálculos feitos para o esforço nos cabos, visto que eles já conseguiam sustentar a estrutura antes da alteração.

\subsection{Preços}

\hspace{1cm}

\begin{enumerate}

	\item \textbf{POSTE}

		As empresas POSTEFER, REPUME iluminações e TROPICO foram procuradas para solicitação de orçamento do poste para fixação do cabo. Somente a empresa REPUME respondeu a solicitação. Os produtos são os seguintes:

		\begin{itemize}
			\item Poste Cônico Contínuo Reto Flangeado, fabricado em chapa de aço SAE 1010/1020. Com 10 metros de altura útil. Diâmetro de base 170 mm, diâmetro de topo 60,3 mm.
			\item Chumbador para a fixação de flange do poste 3/4" X 500 X 70 mm com porcas e arruelas.
		\end{itemize}

		Os preços são mostrados na tabela \ref{tab:precoposte}. São 4 postes de fixação de apoio para o balão e 16 pontos de fixação.

		\begin{table}[htp]
			\centering
			\caption{Preços do poste (Fixação)}
			\begin{tabular}{ | c | c | c | c |}
				\hline
				\textbf{Qdt} &\textbf{Peça} &\textbf{Preço unitário} &\textbf{Preço total}\\ \hline
				4 & Poste & R\$ 1.389,00 & R\$ 5.556,36 \\ \hline
				16 & Chumbador & R\$ 21,41 & R\$ 342,56 \\
				 \hline
			\end{tabular}
			\label{tab:precoposte}
		\end{table}

		Em relação aos cabos de fixação, serão necessários 100 metros do cabo de aço classe 8x19 alma de Fibra com diâmetro de 6.4 mm, como já foi apresentado. As empresas  A.CAMARGO, CABLEMAX,  SIVA forneçem o produto, o preço médio do metro é de  R\$ 5,00, portanto o preço total é R\$ 480,00.

		\hspace{1cm}

	\item \textbf{MOTOR}

		A tabela \ref{tab:precomotor} mostra a quantidade de cada material que cada balão vai precisar para ser fixado no teto do prédio.

		\begin{table}[htp]
			\centering
			\caption{Preços do restante dos materiais da sustentação (Fixação)}
			\begin{tabular}{ | c | c |}
				\hline
				\textbf{Quantidade} &\textbf{Item} \\ \hline
				1 &Guincho Winch 8500 lbs / 3.850Kg com cabo de aço com 28 metros \\ \hline
				1 &Controle remoto com cabo de 3 metros \\ \hline
				1 &Gancho\\ \hline
				1 &Rolete guia\\ \hline
				1 &Caixa de solenoide\\
				 \hline
			\end{tabular}
			\label{tab:precomotor}
		\end{table}

		O modelo do guincho é o seguinte: Guincho Winch 8.500lbs / 3.850Kg, linha profissional com cabo de aço com 28 metros c/ Controle remoto COM FIO + Gancho+ Rolete guia + Caixa de solenoide.

		Custo Unitário dos itens da tabela \ref{precomotor} excetuando o guincho : R\$1876,00 + R\$ 118,92 = R\$1994,92. Os R\$ 118,92 são referentes ao frete que tem o prazo de 5 dias. Como serão necessários mais de um motor para cada balão, caso a compra seja no atacado cada um é vendido a R\$1700,00. Cada balão precisará de 2 conjuntos dos itens da tabela \ref{precomotor}. O custo total para os 5 balões é de R\$36.949,20. Os preços foram obtidos a partir de pesquisa e comparação entre concorrentes, escolhendo a empresa \cite{precoMotor}.

		\hspace{2cm}
	\item  \textbf{GÁS HÉLIO}

		Dentre as distribuidoras de gás hélio no Distrito Federal analisamos a concorrência entre a Oxigmed Gases e Equipamentos e a Praxair. Optamos por utilizar cilindros de 10$m^3$ de hélio para inflarmos o balão. A tabela \ref{precogas} mostra o preço do gás em duas fornecedoras.

		\begin{table}[htp]
			\centering
			\caption{Preço do gás Hélio}
			\begin{tabular}{ | c | c | c |}
				\hline
				\textbf{Fornecedor} &\textbf{Preço unitário do cilíndro} &\textbf{Preço total}\\ \hline
				Oxigmed Gases e Equipamentos & R\$ 1.650,00 & R\$ 19.800,00 \\ \hline
				Praxair & R\$ 1.330,00 & R\$ 15.960,00 \\
				 \hline
			\end{tabular}
			\label{precogas}
		\end{table}

		Como o balão tem um volume de 113,1 $m^3$ serão necessários aproximadamente 12 cilindros para inflar o balão.

		Optamos pela distribuidora Praxair como fornecedora de gás hélio em decorrência do menor preço e confiabilidade. Como teremos um total de 5 balões será necessário um valor de R\$ 79.800,00.

		A empresa Park e Ação faz a recarga de cilindros vazios. O preço para recarregar 2$m^3$ de hélio é de R\$ 200,00.

		Considerando o preço do volume de gás hélio como R\$ 100,00 o metro cúbico e arredondado o valor da reposição para 40$m^3$ de gás por balão será necessário repor 200$m^3$ por R\$100,00 o metro cúbico de gás ao final de 4 meses. O que nos dá um valor de R\$ 20.000,00 a cada 4 meses. Desta forma, o gás do balão terá um custo médio mensal de R\$5.000,00.

		\begin{itemize}
			\item Custo inicial do gás: R\$  79.800,00
			\item Custo mensal do gás:  R\$5.000,00.
		\end{itemize}

		Estes valores foram obtidos atraves de pesquisa e seleção de fornecedor, escolhendo as empresas \cite{precoGasHelio} e \cite{oxigmed}.

		\hspace{1cm}

	\item  \textbf{Material do envelope}

		A pesquisa apontou que dois materiais poderiam ser usados para o confeccionar o envelope do balão, o polietileno linear de baixa densidade (PELBD) e o polietileno de baixa densidade (PEBD), porém na pesquisa de mercado não foi encontrado uma empresa que fornecessem e/ou confeccionassem o envelope.

		Uma saída para a solução desse imprevisto foi a comprar de balões inflãveis do tipo Blimp como mostrado na figura \ref{img:blimp2}, esses balões são feitos com PVC pneumático.

		\begin{figure}[htp]
			\centering
			\caption{Exemplo de balão \textit{blimp} usado em publicidade. Fonte: OLX}
			\includegraphics[width=8cm, height=5cm]{figuras/blimp}
			\label{img:blimp2}
		\end{figure}

		O preço desse tipo de balão é por volta de R\$ 1000,00, dependendo do tamanho do balão. A empresa Fly Balloon foi solicitada via-email para o orçamento de um balão Blimp esférico de 3 metros de raio. As informações constam na tabela \ref{tab:precoblimp}.

		\begin{table}[htp]
			\centering
			\caption{Informações sobre o balão}
			\begin{tabular}{ | c | c |}
				\hline
				\textbf{Característica} &\textbf{Valor numérico} \\ \hline
				Raio & 3 metros  \\ \hline
				Material  & PVC pneumático \\
				 \hline
				Preço  & R\$ 1.100,00 \\
				 \hline
			\end{tabular}
			\label{tab:precoblimp}
		\end{table}

		Como serão 5 balões, o valor total será de  R\$ 5.500,00. Este valores foram obtidos a partir de pesquisa na empresa \cite{balloon}.
\end{enumerate}

\subsubsection{Custo total da estrutura}

	Após realização de pesquisa e seleção de fornecedores, todos os valores foram compilados, chegando ao seguinte resultado:

\begin{itemize}
	\item Custo inicial total: R\$ 128.628,02
	\item Custo mensal total: R\$ 5.000,00
\end{itemize}

% section subprojeto_de_estrutura_e_sistema_a_reo (end)

\section{Subprojeto da Eletrônica Embarcada} % (fold)
\label{sec:subprojeto_da_eletr_nica_embarcada}
	Problema: O sistema de monitoramento SUM tem como uma de suas principais funções efetuar o reconhecimento de situações de risco e, para isso, deve-se ter uma qualidade elevada de captação de imagem. Para desempenhar esta função, faz-se necessário a utilização de uma câmera com um grande alcance e alta resolução, além de de componentes que ajudarão na estabilização da mesma. Por serem muitos fatores que podem prejudicar o bom funcionamento do nosso sistema, foram selecionados alguns sensores que auxiliarão na estabilização da Payload e por consequência na captação da imagem.

	A princípio, pensou-se nos principais fatores climáticos que poderiam influenciar de maneira negativa o funcionamento do sistema e com isso foi chegada à conclusão que a velocidade do vento, pressão do ar, temperatura ambiente e do sistema e a umidade são os fatores mais determinantes. A justificativa é bem simples, com a variação repentina da velocidade do vento, o payload poderá, eventualmente, se inclinar de tal forma a perder a visualização da área a ser monitorada. Com uma variação da pressão o balão poderia perder altitude e assim, como dito anteriormente, não conseguir captar as imagens do local a ser monitorado. Com relação à temperatura, por se tratar de componentes eletrônicos, se faz necessário o conhecimento a respeito da temperatura na qual eles estão operando, visto que, em certas temperaturas, os componentes podem funcionar incorretamente, como já previsto pelo fabricante. E por ultimo, mas não menos importante, o sensor de umidade será utilizado para saber quando o balão estará funcionando ou não, uma vez que, em dias com muita umidade, o balão não funcionará.

	O subprojeto de eletrônica embarcada do Payload é baseado no funcionamento de 3 microcontroladores: o Arduino UNO, o Raspberry PI 2 e o Intel Galileo Gen 2. A integração destes microcontroladores com os outros componentes está descrita na figura \ref{img:func-geral-ele}.

\begin{figure}[H]
\centering
\caption{Funcionamento Geral Eletrônica Embarcada}
\includegraphics[scale=0.5]{figuras/eletronica}
\label{img:func-geral-ele}
\end{figure}

	No diagrama de funcionamento geral da eletrônica embarcada do projeto, estão presentes sensores, atuadores, câmeras e painéis de LEDs infravermelhos. A utilização destes componentes visa a qualidade na captação de imagens, sendo que cada um desempenhará um papel específico. Para garantir a qualidade na captação das imagens, o sistema SUM busca, com o uso de câmeras de alta resolução, reduzir qualquer interferência que possa prejudicar o funcionamento do sistema ou a análise de situações de risco.

	Um dos problemas que poderia ser prejudicial ao sistema é a mudança brusca da orientação do Payload causada, por exemplo, por uma rajada de vento. Essa mudança brusca pode mudar a direção para qual a câmera estava focando, causando a perda de cenas que, possivelmente, possam ser necessárias para uma análise de risco.

	Visando a garantia de que esse possível problema não venha interferir no funcionamento do sistema, tornou-se necessário o desenvolvimento de um sistema de estabilização do Payload. Neste sistema, estarão presentes os seguintes sensores  inerciais: LM35(Temperatura), HMC5883L(Bússola), L3G4200D(movimento), ADXL345(acelerometro), BMP180(pressão) e DHT11(umidade). Estes sensores estarão conectados ao Arduino UNO que, por sua vez, interpretará estes dados e enviará essas informações para o Raspberry PI, por meio de uma comunicação serial.

	O Raspberry também estará conectado ao Aweek, um painel composto por vários LEDs para iluminação em infravermelho. Por fim, O Raspberry PI enviará informações para o microcontrolador Intel Galileo Gen 2, indicando a necessidade de estabilização da estrutura, de acordo com a interpretação feita pelo Arduino. Caso seja necessária a estabilização, o Galileo decidirá se ativará um motor de passo para realizar a estabilização através do trilho situado no payload, ou se ativará o atuador Reaction Wheel, em casos em que uma rajada de vento provoque a rotação do Payload.

	Outra preocupação do nosso sistema está relacionada com a transmissão das imagens até a central de monitoramento. O Raspberry PI será o responsável por transmitir as imagens em tempo real para a central. Ele receberá os dados da câmera, através de um cabo (flat) específico para câmera a Waveshare OV5647 Night Vision, e cada balão transmitirá as informações para o balão mais próximo da central. A comunicação entre eles será sem fio, montando uma rede intranet. O balão que receberá todas as informações, as transmitirá para a central através de um cabo de Ethernet e o computador receptor realizará todo o procedimento desejado com as imagens.

	A seguir será apresentado com maiores detalhes cada tópico da solução.

\subsection{Microcontroladores e Microprocessadores}

Os microcontroladores e microprocessadores serão responsáveis por integrar todas as atividades do sistema, seja a aquisição, armazenamento, transmissão de dados, obtidos por sensores ou câmeras, conversão de dados analógicos em digitais ou o controle do sistema. Essas atividades exigirão determinados requisitos, de acordo com a sua aplicação. Logo, vê-se a necessidade de especificar os microprocessadores e microcontroladores responsáveis por cada setor. A iniciativa de usar um microcontrolador é baseada no fato deste possuir diversos periféricos e um processador embutidos em um único circuito integrado. Esta característica minimiza o tamanho físico do projeto e facilita a implementação de várias aplicações \cite{prado2009implementaccao}.  Contudo, as CPUs dos microcontroladores são menos poderosas do que as dos microprocessadores, suas instruções, geralmente, se limitam às instruções mais simples, sua frequência de clock é menor e seu espaço de memória endereçável costuma ser menor \cite{rucinski}. Porém, estas desvantagens não interferem na viabilidade do projeto, visto que ao realizar a análise do seu custo e o comparativo entre os dois, pode-se perceber que o microprocessador possui um custo muito mais elevado, uma vez que existe a necessidade de implementação de periféricos que, geralmente, estão presentes em uma placa de um microcontrolador. Esta análise torna perceptível a escolha de microcontroladores para o projeto.

Em condições desfavoráveis como, por exemplo, um fluxo de ar repentino, a rápida estabilização do balão se mostra essencial para que se mantenha o foco em determinada região da qual as imagens estão sendo captadas, visando a garantia de que não haverá perdas de cenas, o que poderia interferir no reconhecimento de situações de risco. O setor responsável pelo controle de estabilização do payload exigirá uma frequência de clock muito alta, levando em consideração que esta estabilização deverá ser realizada rapidamente. Portanto, essa função será desempenhada pelo Intel Galileo Gen 2, pois este admite frequências de clock de até 400 MHz.

Para os setores de armazenamento e transmissão de imagens das câmeras, o Raspberry PI 2 foi considerado o ideal, dado sua eficiência em termos de processamento de dados. Outra vantagem de se utilizar o Raspberry é a sua compatibilidade com a linguagem Python, o que facilitará o desenvolvimento do algoritmo responsável pela compressão de vídeo.

Para o setor voltado para a captação de dados dos sensores, decidiu-se que o ideal seria utilizar o Arduino UNO, visto que este possui grande compatibilidade com os shields escolhidos e seu ambiente de desenvolvimento (IDE) também é compatível com o Intel Galileo Gen 2, o que facilitará a codificação do mesmo, além de uma quantidade razoável de portas disponíveis.

As especificações dos microcontroladores estão relacionadas na tabela \ref{table:microprocessadores}:

\begin{table}[H]
\centering
\caption[Especificações dos microcontroladores]{Especificações dos Microcontroladores~\cite{intelGalileo},\\\cite{rsppi}, \cite{arduino}}
\begin{tabular}{p{3cm}|p{3cm}|p{3cm}|p{3cm}|}
\cline{2-4}
 & Intel Galileo Gen 2 & Raspberry PI 2 & Arduino UNO \\ \hline
\multicolumn{1}{|l|}{Microcontrolador} & \multicolumn{1}{c|}{-} & \multicolumn{1}{c|}{-} & ATmega328 \\ \hline
\multicolumn{1}{|l|}{Processador} & SoC Quark X1000 - 32 bits & Broadcom BCM2836 SoC & \multicolumn{1}{c|}{-} \\ \hline
\multicolumn{1}{|l|}{Arquitetura} & x86 & Quad-core ARM Cortex-A7 & \multicolumn{1}{c|}{-} \\ \hline
\multicolumn{1}{|l|}{Memória} & DDR3 de 256 MB, SRAM embarcada de 512 KB, NOR Flash de 8 MB e EEPROM padrão de 8 KB on-board & 1 GB de RAM & 32K (0.5 usado pelo bootloader) \\ \hline
\multicolumn{1}{|l|}{Clock} & 400 MHz & 900 MHz & 16MHz \\ \hline
\multicolumn{1}{|l|}{GPU} & \multicolumn{1}{c|}{-} & VídeoCore IV & \multicolumn{1}{c|}{-} \\ \hline
\multicolumn{1}{|l|}{Portas analógicas} & 6 & \multicolumn{1}{c|}{-} & 6 \\ \hline
\multicolumn{1}{|l|}{Portas digitais} & 14 & 26 (GPIO) & 14 \\ \hline
\multicolumn{1}{|l|}{Portas PWM} & 6 (12-bit) & \multicolumn{1}{c|}{-} & 6 \\ \hline
\multicolumn{1}{|l|}{Tensão de operação} & 12 V & 5 V & 5 V \\ \hline
\multicolumn{1}{|l|}{Corrente máxima} & 2 A & 1 A & 40 mA \\ \hline
\multicolumn{1}{|l|}{Alimentação} & 7 - 15 V & 5 V & 7 -12 Vdc \\ \hline
\multicolumn{1}{|l|}{Interface Ethernet} & 10/100 Mbps & 10/100 Mbps & \multicolumn{1}{c|}{-} \\ \hline
\multicolumn{1}{|l|}{Saída de vídeo e Áudio} & \multicolumn{1}{c|}{-} & HDMI e Av & \multicolumn{1}{c|}{-} \\ \hline
\end{tabular}
\label{table:microprocessadores}
\end{table}

\subsection{Sensores}

O funcionamento dos sensores utilizados no Payload está baseado na coleta de informações necessárias para a estabilização o balão. Os sensores já possuem, internamente, um conversor de sinal analógico para digital e também um filtro para reduzir ruídos. O processo de funcionamento dos sensores se inicia com a coleta de dados analógicos e então os converte para digital, logo em seguida, é realizada uma filtragem dentro do próprio sensor, visto que na etapa de coleta são adquiridos ruídos. Posteriormente, as informações são passadas para uma memória e para um bloco onde serão controladas no sistema de comunicação I2C. O sistema de comunicação I2C possibilita utilizar, em um mesmo sistema, componentes de tecnologias construtivas diferentes sem que haja incompatibilidade e nem conflitos na comunicação. A transmissão da informação entre os dispositivos é feita através de dois fios, serial data DAS e serial clock SCL.

	Na figura \ref{img:funcionamentoI2c} é ilustrado o funcionamento da comunicação I2C.

\begin{figure}[htp]
	\centering
	\caption[Exemplo de funcionamento da comunicação I2C]{Exemplo de funcionamento da comunicação I2C~\cite{microcontrolandos}}
	\label{img:funcionamentoI2c}
	\includegraphics[width=0.8\textwidth]{figuras/1}
\end{figure}

Os dispositivos ligados em Inter IC possuem um endereço fixo (cada componente recebe um endereço específico), e podemos configurá-los para receber ou transmitir dados; dessa maneira eles podem ser classificados de várias formas, como: mestres (MASTER), escravos (SLAVE), entre outras.

Uma das vantagens do padrão I2C é que ele não fixa a velocidade de transmissão (freqüência), pois ela será determinada pelo circuito MASTER (transmissão do SCL).

A seguir, será explicado o funcionamento dos sensores utilizados no Payload e serão mostrados seus respectivos diagramas funcionais.

\pagebreak

\begin{itemize}
  \item Acelerômetro
		\begin{figure}[H]
			\centering
			\caption{Sensor ADXL345}
			\label{img:ADXL345}
			\includegraphics[scale=1.5]{figuras/ADXL345}
		\end{figure}

		O acelerômetro detecta a presença ou a falta de movimento pela comparação da aceleração em qualquer eixo com limiares definido pelo usuário e também possui um sensor de queda livre que informa se o dispositivo está caindo.

		Estas funções podem ser mapeadas individualmente para qualquer um dos dois pinos de saída de interrupção (INT1 OU INT2). Um sistema de memória integrado pode ser usado para armazenar dados e minimizar a atividade do processador e diminuir o consumo geral de energia do sistema. A figura \ref{img:acelerometro} descreve o seu funcionamento.

		\begin{figure}[H]
	    \centering
	    \caption[Diagrama funcional acelerômetro ADXL345]{Diagrama funcional acelerômetro ADXL345~\cite{acelerometro}}
	    \label{img:acelerometro}
	    \includegraphics[width=0.8\textwidth]{figuras/2}
	  \end{figure}

\pagebreak

  \item Barômetro
		\begin{figure}[H]
			\centering
			\caption{Sensor BMP180}
			\label{img:bpm180}
			\includegraphics[scale=0.15]{figuras/bpm180}
		\end{figure}

		O BMP180 (figura \ref{img:bpm180}) consiste em um sensor piezoresistivo, um conversor AD e uma unidade de controle com E2PROM e uma interface serial I2C. O sensor entrega o valor da pressão e temperatura, o E2PROM tem armazenado 176 bits para calibração de dados que são usados para compensar o deslocamento, temperatura e outros parâmetros do sensor. O sensor é projetado para ser conectado diretamente a um microcontrolador através da comunicação I2C. O seu funcionamento está descrito na figura \ref{img:barometro}.

		\begin{figure}[H]
	    \centering
	    \caption[Diagrama funcional Barômetro BMP085]{Diagrama funcional Barômetro BMP085~\cite{barometro}}
	    \label{img:barometro}
	    \includegraphics[width=0.7\textwidth]{figuras/3}
	  \end{figure}
	  
\pagebreak

  \item Giroscópio

		\begin{figure}[H]
			\centering
			\caption{Sensor L3G4200D}
			\label{img:L3G4200D}
			\includegraphics[scale=0.25]{figuras/L3G4200D}
		\end{figure}

		O L3G4200D (figura \ref{img:L3G4200D}) é um sensor de velocidade angular de três eixos de baixa potência capaz de proporcionar estabilidade e sensibilidade sobre a temperatura e tempo. Inclui uma interface IC capaz de proporcionar a velocidade angular medida para o mundo exterior através de uma interface digital (I2C ou SPI) Essa interface IC é fabricada usando um processo CMOS que permite um nível elevado de integração para a criação de um circuito. O seu funcionamento pode ser visualizado na figura \ref{img:giroscopio}.

	  \begin{figure}[H]
	    \centering
	    \caption[Diagrama Funcional Giroscópio L3G4200D]{Diagrama Funcional Giroscópio L3G4200D~\cite{giroscopio}}
	    \label{img:giroscopio}
	    \includegraphics[width=0.7\textwidth]{figuras/4}
	  \end{figure}

\pagebreak

  \item Magnetômetro

		\begin{figure}[H]
			\centering
			\caption{Sensor HMC5883L}
			\label{img:HMC5883L}
			\includegraphics[scale=0.2]{figuras/HMC5883L}
		\end{figure}

		O HMC5883L (figura \ref{img:HMC5883L}) é um composto por um trio de sensores e circuitos de suporte para aplicações específicas para medir campos magnéticos. Com uma fonte de alimentação aplicada, o sensor converte qualquer campo magnético incidente nas direções dos eixos sensíveis a uma saída de tensão diferencial. A figura \ref{img:magnetometro} descreve seu funcionamento.

	  \begin{figure}[H]
	    \centering
	    \caption[Diagrama funcional magnetômetro HMC5883L]{Diagrama funcional magnetômetro HMC5883L~\cite{magnetometro}}
	    \label{img:magnetometro}
	    \includegraphics[width=0.8\textwidth]{figuras/6}
	  \end{figure}
	  
\pagebreak

  \item Sensor de umidade

		\begin{figure}[H]
			\centering
			\caption{Sensor DHT11}
			\label{img:DHT11}
			\includegraphics[scale=0.6]{figuras/DHT11}
		\end{figure}

		O DHT11 (figura \ref{img:DHT11}) é um sensor que permite fazer a leitura da umidade entre 20 a 90\% e também de temperatura, e pode ser utilizado juntamente com arduinos. É alimentado com tensão de 3.5V e corrente de 200$\mu$A, possui tempo de resposta de 2 segundos, precisão de medição de umidade de mais ou menos 5.0\% UR e tem dimensões de 23 x 12 x 5 mm. A figura \ref{img:sensorumidade} descreve seu funcionamento.

	  \begin{figure}[H]
	    \centering
	    \caption[Diagrama funcional sensor de umidade]{Diagrama funcional sensor de umidade~\cite{sensorhumidade}}
	    \label{img:sensorumidade}
	    \includegraphics[width=0.8\textwidth]{figuras/5}
	  \end{figure}
\end{itemize}

Geralmente, os microcontroladores processam os dados obtidos por sensores e em suas saídas são encontrados valores analógicos, logo é necessário transformá-los em valores digitais. Para executar essa atividade, é preciso do conversor A/D, que inter-faceiam os dispositivos de medidas e o microcontrolador. Nesses conversores, quanto maior o número bits de saída, melhor ele será. Por exemplo, um conversor que tem uma saída de quatro bits possui dezesseis degraus de indicação, ou seja, pode definir uma escala de dezesseis valores diferentes. Se o circuito converte sinais na faixa de 0V a 1V, é preciso ter cuidado para que os sensores usados trabalhem nessa faixa. Um amplificador operacional pode ter um ganho programado para evitar esses problemas. Então, as saídas terão um número n de pinos nas quais as saídas nos níveis lógicos 0 ou 1 são obtidos conforme a tensão de entrada.

Como já mencionado anteriormente, a conversão de dados analógicos para digitais será realizada internamente nos sensores, esta característica pode ser visualizada em seus respectivos diagramas funcionais. Isto significa que os sensores fornecerão valores digitais em suas saídas, que estarão conectadas ao microcontrolador Arduino UNO.

\subsection{Sistemas de Câmeras}

Cada balão portará 1 câmera direcionada para a região em que se efetuará a monitoração. A câmera escolhida foi a Waveshare OV5647 Night Vision, com as especificações técnicas presentes nos itens abaixo e nas imagens \ref{img:Waveshare} e \ref{img:painel}.

\begin{itemize}
	\item 5MP.
	\item Vídeo: 1080 p.
	\item Abertura (F): 2.9
	\item Distância Focal: 3.29 mm.
	\item Diagonal: 72.4 mm.
	\item Dimensões: 25 mm x 24 mm x 6 mm.
	\item Suporta até 2 LEDs infra-vermelhos.
	\item Massa: $1.7 \cdot 10^{-2}$ kg.
	\item Preço: U\$30.99.
\end{itemize}

\begin{figure}[H]
  \centering
  \caption[Waveshare OV5647 Night Vision em destaque]{Waveshare OV5647 Night Vision em destaque~\cite{amazon1}}
  \label{img:Waveshare}
  \includegraphics[width=0.8\textwidth]{figuras/RSP}
\end{figure}

\begin{figure}[H]
  \centering
  \caption[Painel infra-vermelho]{Painel infra-vermelho~\cite{amazon2}}
  \label{img:painel}
  \includegraphics[scale=0.7]{figuras/painel}
\end{figure}

A câmera foi escolhida dada a sua alta resolução, fácil interface com o Raspberry PI, o sensor ser adequado para ser utilizado com o infravermelho para filmagens noturnas. Além disso, possui dimensões pequenas. O fabricante não informa o alcance do infravermelho para filmagens noturnas, dessa forma faz-se necessária a utilização conjunta com  câmera de um painél infravermelho externo. O painel escolhido é denominado: Aweek 850 nm, 60 LEDs IR com especificações:

\begin{itemize}
	\item Comprimento de onda: 850 nm.
	\item Consumo: 12 W.
	\item Tensão de operação: 12 VDC.
	\item Alcance: 60 m.
	\item Massa: 0.5 kg.
	\item Preço: U\$27.88.
\end{itemize}

As câmeras serão fixadas ao balão, e por estarem acondicionadas em seus respectivos invólucros (caixas de proteção) deverão continuar operando perfeitamente sob temperatura ambiente
entre 0 e 40$^{\circ}$C e umidade relativa do ar de até 90\%.

\subsection{Estabilização da Carga útil}

Embora que a princípio o balão trabalhará com altitude fixa, este tem o grau de liberdade para mudar de orientação em torno dos eixos ZB, YB e XB (considera-se o sistema de referência Body Axes), figura \ref{img:eixosreferencia}. O sistema de referência nos eixos do corpo tem origem geralmente no centro de massa, e utilizada para referenciar aeronaves, neste caso será aplicado à payload do balão. Estas mudanças de orientação ocasionarão a rotações involuntárias de câmeras embarcadas no balão, dessa forma faz-se necessária a estabilização do movimento.

\begin{figure}[H]
  \centering
  \caption{Eixos de referência em destaque.}
  \label{img:eixosreferencia}
  \includegraphics[width=0.8\textwidth]{figuras/estrutura}
\end{figure}

Tal movimento de rotação pode ser induzido pelas forças aerodinâmicas que agem no balão quando o fluxo de ar faz-se presente. O sistema de controle que seria capaz de estabilizar o sistema frente a uma perturbação seria classificado como de malha fechada, isso significa que um conjunto de sensores inerciais (acelerômetro, giroscópio) deve ser empregado para além de detectar a perturbação, verificar se o sistema de controle está sendo efetivo. Dessa forma o sistema de controle de malha fechada verifica se a saída condiz com as especificações de estabilidade do sistema, para se ter certeza de que a estabilização está sendo feita. O sistema de controle atuaria de forma intermitente enquanto a estabilização não fosse bem sucedida. Para fins de viabilidade, o sistema de controle empregado deve ser capaz de estabilizar a payload (setor de equipamentos embarcados) rapidamente, para se ter qualidade nas imagens geradas pela câmera.

Um provável atuador para o eixo ZB, ou seja, mecanismo capaz de efetuar a estabilização seria um Reaction Wheel. Um Reaction Wheel é um dispositivo frequentemente utilizado para o controle de atitude de satélites, consiste de um disco massivo acoplado a um eixo giratório. O princípio que o dispositivo usa para efetuar a estabilização é o momento de inércia do disco, dependendo da interpretação do algoritmo de controle das leituras dos sensores, sua rotação é ativada com velocidade e sentido determinados, executando-se a estabilização (anula a rotação da payload do balão no eixo). Tal atuador se encontrará no interior da payload. Como o Reaction Wheel, geralmente, é projetado para atuar no ambiente espacial, seu custo costuma ser bem elevado, na ordem de milhares de dólares. Como a carga útil não enfrentará condições similares ao espaço, não há a necessidade da compra de um modelo qualificado para o espaço, dessa forma é possível se optar pelo desenvolvimento do atuador. Os componentes para o desenvolvimento do Reaction Wheel são os seguintes:

	\begin{itemize}
		\item Tarugo de Alumínio (cilindro de alumínio usinável) (figura \ref{img:tarugo}).

			\begin{figure}[H]
			  \centering
			  \caption{Tarugo de Alumínio.}
			  \label{img:tarugo}
			  \includegraphics[width=0.8\textwidth]{figuras/tarugo}
			\end{figure}

			Seria utilizado para a fabricação do disco maciço. A massa e o diâmetro corretos desse disco dependeriam de uma pesquisa cientifica efetuada com a payload.

			Preço: para a dimensão de 8cm (diâmetro) e 4 cm de altura o preço varia de R\$150,00 à R\$350,00.

		\item Impressão 3D.

			Seria utilizada para a fabricação do envoltório do dispositivo.

			Preço: estima-se entre R\$100 e R\$250

		\item Arduíno Nano (figura \ref{img:arduinoNano})

			\begin{figure}[H]
				\centering
				\caption{Arduíno Nano V3.0}
				\label{img:arduinoNano}
				\includegraphics[width=0.8\textwidth]{figuras/arduinoNano}
			\end{figure}

			Seria utilizado para efetuar o controle da velocidade de rotação,por meio do driver para o motor.

			Preço: Flipeflop R\$59.90, frete não incluso ; Multilógica R\$269.00, frete não incluso; Hobbyking R\$25,77; frete não incluso.

		\item Fios.

			Preço: estima-se em um total de R\$25,00.

		\item Conectores.

			Preço: estima-se em um total de R\$50,00.

		\item Motor (figura \ref{img:motorAK})

			\begin{figure}[H]
				\centering
				\caption{Motor de Passo NEMA 23, modelo AK23/21F8FN1}
				\label{img:motorAK}
				\includegraphics[width=0.4\textwidth]{figuras/motorAK}
			\end{figure}

			Seu eixo estaria acoplado ao disco maciço de alumínio.

			O motor especificado, modelo AK23/21F8FN1, consome 2.8A, necessita de uma tensão de 3.36Vdc e torque de 21 kgf.cm.

			Preço: R\$309,00.

		\item Driver para motor de passo.

			O  driver escolhido e que se adequa ao motor utilizado é o SparkFun AutoDriver - Stepper Motor Driver. O mesmo modelo também será utilizado para efetuar o controle do trilho do balão, portanto suas especificações serão listadas posteriormente.
	\end{itemize}

Para a estabilização do eixo YB pode ser utilizado um trilho para mover a posição da bexiga, alterando o ângulo de pitch, de forma a nivelar o plano seccional horizontal da payload com o solo. Tal trilho está indicado na estrutura conceitual da payload, figura \ref{img:trilhoestrutura}.

\begin{figure}[H]
  \centering
  \caption{Trilho em destaque na estrutura conceitual da payload.}
  \label{img:trilhoestrutura}
  \includegraphics[width=0.6\textwidth]{figuras/e2}
\end{figure}

No caso o eixo XB, a estabilização pode ser feita através da variação da altitude do balão em intervalos de distância pré-definidos. Essa variação da altitude pode ser feita através da retração e liberação do cabo na carretilha em solo. A instabilidade no eixo ZB não afetará significativamente a qualidade da imagem, desde que a estabilização nos outros dois eixos seja efetiva.

Uma provável automação efetuada pelo balão será a avaliação de sua própria segurança. Por meio de sensores de tensão no cabo (dinamômetro) preso na estrutura da figura \ref{img:caboancoragem}, se esta aumentar acima de um nível critico, este será automaticamente recolhido por meio do rotor motorizado em solo, e a estação de solo será informada. Assim que o sensor em solo sinalizar normalidade na velocidade do vento este será novamente elevado.

\begin{figure}[H]
  \centering
  \caption{Conexão com cabo de ancoragem em destaque.}
  \label{img:caboancoragem}
  \includegraphics[width=0.8\textwidth]{figuras/e3}
\end{figure}


Mais uma automação essencial será a sua elevação e retração automática para o período de monitoração determinada.

O funcionamento completo do sistema de estabilização se dará da seguinte forma: quando o sistema aéreo é ativado, ocorrerá a auto-orientação da carga útil (payload). Tal auto-orientação buscará direcionar a câmera para uma dada região pré determinada. Posteriormente, qualquer perturbação que gere alteração da atitude (orientação) da carga útil deverá ser corrigida pelos atuadores, isto é, Reaction Wheel e trilho.

O reaction wheel comercial que pode ser utilizado é o da companhia Clyde Space. A tabela \ref{tab:reactionWheel} mostra as especificações técnicas do reaction wheel.

\begin{table}[H]
	\centering
  \caption[Especificação do Reaction Wheel]{Especificação do Reaction Wheel~\cite{clyde}}
	\begin{tabular}{|l|c|c|l|}
	\hline
	\rowcolor[HTML]{C0C0C0}
	\textbf{Característica}          & \textbf{Valor} & \textbf{Unidades} & \textbf{Notas}        \\ \hline
	Máxima velocidade volante        & 66500          & rpm               & @28 V                 \\ \hline
	Torque máximo à 6500 rpm         & 26             & mNm               & @28 V                 \\ \hline
	Torque máximo até 2500 rpm       & 40             & mNm               & @28 V                 \\ \hline
	Faixa de temperatura em operação & -20 a 50       & ºC                & \multicolumn{1}{|c|}{-} \\ \hline
	Faixa de temperatura em repouso  & -30 a 60       & ºC                & \multicolumn{1}{|c|}{-} \\ \hline
	Consumo quando inativo           & 1.5            & W                 & @28 V                 \\ \hline
	Consumo sem carga                & \textless 12   & W                 & @28 V                 \\ \hline
	Consumo em torque máximo         & \textless 28   & W                 & @28 V                 \\ \hline
	Inércia do disco                 & 0.001766969    & kg*$m^{2}$        & \multicolumn{1}{|c|}{-} \\ \hline
	Massa total do dispositivo       & 1.5            & Kg                & \multicolumn{1}{|c|}{-} \\ \hline
	\end{tabular}
	\label{tab:reactionWheel}
\end{table}

Para fazer a bexiga se locomover no trilho, faz-se necessária a utilização de um motor de passo. O motor escolhido foi o de modelo AK23/R100F6FN1.8-G10-LINIX com caixa de redução, devido ao torque, dimensões e consumo. Exemplo disponível na imagem \ref{img:motorpasso}.

\begin{figure}[H]
  \centering
  \caption[Motor de Passo AK23/R100F6FN1.8-G10-LINIX]{Motor de Passo AK23/R100F6FN1.8-G10-LINIX~\cite{robocore}}
  \label{img:motorpasso}
  \includegraphics[width=0.8\textwidth]{figuras/M}
\end{figure}

Dados técnicos do motor:

\begin{itemize}
	\item Tensão: 2.4 VDC.
	\item Corrente: 3 A.
	\item NEMA: 23.
	\item Folga: a folga do redutor é de 30 arcminutos (0.5 graus).
	\item Marca: LINIX.
	\item Preço: R\$359.00.
\end{itemize}

Para ser controlado, esse motor requer um driver, o modelo escolhido, por se adequar as especificações técnicas do motor foi o SparkFun AutoDriver - Stepper Motor Driver com as seguintes especificações~\cite{pololu}:

\begin{itemize}
	\item Detecção de superaquecimento.
	\item Deterção de excesso de corrente.
	\item Controlado por SPI.
	\item ADC de 5 bits.
	\item Faixa de tensão: 8 - 45 V.
	\item Corrente suportada: 3 A.
	\item Preço: U\$34.95.
\end{itemize}

\subsection{Interação Hardware - Software do Payload}
\label{sec:interacao_hard}

	Serão abordados os procedimentos de programação dos controladores envolvidos no funcionamento da carga útil (payload), e também a lógica envolvida nos algoritmos a serem implementados.

	\subsubsection{Arduino UNO}

		O Arduino será programado através de sua IDE, que permite que sketches sejam criados e mandados para a placa. A linguagem de programação é modelada segundo a linguagem Wiring, baseada em C e C++. O código é traduzido para a linguagem C e é transmitido para o compilador, que realiza a interpretação dos comandos para uma linguagem que o microncontrolador possa entender. O esquemático dos pinos do microcontrolador e do Arduino podem ser observados na imagem \ref{img:atmega}.

		Ao realizar o upload do código para a placa, o Arduino não precisa mais estar conectado ao computador. Ele executará o sketch criado, contanto que esteja conectado a uma fonte de alimentação.~\cite{embarcados1}

		\begin{figure}[H]
		  \centering
		  \caption[Diagrama dos pinos do microcontrolador do Arduino]{Diagrama dos pinos do microcontrolador do Arduino~\cite{embarcados2}}
		  \label{img:atmega}
		  \includegraphics[width=1\textwidth]{figuras/ATMEGA}
		\end{figure}

		Os pinos, de entrada e saída de informações, do Arduino UNO possuirão as seguintes conexões:
		\begin{description}
			\item[A0 e A1:] Entrada de dados do magnetômetro HMC5883L;
			\item[A2 e A3:] Entrada de dados do giroscópio L3G4200D;
			\item[A4 e A5:] Entrada de dados do acelerômetro ADXL345;
			\item[D2 e D3:] Entrada de dados do barômetro BMP085.
		\end{description}

		De acordo com a análise desses três sensores inerciais, o microcontrolador poderá comunicar-se com o Raspberry para que este informe ao Galileo e, por fim, seja ativado o sistema de estabilização do balão.

		D4 e D5: Entrada de dados do sensor 1 e do sensor 2 de temperatura LM35. O sensor 1 será utilizado exclusivamente para o monitoramento do aquecimento do sistema embarcado, verificando se houve sobrecarga dos componentes. O sensor 2 captará a temperatura do gás do balão.

		A comunicação entre esses sensores e o microcontrolador se dará da seguinte forma: de acordo com o dado obtido do ambiente, o sensor informará um determinado valor de tensão em seu terminal que estará conectado aos pinos de entrada de informação. O microcontrolador, de acordo com o que foi programado, relacionará este valor de tensão com o seu significado real medido pelo sensor.

		O Arduino processará os dados obtidos pelos sensores e verificará se há algo anormal com sistema. Caso haja algum problema, o Arduino transmitirá essas informações para que o Raspberry PI 2 possa enviar um alerta à base de solo, tendo em vista que esses estarão conectado por uma comunicação serial. Caso o sistema não apresente comportamento inadequado, o Arduino transmitirá apenas os dados de interesse do cliente.

	\subsubsection{Intel Galileo Gen 2}

		Sua programação será realizada no mesmo ambiente de desenvolvimento (IDE) do Arduino. Essa placa roda o sistema operacional Linux, que vem previamente instalado, configurado e com as bibliotecas de software do Arduino, o que torna mais viável sua programação na mesma IDE.~\cite{embarcados3}

		\begin{figure}[H]
			\centering
			\caption[Esquemático das entradas e saídas do Intel Galileo]{Esquemático das entradas e saídas do Intel Galileo~\cite{embarcados3}}
			\label{img:galileo2}
			\includegraphics[width=1\textwidth]{figuras/galileo2}
		\end{figure}

		O Galileo será usado exclusivamente para a estabilização do balão, estará conectado aso seguintes componentes:  ao atuador Reaction Wheel; ao motor de passo e seu driver responsável pela estabilização através do trilho; ao Raspberry PI 2. A comunicação com o Raspberry será necessária, visto que este estará conectado ao Arduino que, por sua vez, notificará qualquer mudança no sistema captada pelos sensores.

		Os pinos de entrada e saída de informações do Intel Galileo Gen 2 terão as seguintes conexões:
		\begin{description}
			\item[IO0, IO1, IO2 e IO3:] Estarão conectados ao controlador do atuador Reaction Wheel. A mudança de estado lógico, nestas entradas, indicará que os sensores captaram mudanças climáticas que provocaram a desestabilização do balão e que o atuador será ativado.
			\item[IO4, IO5, IO6 e IO7:] Estarão conectados ao motor de passo que realizará a estabilização através do trilho da estrutura. A mudança de estado lógico dessas entradas segue o mesmo princípio explicado no item anterior, porém estes pinos realizarão a estabilização via controle do trilho.
		\end{description}

	\subsubsection{Raspberry Pi 2}

		O Raspberry PI 2 será programado em Python pois, além de sua praticidade, é o recomendado pela Fundação Raspberry Pi mas qualquer linguagem que compilar para ARMv6 (Pi 1) ou ARMv7 (Pi 2) pode ser usada. Sendo que C, C++, Java, Scratch, e Ruby já vêm previamente instalados.~\cite{raspberrypi}

		O Raspberry estará conectado à câmera do balão através de seu conector específico para câmera, para a realização do processamento de imagens. Também estará conectado ao Arduino e ao Galileo, através de uma comunicação serial, a fim de transmitir os dados obtidos pelos sensores.

		\begin{figure}[H]
			\centering
			\caption[Entradas e saídas do Raspberry Pi 2]{Entradas e saídas do Raspberry PI 2~\cite{filipeflop}}
			\label{img:raspberryPi2}
			\includegraphics[width=1\textwidth]{figuras/raspberryPi2}
		\end{figure}

		As conexões, do Raspberry com outros componentes, estão descritas a seguir:
		\begin{description}
			\item[USB1:] Comunicação serial com o Arduino. Essa conexão será dada exclusivamente para o recebimento dos dados dos sensores, assim como um alerta de anomalia no sistema.
			\item[USB2:] Comunicação serial com o Galileo. Essa conexão será dada para o envio dos dados dos sensores que foram recebidos a fim de avisar ao sistema de estabilização que sua ativação será necessária.
			\item[Conector Ethernet:] Realizará a comunicação com a estação do solo. Esta conexão servirá para que tanto os dados recebidos dos sensores quanto as imagens captadas pelas câmeras sejam transmitidos até a estação de solo
			\item[Conector Câmera:] Entrada da câmera Waveshare OV5647 Night Vision. Essa conexão servirá para a comunicação entre o Raspberry e a câmera, a fim de captar os dados transmitidos e processá-los.
		\end{description}

		Os algoritmos que serão implementados terão a função de captar os valores fornecidos em determinados pinos dos microcontroladores, seja vindo diretamente do sensor ou da comunicação serial entre os microcontroladores. Em seguida, essa informação será interpretada e comparada com os valores pré-determinados. Caso os dados obtidos desrespeitem as condições impostas, o microcontrolador realizará um processo, sendo esse processo algo relacionado a autonomia do balão ou à sua comunicação com outras partes do sistema, assim como o monitoramento do seu funcionamento. Por exemplo, esse processo pode ser ativar em nível lógico alto um de seus pinos de saídas, que estarão ligados ao atuador responsável pela estabilização do balão, a fim de que este seja ligado. Esse processo também poderia ser mandar um sinal para o Raspberry para que esse possa se comunicar com a estação de solo, avisando qualquer mudança suspeita no sistema.

		O Raspberry PI conterá um algoritmo que realizará a compressão dos vídeos enviados pela câmera no padrão H264.		Esse é um padrão de codificação de vídeo de última geração presente em videoconferências, bancos de dados, streaming de vídeo e TV digital, entre outros que nos permite o uso de técnicas avançadas como: esquema de previsão no qual utiliza um quadro de referência para comparação e codifica apenas os pixels que foram modificados, maior compressão de movimento, que leva em consideração que grande parte do próximo quadro pode ser encontrada no quadro anterior, porém em um lugar diferente~\cite{axis1}. A compressão de vídeo é essencial porque um vídeo em seu formato original requer muito mais espaço de armazenamento e capacidade de transmissão do que o mesmo vídeo em sua forma comprimida~\cite{ostermann}.

		No campo da codificação (compressso/descompressão), o padrão H.264/AVC é um dos mais modernos padrões de codificação de vídeo~\cite{sullivan}, com uma alta taxa de compressão e diversas configurações que se ajustam a diferentes necessidades, desde o armazenamento em um disco até o streaming de vídeo~\cite{wiegand}.

		À medida que a compressão se torna mais eficiente, os cálculos vão se tornando mais complexos, exigindo maior poder de processamento. Isto gera uma grande carga computacional, tornando necessário o uso de um hardware de alta capacidade para seu tratamento, o que justifica a escolha do Raspberry para o setor de processamento de imagens. Como uma forma de conseguir poder de processamento, são usadas técnicas de processamento paralelo~\cite{mattson}, na qual várias unidades de processamento trabalham em conjunto para resolver um problema. A computação paralela será utilizada na compressão do projeto, visando resolver possíveis problemas no processamento.

		Um vídeo, segundo o Padrão H.264/AVC, é definido como uma sequência do mesmo em um formato particular: a sintaxe H.264/AVC. Essa sintaxe determina a estrutura precisa de uma sequência de vídeo que esteja segundo um padrão como, por exemplo, o modo de representação binária de cada elemento. A sintaxe H.264/AVC é hierárquica, sendo que o início é o seu nível mais alto, que é o nível da sequência de vídeo até o nível dos macro-blocos. Ao realizar a compressão sobre um vídeo, este se torna constituído por um fluxo de unidades NAL(Network Abstraction Layer). O SPS (Sequence Parameter Set) e o PPS (Picture Parameter Set) fazem parte da NAL, que também contém dados importantes para o decodificador. Todos os dados do vídeo serão armazenados nas unidades de tipo NAL. A sequência de vídeo é, simplesmente, um bloco NAL do tipo fatia IDR (Instantaneous Decoder Refresh) seguida por zeros ou mais NALs desse tipo~\cite{morais}. A estrutura hierárquica da sintaxe do H.264 está ilustrada na figura \ref{img:estruturah264}.

		\begin{figure}[H]
			\centering
			\caption[Estrutura hierárquica da sintaxe do H.264]{Estrutura hierárquica da sintaxe do H.264. Adaptado de \citeauthoronline{richardson} (\citeyear{richardson})}
			\label{img:estruturah264}
			\includegraphics[width=1\textwidth]{figuras/h264}
		\end{figure}

	\subsection{Sistema de Telecomunicações}

	O sistema de telecomunicação será usado para transmissão do vídeo e das fotos que a câmera captar. O sistema começa com a fonte de informação (câmera), dela partirá todos os dados que buscamos transmitir. Os formatos dos dados são digitais transmitidos em bits a uma velocidade dependente da câmera e do cabo que serão utilizados. Será utilizado um cabo flat para a ligação da câmera ao microcontrolador.

	O cabo sairá da câmera e será ligado ao Raspberry PI 2 (mini computador), que estará sendo usado para controlar os setores de armazenamento de imagens das câmeras. O minicomputador terá instalado o GStreamer ,que é uma framework que controlará o fluxo de dados, o ideal desse sistema é que ele lida especificamente com stream multimídia, como áudio e vídeo em uma latência muito baixa.

	Como pode se notar nas especificações do Raspberry PI 2, este apenas tem interface Ethernet, para termos rede necessitamos de adquirir um cabo de rede normal, mas para o caso deste projeto, não seria viável atravessar vários cabos com média de 100 metros pela universidade, então nesse caso pretendemos ligar o Raspberry PI a uma rede sem fio e por isso é necessário adquirir uma Antena Wireless Omini 8dbi Tp-link 2408cl Tl-ant2408cl 27cm para cada balão, cada uma custa R\$ 27,00~\cite{mercadolivre1} e se encontra no padrão de transmissão que oferece um alcance que precisamos e ela pode ser conectado com adaptador RP\_SMA para flat ao microcontrolador, então este componente se encontra dentro das nossas necessidades.

	\begin{figure}[H]
		\centering
		\caption[Antena Omini]{Antena Omini~\cite{mercadolivre1}}
		\label{img:antenaOmini}
		\includegraphics[width=0.4\textwidth]{figuras/antena}
	\end{figure}

  Sendo postos 5 balões ao redor do campus, ligaremos via intranet, rede interna, fechada e exclusiva sem a necessita de internet, os balões 1-3 com o 4 que se encontrará mais perto da central, a conexão sem fio entre os balões é mais viável por não haver obstáculos entre eles, os dados serão passados do balão 4 via cabo ethernet ate a central, por volta de 60 metros, pois esse estágio da comunicação pode ser encontrado obstáculos físicos para a transmissão, como exemplo carros, arvores, paredes, então a utilização do cabo pode diminuir a perda ou as interferências nos sinais. O mapa do campus com as posições dos balões pode ser observado na imagem \ref{img:campusPosicaoCam}.

  \begin{figure}[H]
    \centering
    \caption[Mapa do Campus com as posições dos balões]{Mapa do Campus com as posições dos balões~\cite{mapa1}}
    \label{img:campusPosicaoCam}
    \includegraphics[width=0.6\textwidth]{figuras/localizacao-baloes}
  \end{figure}

	Caso se queira realizar a transmissão para mais de um computador, será feito um switch, que possui um exemplo na imagem \ref{img:antenaOmini}. A função do switch é encaminhar dados de um dispositivo para outro dentro uma mesma rede. “Ele registra os endereços MAC (Controle de Acesso à Mídia) de todos os dispositivos conectados a ele. (Cada dispositivo de rede tem um endereço MAC exclusivo, que consiste em uma série de números e letras definidos pelo fabricante, e o endereço pode ser muitas vezes encontrado na etiqueta do produto). Quando um switch recebe dado, ele os encaminha apenas à porta que estiver conectada a um dispositivo com o endereço MAC correto do destino.”~\cite{axis2}.

	\begin{figure}[H]
		\centering
		\caption[Exemplo de Switch]{Exemplo de Switch~\cite{axis2}}
		\label{img:antenaOmini}
		\includegraphics[width=0.6\textwidth]{figuras/switch}
	\end{figure}

	Com um switch de rede, a transferência de dados é gerenciada de maneira muito eficiente, pois o tráfego de dados pode ser direcionado de um dispositivo para outro sem afetar nenhuma outra porta do switch.~\cite{axis2}

	A central mostrará o que se passa na câmera em tempo real, mas também será possível guardar momentos programados para um analise futura. As configurações, alterações e analises serão todas realizadas por softwares específicos.

	\subsection{Manutenção}

		O Balão Cativo funcionará vinte e quatro horas por dia durante os sete dias das semanas. Dito isso, a manutenção do sistema deve ser realizadas frequentemente a fim de corrigir ou evitar paradas duradouras de funcionamento . Alguns passos devem ser tomados quando houver esse tipo de procedimento. O sistema terá dois tipos de manutenção: a corretiva e a preventiva.

		Manutenção Corretiva: Quando o equipamento quebrar ou deixar de funcionar, o profissional responsável será chamado para resolver a questão. Esses casos acontecerão raramente, mas é preciso ter bastante cuidado nessas ocasiões, pois a manutenção durará vinte e quatro horas, caso o problema seja de fácil resolução e setenta e duas horas, caso seja um problema de difícil resolução ou a peça defeituosa seja de difícil aquisição.

		Manutenção Preventiva: A cada semana, um balão receberá vistoria dos técnicos. Os componentes e conexões serão reavaliados para analisar o desempenho e observar se está dentro do esperado. Esse tipo de manutenção evitará que todos sejam pegos de surpresa, além de evitar maiores defeitos dos equipamentos. Isso culminará num menor número de manutenção corretiva e, consequentemente, menos dinheiro desembolsado.
		Esse tipo de manutenção irá durar cerca de trinta minutos para cada balão.

		Além disso, um caminho mais seguro pode ser a melhor solução. Quando um dos balões estiverem em manutenção, a equipe deve ter um balão reserva, tendo a mesma função do step nos carros. Dessa maneira, quando o balão for retirado para a manutenção, outro será colocado no local para manter a segurança do estacionamento. Caso o balão reserva esteja com defeito, as seguintes providencias deverão ser tomadas:

		\begin{enumerate}
				\item Quando o balão estiver com problemas, os seguranças serão acionados e deverão cuidar do local enquanto o balão estiver inativo;
				\item As autoridades responsáveis pelo campus receberão um aviso de que deverão reforçar a segurança do Campus nesse período;
				\item Os alunos, professores e funcionários serão alertados, na entrada do estacionamento, para tomarem maior cuidado.~\cite{resolvemicro}
		\end{enumerate}


		\subsection{Peso, Consumo e Custo dos Componentes do \textit{PayLoad}} % (fold)
		\label{sub:peso_e_custo}

		  Inicialmente, foi realizado o compativo entre orçamentos com possíveis fornecedores, visando obter um projeto mais viável.
		  O levantamento destes custo é apresentado na seguinte Tabela Comparativa \ref{tab:CompCusto}.

\begin{table}[H]
  \scalefont{1}
    \centering
    \caption{Tabela comparativa de custos dos componentes eletrônicos.}
    \begin{tabular}{|c|c|c|c|}
      \hline
      \cellcolor[HTML]{FFFFFF}{\color[HTML]{000000} \textbf{Componente}} & \textbf{Orçamento 1} & \textbf{Orçamento 2} & \cellcolor[HTML]{FFFFFF}{\color[HTML]{000000} \textbf{Orçamento 3}} \\ \hline
      Sensor LM35                                                        & R\$ 14,89        & R\$ 7,50			      & R\$ 13,45                                                          \\ \hline
      Giroscópio L3G4200D                                                & R\$ 23,32        & R\$ 28,23		              & R\$ 43,26                                                         \\ \hline
      Sensor DHT11                                                       & R\$ 10,28        & R\$ 12,61			      & R\$ 13,78                                                         \\ \hline
      Raspbarry PI                                                       & R\$ 129,44       & R\$ 143,20		      & \$ 32,64 = R\$ 127,28                                               \\ \hline
      Arduíno Uno                                                        & R\$ 73,78        & R\$ 92,27			      & R\$ 42,99                                                          \\ \hline
      Placa de Fenolite                                                  & R\$ 7,72         & R\$ 7,90                        & R\$ 2,82                                                           \\ \hline
      Acelerômetro ADXL345                                               & R\$ 17,00        & R\$ 15,00			      & R\$ 22,00                                                          \\ \hline
      BMP180                                                             & R\$ 15,53        & R\$ 11,21		              & R\$ 9,11                                                           \\ \hline
      Waveshare OV5647 Night Vision                                      & R\$ 114,56       & R\$ 107,19	              & R\$ 102,62                                               \\ \hline
      Aweek 850nm                                                        & R\$ 96,11        & R\$ 133,49		      & R\$ 129,31                                               \\ \hline
      Antena Wiriless Omini                                              & R\$ 26,86        & R\$ 27,42		              & R\$ 27,00                                                          \\ \hline
      HMC5883L Magnetômetro                                              & R\$ 9,15         & R\$ 9,62 			      & R\$ 11,64                                                  \\ \hline
      Motor de Passo                                                     & R\$ 359,00       & R\$ 449,99		      & R\$ 427,99                                                         \\ \hline
      Inter Galileo Gen 2                                                & R\$ 184,92       & R\$ 221,90		      & R\$ 242,91                                                 \\ \hline
    \end{tabular}
    \label{tab:CompCusto}
  \end{table}



		  A escolha dos fornecedores foi realizada visando maior viabilidade ao projeto. Na Tabela \ref{tab:pesoCustoComp} abaixo, está o levantamento do peso, do consumo energético e dos preços
		  dos componentes da Eletrônica Embarcada escolhidos após a análise dos fornecedores.

  \begin{table}[H]
    \scalefont{0.9}
    \centering
    \caption{Peso, consumo e custo dos componentes da \textit{payload}.}
    \begin{tabular}{|c|c|c|c|}
      \hline
      \cellcolor[HTML]{FFFFFF}{\color[HTML]{000000} \textbf{Componente}} & \textbf{Peso(g)} & \textbf{Consumo Energético (W)} & \cellcolor[HTML]{FFFFFF}{\color[HTML]{000000} \textbf{Preço(R\$)}} \\ \hline
      Sensor LM35                                                        & 5                & 280$\mu$			      & R\$ 7,50                                                           \\ \hline
      Giroscópio L3G4200D                                                & 5                & 15$\mu$		              & R\$ 99,90                                                          \\ \hline
      Sensor DHT11                                                       & 15               & 7.5m			      & R\$ 12,90                                                          \\ \hline
      Raspbarry PI                                                       & 45               & 10			      & \$ 32,64= R\$ 127,28                                               \\ \hline
      Arduíno Uno                                                        & 28               & 6				      & R\$ 42,99                                                          \\ \hline
      Cabos e Fios                                                       & 4000             & Não se aplica 		      & R\$ 40,00                                                          \\ \hline
      Placa de Fenolite                                                  & 31,3             & Não se aplica                   & R\$ 2,82                                                           \\ \hline
      Acelerômetro ADXL345                                               & 16               & 350$\mu$			      & R\$ 75,00                                                          \\ \hline
      BMP180                                                             & 1,2              & 330$\mu$			      & R\$ 9,11                                                           \\ \hline
      Waveshare OV5647 Night Vision                                      & 170              & 3		                      & \$ 30,99= R\$ 124,75                                                 \\ \hline
      Aweek 850nm                                                        & 500g             & 15			      & \$ 27,88 = R\$ 108,72                                                \\ \hline
      Reaction Wheels                                                    & -----            & 28  			      & -----                                                              \\ \hline
      Antena Wiriless Omini                                              & 70               & Não se aplica		      & R\$ 27,00                                                          \\ \hline
      HMC5883L Magnetômetro                                              & 1,2              & 250$\mu$				      & R\$ 42,32                                                  \\ \hline
      Motor de Passo                                                     & 1200             & 9			   	      & R\$ 359,00                                                         \\ \hline
      Inter Galileo Gen 2                                                & 340              & 15			      & \$104,49= R\$ 407,48                                                 \\ \hline
      \rowcolor[HTML]{9B9B9B}
      \textbf{Total}                                                     & \textbf{6426,5}  & 86.33			      & \textbf{R\$ 1477,66}                                               \\ \hline
    \end{tabular}
    \label{tab:pesoCustoComp}
  \end{table}


  \textbf{Observação}: Esse levantamento é por balão, sendo que este conterá apenas uma câmera. Para os balões que conterão duas câmeras, será necessário a inclusão de um Raspberry PI
  e um painel de LEDs IR, o Aweek. Logo, o valor estipulado para os balões que contém duas câmeras é de R\$ 1461,10. Além disso, o Reaction Wheel não tem nenhuma característica divulgada,
  pois os valores só são informados depois que é feito um orçamento. Optando-se pela montagem do Reaction Wheel, é perceptível que não foi possível se encontrar um orçamento preciso ou mais
  de um orçamento para alguns dos componentes. Isso ocorreu devido a fatores relacionados a parâmetros  do equipamento que demandam pesquisa científica tais como momento de inércia, e também
  escassez de fornecedores. Os valores obtidos em dólares foram convertidos no dia 29/10/2015, quando \$ 1 = R\$ 3.8997.

% subsection peso_e_cuso (end)

% section subprojeto_da_eletr_nica_embarcada (end)

\section{Subprojeto da Estação de Solo} % (fold)
\label{sec:subprojeto_da_esta_o_de_solo}
Esta seção engloba todas as características dos componentes e funcionamento dos mesmos dentro da Estação Solo. Especificando todo o funcionamento do projeto, desde obtenção, processamento, identificação de risco, armazenamento e o processo de interação com a segurança do Campus.

\subsection{Armazenamento dos Dados}

	Neste projeto será utilizada a câmera Waveshare OV5647 Night Vision. É uma câmera já voltada para sistemas de vigilância, muito utilizada em escritórios e shoppings.

	Serão utilizadas, no total, 15 câmeras do modelo  Waveshare OV5647, e para conseguir armazenar os videos gravados utilizaremos o HD Seagate Archive 8TB. Como o sistema funcionará  24x7, ou seja, 24 horas por 7 dias da semana, o servidor irá passar as imagens caso a situação seja considerada de risco, as imagens irão para o HD a uma taxa de 1024 Kbps, em um mês (considerando um mês como 30 dias), será gasto um total de 4.63TB.

\begin{table}[H]
	\centering
	\caption[Especificações da Seagate Archive 8TB]{Especificações da Seagate$^{\textregistered}$ Archive 8TB~\cite{seagate}}
	\begin{tabular}{|l|l|}
		\hline
		Capacidade                  & 8TB            \\ \hline
		Modelo                      & ST8000AS0002    \\ \hline
		Interface                   & SATA de 6 GB/s \\ \hline
		Velocidade da rotação       & 5900 RPM       \\ \hline
		Cache                       & 128 MB          \\ \hline
		Impacto máximo de operação  & 80 Gs          \\ \hline
		Tipo de armazenamento       & HDD            \\ \hline
		Comprimento                 & 147.00 mm      \\ \hline
		Largura                     & 101.85 mm      \\ \hline
		Altura                      & 26.1 mm        \\ \hline
		Potência média de operação  & 7.500 W        \\ \hline
		Taxa de transferência       & 600 MB/s       \\ \hline
	\end{tabular}
	\label{tab:my-label}
\end{table}

\begin{table}[htp]
\centering
\caption{Configuração de Hardware}
\begin{tabular}{|p{5cm}|p{10cm}|}
\hline
Processador              & Intel Core i7 - 4700K                                                   \\ \hline
Placa-mãe                & ASRock Z87Killer                                                        \\ \hline
Memoria                  & 16 GB G. Skill Spiner (DDR 3 - 1600/PC3 - 12800), configurada a 1600MHz \\ \hline
Placa de vídeo           & GeForce GT 630 1GB                                                      \\ \hline
Resolução de vídeo       & 1920x1080                                                               \\ \hline
Fonte de alimentação     & Corsair CX500M                                                          \\ \hline
Unidade de inicialização & Kingston HyperX 3k 480 GB                                               \\ \hline
\end{tabular}
\label{tab:configHardware}
\end{table}

As Tabelas \ref{tab:my-label} e \ref{tab:configHardware} são mostradas para apresentar as especificações do HD Seagate Archive 8TB e as configurações necessarias de hardware do computador para suportar esse poderoso HD.

Em tempos de muitas chuvas, ocorre uma grande variação de energia, devido às descargas
elétricas de raios. Para que não se tenha o problema de o sistema parar de funcionar por falta de
energia, e pela variação de energia, não chegar a queimar o sistema ou danificar o sistema,
será utilizado um equipamento que armazenar energia por algum tempo.

O equipamento utilizado para o sistema de energia nobreak, será o \textbf{Nobreak Organizador e
Fonte para 16 câmeras}, da tecnologia ONAT. Com este equipamento, o armazenamento de
dados terá em média 4 horas de autonomia, ou seja, caso por algum motivo a luz acabe o
sistema terá em média 4 horas funcionando perfeitamente \cite{nobreak}.

O Seagate para gravação e backup de imagens deverá ser alimentado pelo sistema de energia
(nobreak), de forma a possibilitar a operação em caso de falta de energia elétrica.

\subsection{Interação com a segurança do Campus}

O sistema de comunicação será dado de maneira manual, ou seja, terá uma pessoa na estação de solo que será responsável por analisar os monitores de vigilância que  informam as áreas de possíveis situações de risco mediante a pontuação preestabelecida no sistema. E caso seja necessário, o operador irá alertar um segurança para que ele possa averiguar tal situação. O sistema será uma ferramenta para o operador, auxiliando e facilitando o monitoramento do estacionamento.

Essa comunicação será dada via voz, utilizando um rádio comunicador, ou walk talk. Este meio de comunicação é bem utilizado em sistemas de vigilância de escritórios, shoppings, em construções civis ou em operações de policiais e bombeiros.

Os Walkie Talkies tem um alcance relativamente alto, exatamente o necessário para suprir a carência de sinal de celular presente na área da FGA, e este equipamento terá um alcance de  56km. Neste sistema de monitoramento será utilizado o rádio comunicador walk talk Cobra Cxr925 56km.

A tabela \ref{table:walk} serve para mostrar as especificações do meio de comunicação entre o operador da estação solo e os seguranças que ficarão responsaveis por andar pelo campus.
\begin{table}[H]
\centering
\caption[Especificações e preço do Walk Talk Cobra Cxr925 56km]{Especificações e preço do Walk Talk Cobra Cxr925 56km~\cite{walk}}
\begin{tabular}{|l|l|}
\hline
Peso        & 68g                          \\ \hline
Alcance     & 56 km                        \\ \hline
Dimensões   & 177.50mm x 49.00mm x 33.00mm \\ \hline
Frequência  & 22 canais                    \\ \hline
Alimentação & 110 V                        \\ \hline
\end{tabular}
\label{table:walk}
\end{table}

\subsection{Processamento dos dados}

Para o armazenamento se tornar eficiente e confiável, além de resiliente, será implementado o
padrão RAID(\textbf{Redundant Array of Independent Disks}) em seu nível um também conhecido por
Mirror.

Todas as informações vindas processadas para armazenamento serão copiadas simultaneamente em dois HDs, reduzindo a performance porém mantendo assim a segurança, pois caso haja algum problema técnico em um dos armazenamentos, não haverá nenhuma perda de informação. Neste projeto serão utilizados 4 HD Seagate Archive 8TB para o armazenamento das imagens do monitoramento, formando 2 pares de HDs, que serão redundantes \cite{raid}.

Cada par será capaz de armazenar informações por 30 dias, uma vez cheio as informações passarão a ser armazenadas no par ocioso, formando assim um total de 60 dias de armazenamento de informação, ao final deste período, informações armazenadas serão a ser eliminadas, sendo assim necessário realizar cópias para outros dispositivos caso seja necessário o uso em um período posterior.O tempo em média para recuperação das imagens será de uma a duas horas.

\subsubsection{Tolerância a Falhas}

Um software bem projetado corretamente desde a sua elaboração, não necessita de técnicas de tolerância para software, mesmo que ainda não seja possível garantir na pratica que todo programa estarão corretos \cite{webertolerancia}.

Nesse projeto para que se não tenha problemas em armazenar os dados que forem necessários ou com problemas de ataques externos, será utilizado um dos métodos de tolerância a falhas, e esse método é chamado de \textbf{Diversidade (ou programação de n-versões)}.

\begin{itemize}
	\item Diversidade (ou programação n-versões)
\end{itemize}
\par


Diversidade, também chamada programação diversitária, é uma técnica de redundância usada para obter tolerância a falhas em software. A partir de um problema a ser solucionado são implementadas diversas soluções alternativas, sendo a resposta do sistema determinada por votação.

Esse método consiste em projetar o software N vezes, sendo cada versão independente das demais, e caso o sistema perceba perceba algo de diferente no tempo de execução, será realizado uma votação para solução dos problemas. A votação é feita em tempo de execução e é analisadas todas as versões do software,sendo essas N versões feita por equipes de programadores diferentes, o que faz com que o erro seja manisfestado de maneira diferente em cada versão. A votação pegara cada erro e irá escolher o erro especificado que garanta mais integridade ao sistema e menos perca ou até perca total de dados do sistema.

\textbf{Vantagens}
\par
A vantagem é usar esse método, é de manter uma maior confiabilidade no sistema e na segurança de seus dados.

\textbf{Desvantagens}
\par
Desenvolvedores costumam ter as mesmas práticas de programação, não garantindo a independência das versões.Outra desvantagem é o custo, pois tem se o custo com as equipes de desenvolvimento do software e manutenção do mesmo, a complexidade de sincronização das versões.

\textbf{Situação Hipotética}
\par
Quatro equipes de desenvolvedores são escolhidos para projetar uma versão do software. Utilizando a situação ideal em que todas as equipes não tenham as mesmas práticas de desenvolvimento, são criadas as quatro versões são criadas. Em um momento o sistema detecta uma atividade diferente da que foi projetado e é iniciada a verificação e votação das quatro versões do sistema.A votação diagnosticou que o erro se manisfestou de maneira em que seria menos recomendado usar a versão quatro, mas na versão dois o erro é mais fácil de ser corrigido com menos ou talvez sem dano algum ao sistema. Enquanto o erro na versão quatro é corrigido para que seja mais efetivo caso ocorra outro problema, as demais versões estarão funcionando perfeitamente.
\\
\\

\subsection{Operacionabilidade do Sistema}

O diagrama de Arquitetura do sistema pode ser observado na imagem \ref{img:Operacionabilidade do Sistema}.

\begin{figure}[]
	\centering
	\caption{Diagrama de Arquitetura: Operacionalibilidade do Sistema}
	\includegraphics[width=1\textwidth]{figuras/OperacionabilidadedoSistema}
	\label{img:Operacionabilidade do Sistema}
\end{figure}

O sistema funcionará da seguinte maneira: no balão ocorrerá o processamento e tratamento das imagens recebidas (Processamento das câmeras). Após serem tratadas, estas imagens serão armazenadas em um HD da estação de solo(Gerenciamento de dados).

Em outro processo paralelo, estas imagens  irão fornecer os dados para que o sistema execute suas funcionalidades, como o gerenciamento das imagens, a identificação e pontuação das situações de risco (Processamento). E a saída do sistema será por meio de uma interface que interagirá com o operador da estação de solo.

\subsection{Processo de Monitoramento}
O processo de monitoramento do Sistema Unificado de Monitoramento (SUM), relativo ao modus operandi dos agentes do sistema, é ilustrado pelo diagrama \ref{img:Processo de Monitoramento}.

\begin{figure}[]
\centering
\caption{Processo de Monitoramento}
\includegraphics[width=1.0\textwidth]{figuras/Processodemonitorament}
\label{img:Processo de Monitoramento}
\end{figure}
As atividades contempladas neste processo estão descritas abaixo:
\\

\textbf{Identificar comportamento ou atividade suspeita} - O processo é iniciado a partir da identificação de comportamento ou atividade suspeita por parte do balão de monitoramento, que leva em consideração fatores de risco, tabela \ref{tab:criteriosRisco}, que determinam se uma atividade é normal, duvidosa ou suspeita.
\\

\textbf{Emitir alerta à central de Monitoramento} - Nesta atividade o sistema do balão, após ter identificado uma atividade suspeita nas dependências do estacionamento, emiti um sinal de alerta à Central de monitoramento contatando o operador do sistema.
\\

\textbf{Avaliar possível atividade criminosa} - Nesta atividade o operador do sistema, após ter sido notificado pelo balão sobre uma atividade suspeita, acompanhará através das câmeras de vídeo do balão, em tempo real, a ação do suspeito, avaliando se esta é uma atividade criminosa.
\\

\textbf{Desativar alerta do balão de monitoramento} - Nesta atividade o operador do sistema, após ter recebido e avaliado o alerta de atividade suspeita emitido pelo balão e concluído que este não retrata uma atividade criminosa, desativará o alerta do balão para aquela ação em específico, encerrando o processo.
\\

\textbf{Alertar a equipe de Segurança} - Nesta atividade o operador do sistema, após ter concluído que o alerta do balão se trata de fato de uma atividade criminosa, irá contactar a equipe de Segurança em campo emitindo um alerta via rádio.
\\

\textbf{Averiguar a situação do local} - Nesta atividade a equipe de Segurança, após receber um alerta do operador do sistema sobre uma atividade criminosa, irá averiguar a situação , no local informado, para avaliar a possibilidade de intervenção e/ou impedimento da ação criminosa.
\\

\textbf{Tomar medidas cabíveis} - Nesta atividade a equipe de Segurança, após concluir que há possibilidade de intervenção e/ou impedimento da ação criminosa, tomará as medidas cabíveis para que o infrator seja detido, encerrando o processo.
\\

\textbf{Contactar as autoridades locais} - Nesta atividade a equipe de Segurança, após concluir que não há possibilidade de intervenção e/ou impedimento da ação criminosa por quaisquer razões, irá contactar as Autoridades locais responsáveis para que estes tomem as medidas cabíveis à situação.
\\
\par
\textbf{Requisitos do Sistema de Monitoramento}

A rastreabilidade dos requisitos do sistema pode ser observada na tabela \ref{tab:Requisitos do Sistema de Monitoramento}.

\begin{table}[H]
\centering
\caption{Requisitos do Sistema de Monitoramento}
\begin{tabular}{|l|l|}
\hline
\textbf{Sigla}        & \textbf{Descricao} \\ \hline
NEC     & Necessidade                      \\ \hline
CAR   	& Característica				   \\ \hline
UC  	& Casos de uso                     \\ \hline
RNF 	& Requisito Não-Funcional          \\ \hline

\end{tabular}
\label{tab:Requisitos do Sistema de Monitoramento}
\end{table}

\textbf{Necessidades}
\par
A partir do processo de monitoramento descrito no tópico acima abstraiu-se as necessidades dos agentes do sistema.

As necessidades são:
\begin{itemize}
	\item NEC01 - O operador do sistema precisa receber um alerta do sistema de monitoramento quando houver uma atividade suspeita ocorrendo no estacionamento.
	\item NEC02 - O operador do sistema precisa visualizar em tempo real a ação do suspeito através das câmeras do balão de monitoramento.
	\item NEC03 - O operador do sistema precisa conseguir aproximar a imagem de forma a ser possível enxergar detalhes da ação do suspeito.
	\item NEC04 - O operador do sistema precisa visualizar áreas específicas do estacionamento para acompanhar o que ocorre em cada setor.
\end{itemize}

\textbf{Características}
\par
As características deviradas a partir das necessidades dos agentes do sistema são:
\begin{itemize}
	\item CAR01 - O sistema irá prover alertas de atividade suspeita na interface do usuário.
	\item CAR02 - O sistema deve permitir que um alerta de atividade suspeita seja desativado.
	\item CAR03 - O sistema deve permitir o acesso às câmeras de vídeo do balão de monitoramento.
\end{itemize}

\subsection{Diagrama de Caso de Uso}

O diagrama de casos de uso do sistema apresenta em modo gráfico as funcionalidades e os Atores do sistema, o mesmo pode ser observado na imagem \ref{img:Casos de Uso}.
\begin{figure}[H]
	\centering
	\caption{Diagrama de Caso de Uso}
	\includegraphics[width=1\textwidth]{figuras/casodeUso}
	\label{img:Casos de Uso}
\end{figure}

\begin{itemize}
	\item UC01 - Captar e processar imagens
		\begin{enumerate}
			\item Descrição

				Este caso de uso se refere ao balão capturar e enviar as imagens para a estação de solo em um formato processável.

			\item Pré-condições

				Não possui pré condições.

			\item Pós-condições
				\begin{itemize}
					\item A estação de solo recebe as imagens em formato armazenável.
				\end{itemize}

			\item Atores
				\begin{itemize}
					\item Balão.
				\end{itemize}


			\item Fluxo básico
				\begin{enumerate}
					\item O balão captura as imagens;
					\item O balão processa as imagens transformando em formato legível e armazenável;
					\item O balão envia as imagens à estação de solo.
					\label{item:balaoEnvia}
				\end{enumerate}

			\item Fluxo alternativo A
				\begin{enumerate}
					\item No passo \ref{item:balaoEnvia};
					\item O balão não consegue enviar as imagens;
					\item O fluxo volta para o passo \ref{item:balaoEnvia}.
				\end{enumerate}
		\end{enumerate}

		\item UC02 - Gerenciamento de imagens
			\begin{enumerate}
				\item Descrição

					Este caso de uso se refere ao sistema fazer processamentos para encontrar atividades de risco.

				\item Pré-condições
					\begin{itemize}
						\item O balão enviou as imagens.
					\end{itemize}

				\item Pós-condições
					\begin{itemize}
						\item Imagens processadas e encontradas possíveis atividades de risco.
					\end{itemize}

				\item Atores
					\begin{itemize}
						\item estação de solo.
					\end{itemize}


				\item Fluxo básico
					\begin{enumerate}
						\item A estação de solo recebe as imagens do balão;
						\item O sistema encontra e conta todas as atividades de risco na imagem;
						\item O sistema marca todas as atividades;
						\item Fim do fluxo.

					\end{enumerate}
			\end{enumerate}

			\item UC03 - Informar grau de risco da atividade.
				\begin{enumerate}
					\item Descrição

						Este caso de uso se refere a pontuar cada atividade de risco encontrada pelo caso de uso UC02.

					\item Pré-condições
						\begin{itemize}
							\item As atividades suspeitas já foram encontradas.
						\end{itemize}

					\item Pós-condições
						\begin{itemize}
							\item As atividades suspeitas estão pontuadas.
						\end{itemize}

					\item Atores
						\begin{itemize}
							\item estação de solo.
						\end{itemize}


					\item Fluxo básico
						\begin{enumerate}
							\item A estação de solo recebe as atividades suspeitas;
							\item O sistema identifica em qual das condutas de risco a atividade se encontra;
							\item O sistema cria uma nota para a atividade;
							\item Fim do fluxo.

						\end{enumerate}
				\end{enumerate}

		\item UC04 - Identificar situação suspeita.
			\begin{enumerate}
				\item Descrição

					Este caso de uso especifica a ação do sistema em disparar alertas de possíveis roubos para o operador do sistema e selecionar as cores para emoldurar as imagens de acordo com seu ranking.

				\item Pré-condições
					\begin{itemize}
						\item O operador do sistema tem que estar logado no sistema.
					\end{itemize}

				\item Pós-condições
					\begin{itemize}
						\item O operador tem informações o suficiente para tomar a decisão de caracterizar um roubo ou não.
					\end{itemize}

				\item Atores
					\begin{itemize}
						\item Operador do sistema;
						\item estação de solo.
					\end{itemize}


				\item Fluxo básico
					\begin{enumerate}
						\item O sistema detecta uma atividade suspeita;
						\item O sistema caractegoriza a imagem de acordo com o \textit{ranking};
						\label{item:imagemAcordoRanking}
						\item O sistema ordena as imagens de acordo com sua probabilidade definida no \textit{ranking};
						\item O sistema apresenta as imagens com a moldura da cor específica para o ranking de acordo com a tabela \ref{tab:coresRisco};

						\begin{table}[]
							\centering
							\caption{Cores das molduras das imagens}
								\begin{tabular}{|c|c|}
								\hline
									Risco & Cor \\ \hline
									0 & {\color[HTML]{000000} Cor da moldura} \\ \hline
									1 & {\color[HTML]{9ACD32} Cor da moldura} \\ \hline
									2 & {\color[HTML]{FFFACD} Cor da moldura} \\ \hline
									3 & {\color[HTML]{EEE8AA} Cor da moldura} \\ \hline
									4 & {\color[HTML]{BDB76B} Cor da moldura} \\ \hline
									5 & {\color[HTML]{DAA520} Cor da moldura} \\ \hline
									6 & {\color[HTML]{FFA07A} Cor da moldura} \\ \hline
									7 & {\color[HTML]{A0522D} Cor da moldura} \\ \hline
									8 & {\color[HTML]{CD5C5C} Cor da moldura} \\ \hline
									9 & {\color[HTML]{CD3333} Cor da moldura} \\ \hline
									10 & {\color[HTML]{B22222} Cor da moldura} \\ \hline
									11 & {\color[HTML]{EE3B3B} Cor da moldura} \\ \hline
									12 & {\color[HTML]{FF3030} Cor da moldura} \\ \hline
									13 & {\color[HTML]{FF4500} Cor da moldura} \\ \hline
									14 & {\color[HTML]{FF0000} Cor da moldura} \\ \hline
								\end{tabular}
								\label{tab:coresRisco}
						\end{table}

						\item O sistema mostra o maior ranking em primeiro lugar.
						\label{item:maiorRanking}
					\end{enumerate}

				\item Fluxo alternativo A
					\begin{enumerate}
						\item O sistema está no passo \ref{item:imagemAcordoRanking};
						\item O sistema encontra um erro ao avaliar os pontos para fazer o ranking;
						\item O sistema da a imagem o ranking 5 e coloca nela o aviso ``Não foi possível analisar essa imagem'';
						\item O sistema retorna ao passo \ref{item:maiorRanking} do fluxo principal.
					\end{enumerate}
			\end{enumerate}

				\item UC05 - Observar áreas que merecem mais atenção
					\begin{enumerate}
						\item Descrição

							Este caso de uso se refere ao ato do operador do sistema poder escolher qual imagem quer ver no momento
						\item Pré-condições
							\begin{itemize}
								\item As imagens estão pontuadas de acordo com sua criticidade.
							\end{itemize}

						\item Pós-condições

						O operador pode acionar a segurança.

						\item Atores
							\begin{itemize}
								\item estação de solo;
								\item Operador do sistema.
							\end{itemize}


						\item Fluxo básico
							\begin{enumerate}
								\item O sistema apresenta ao operador diversas imagens;
								\item O operador do sistema seleciona uma imagem;
								\item A imagem fica maior e o operador tem controle da camêra que está gerando aquela imagem;
								\item Fim do fluxo.
							\end{enumerate}
					\end{enumerate}
		\end{itemize}


\subsection{Identificação de Risco} % (fold)
\label{sub:identifica_o_de_risco}


	O sistema SUM, como sabemos, será operado por um operador que terá como responsabilidade observar possíveis casos de roubos a carros. A decisão final sobre a possibilidade de ser um roubo real ou não, cabe ao operador, que terá apoio do sistema para chegar a conclusão final.

	Como o estacionamento da Universidade de Brasília - Campus Gama recebe um número muito grande de carros, é impossível responsabilizar apenas um operador para observar todos os carros ao mesmo tempo, verificando as possibilidades de possíveis roubos ocorrendo, inclusive, em paralelo.

	Para solucionar este problema, o sistema SUM apoiará o operador na escolha de casos suspeitos a serem observados. Ou seja, o sistema apresentará ao operador todos os casos de possíveis roubos ocorrendo no momento, especificando os casos mais importantes e menos importantes.

	A ordenação das imagens mais importantes será feita a partir da utilização de cores na tonalidade avermelhada. As imagens mais críticas serão pintadas de tons avermelhados. Quanto mais crítico, mais próximo do vermelho.

	Utilizando o sistema, o operador saberá exatamente quais imagens merecem atenção e até quais imagens merecem mais atenção que outras imagens, dependendo da quantificação do risco, que é feita pelo sistema. Esta quantificação é feita a partir da observação de critérios que identifiquem um possível caso de roubo a carro.

\subsubsection{Quantificação do Risco}

	Com o objetivo de selecionar as imagens mais importantes a serem analisadas pelo operador, o sistema SUM deverá realizar uma quantificação de critérios que levem a definição de um possível caso de roubo a carro. Estes critérios foram obtidos após a análise de inúmeras imagens que registraram casos de roubo a carros em estacionamentos universitários.

	Os critérios possuem pesos para quantificação, dependendo do quão crítico é o critério analisado. A ponderação dos critérios pode ser observada na tabela \ref{tab:criteriosRisco}:

	\begin{table}[H]
		\centering
		\caption{Identificação dos Critérios de Risco}
		\begin{tabular}{|c|c|c|}
			\hline
			\rowcolor[HTML]{C0C0C0}
			{\color[HTML]{00009B} \textbf{Critérios}}                                   & {\color[HTML]{00009B} \textbf{Descrição}}                                                                                                    & {\color[HTML]{00009B} \textbf{Peso}} \\ \hline
			Proximidade                                                                 & \begin{tabular}[c]{@{}c@{}}Distância de 2m, ou menos, \\ entre um suspeito e o carro \\ analisado.\end{tabular}                              & 1                                    \\ \hline
			\begin{tabular}[c]{@{}c@{}}Permanência próximo \\ ao carro.\end{tabular}    & \begin{tabular}[c]{@{}c@{}}Tempo em que o suspeito \\ permanece ao lado do carro \\ analisado ultrapassa os 30 segundos.\end{tabular}        & 2                                    \\ \hline
			\begin{tabular}[c]{@{}c@{}}Contato físico com a \\ porta.\end{tabular}      & \begin{tabular}[c]{@{}c@{}}O suspeito mantem contato físico com \\ a porta por mais de 10 segundos.\end{tabular}                             & 3                                    \\ \hline
			\begin{tabular}[c]{@{}c@{}}Contato físico com o \\ Porta-Malas\end{tabular} & \begin{tabular}[c]{@{}c@{}}O suspeito mantem contato físico com o \\ porta-malas do carro analisado por mais de \\ 20 segundos.\end{tabular} & 3                                    \\ \hline
			Alarme                                                                      & O alarme do carro analisado está disparando.                                                                                                 & 5                                    \\ \hline
		\end{tabular}
		\label{tab:criteriosRisco}
\end{table}

	O nível de criticidade da imagem analisada será definido a partir do somatório dos critérios observados na imagem.

	Em momento algum o sistema chegará a conclusão de que é um roubo em execução ou não, ele apenas apontará imagens que se enquadram em um possível caso de roubo a carros. A identificação das imagens mais importantes será feita a partir da geração de um Ranking de possíveis casos. Este Ranking será gerado a partir da somatória dos critérios identificados em cada caso.

	O Ranking de imagens será apresentado ao operador na forma de um “mosaico” de imagens, que receberão tons de amarelo a vermelho, dependendo de sua importância no momento. O operador poderá selecionar a imagem para poder controlar a câmera e visualizar a imagem da forma que desejar, verificando se o caso se refere a um caso de roubo ou apenas um engano.

	Para captação destes critérios, o sistema deverá possuir sensores de calor e proximidade, alem das imagens obtidas pelas câmeras.


	\subsection{Viabilidade Econômica}

	Para analisar a viabilidade econômica da Estação Solo, é necessária a identificação e quantificação de todos os gastos que deverão ser realizados para desenvolvimento, manutenção e operação do sistema desenvolvido. Os gastos identificados estão distribuidos em Desenvovimento do Sistema de Software, Infraestrutura e Operação do Sistema. Para a análise a identificação dos gastos relacionados ao Desenvolvimento e Manutenção do sistema, foi utilizada a técnica de Análise de Pontos de Casos de Uso.


	\subsubsection{Análise por Pontos de Casos de Uso}

			A Análise por Pontos de Casos de Uso é um método de Estimativa Funcional de Software que usa os Casos de Uso, criados por Gustav Karner \cite{karner1993resource} em 1993. Esta análise tem como objetivo adaptar a Estimativa por Pontos de Função (APF). Foi criado com o objetivo de viabilizar a estimativa do tamanho funcional do sistema ainda na fase de levantamento dos Casos de Uso, utilizando os próprios documentos gerados nesta fase como \textit{input} para o cálculo da estimativa, como apresenta Herval Freire \cite{freire2003calculando}.

			Segundo \cite{pontosUC}, o processo de contagem utilizando o método de Análise por Pontos de Casos de Uso, que engloba 7 atividades que estão dispostas a seguir:

		\begin{itemize}
			\item Relacionar os Atores do Sistema, classificando-os de acordo com sua complexidade.
			\item Contar os Casos de Uso e atribuir o grau de complexidade de cada um.
			\item Somar o resultado das duas atividades anteriores.
			\item Determinar o fator de complexidade técnica.
			\item Determinar o fator de complexidade ambiental.
			\item Calcular os Pontos de Casos de Uso ajustados.
			\item Calcular a estimativa de horas necessárias para conclusão do sistema.
		\end{itemize}

		Karner \cite{karner1993resource} sugere a utilização de 20 (vinte) homens/hora para cada Ponto de Caso de Uso identificado. Dessa forma, é fácil obter a quantidade de horas necessárias para conclusão do sistema, assim como, o valor aproximado para desenvolvimento do sistema. Para minimizar a mergem de erro, serão contabilizados os Fators Técnicos e Ambientais, que colaboram para a estimativa da complexidade de desenvolvimento do software, aumentando ou não o tempo/custo do projeto.

		Primeiramente deve-se identificar e pontuar os Atores do sistema, como pode ser observado na tabela \ref{tab:atoresSistema}.

		\begin{table}[H]
			\centering
			\caption{Identificação dos Atores do Sistema}
			\label{tab:atoresSistema}
			\begin{tabular}{|c|c|c|c|}
				\hline
				\textbf{Ator} & \textbf{Descrição}                                                                                                                                                                             & \textbf{Complexidade} & \textbf{Peso}                                           \\ \hline
				Balão         & \begin{tabular}[c]{@{}c@{}}Ator responsável por obter todos os dados \\ utilizados para identificação de situações de risco.\end{tabular}                                                      & Simples               & 1                                                       \\ \hline
				Estação Solo  & \begin{tabular}[c]{@{}c@{}}Ator responsável por processar e apresentar os dados \\ obtidos pelo Balão ao Operador. Além de identificar e \\ apontar possíveis situações de risco.\end{tabular} & Simples               & 1                                                       \\ \hline
				Operador      & \begin{tabular}[c]{@{}c@{}}Ator responsável por analisar as situações de risco, \\ verificando a criticidade das mesmas.\end{tabular}                                                          & Complexa              & 3                                                       \\ \hline
				\rowcolor[HTML]{C0C0C0}
				\multicolumn{3}{|c|}{\cellcolor[HTML]{C0C0C0}\textbf{TOTAL}}                                                                                                                                                                           & \multicolumn{1}{l|}{\cellcolor[HTML]{C0C0C0}\textbf{  5}} \\ \hline
			\end{tabular}
		\end{table}

		Após a análise dos Atores, deve-se contar os Casos de Uso do sistema, listando-os e quantificando o número de transições do Caso de Uso e, com isso, obtendo a complexidade do mesmo, como pode ser observado na tabela \ref{tab:idUC}.

		\begin{table}[H]
			\centering
			\caption{Identificação dos Casos de Uso do Sistema}
			\label{tab:idUC}
			\begin{tabular}{|c|c|c|c|c|}
				\hline
				\textbf{Caso de Uso} & \textbf{Descrição}                                                                  & \textbf{\begin{tabular}[c]{@{}c@{}}Nº. de \\ Transações\end{tabular}} & \textbf{Complexidade} & \textbf{Peso} \\ \hline
				UC01                 & \begin{tabular}[c]{@{}c@{}}Captar e processar \\ imagens.\end{tabular}              & 1                                                                     & Simples               & 5             \\ \hline
				UC02                 & \begin{tabular}[c]{@{}c@{}}Gerenciamento de \\ Imagens.\end{tabular}                & 1                                                                     & Simples               & 5             \\ \hline
				UC03                 & \begin{tabular}[c]{@{}c@{}}Informar grau de risco \\ da atividade.\end{tabular}     & 1                                                                     & Simples               & 5             \\ \hline
				UC04                 & \begin{tabular}[c]{@{}c@{}}Identificar situação \\ suspeita.\end{tabular}           & 2                                                                     & Simples               & 5             \\ \hline
				UC05                 & \begin{tabular}[c]{@{}c@{}}Observar áreas que merecem \\ mais atenção.\end{tabular} & 1                                                                     & Simples               & 5             \\ \hline
				\rowcolor[HTML]{C0C0C0}
				\multicolumn{4}{|c|}{\cellcolor[HTML]{C0C0C0}\textbf{TOTAL}}                                                                                                                                               & 25            \\ \hline
			\end{tabular}
		\end{table}

		Com a obtenção das complexidades dos Atores e dos Casos de Uso do sistema, devemos obter o UUCP (Pontos de Casos de Uso) não ajustados, seguindo a seguinte fórmula:

	\begin{center}

		\textbf{UUCP}= UAW + UUCW

		\textbf{UUCP}= 5 + 25

		\textbf{UUCP}= 30
	\end{center}

	Após a obtenção do \textbf{UUCP} (Pontos de Casos de Uso Não ajustados), devemos analisar a complexidade técnica do sistema, que está disposta na tabela \ref{tab:compTecnica}.

	\begin{table}[H]
		\centering
		\caption{Análise da Complexidade Técnica}
		\label{tab:compTecnica}
		\begin{tabular}{|c|c|c|c|c|}
			\hline
			\textbf{Fator} & \textbf{Descrição}                           & \textbf{Peso} & \textbf{Valor} & \textbf{Total} \\ \hline
			T1             & Sistemas Distribuídos.                       & 2             & 4              & \textbf{8}     \\ \hline
			T2             & Tempo de resposta/desempenho.                & 1             & 3              & \textbf{3}     \\ \hline
			T3             & Eficiência do usuário final (on-line)        & 1             & 4              & \textbf{4}     \\ \hline
			T4             & Processamento interno complexo.              & 1             & 4              & \textbf{4}     \\ \hline
			T5             & Reusabilidade do código em outras aplicações & 1             & 0              & \textbf{0}     \\ \hline
			T6             & Facilidade de instalação                     & 0,5           & 3              & \textbf{1,5}   \\ \hline
			T7             & Usabilidade (facilidade operacional)         & 0,5           & 4              & \textbf{2}     \\ \hline
			T8             & Portabilidade                                & 2             & 0              & \textbf{0}     \\ \hline
			T9             & Facilidade de manutenção                     & 1             & 3              & \textbf{3}     \\ \hline
			T10            & Acessos simultâneos (concorrência)           & 1             & 3              & \textbf{3}     \\ \hline
			T11            & Aspectos especiais de segurança              & 1             & 3              & \textbf{3}     \\ \hline
			T12            & Acesso direto para terceiros                 & 1             & 0              & \textbf{0}     \\ \hline
			T13            & Facilidades especiais de treinamento         & 1             & 0              & \textbf{0}     \\ \hline
			\multicolumn{4}{|c|}{\textbf{TOTAL}}                                                           & \textbf{31,5}  \\ \hline
		\end{tabular}
	\end{table}

	A partir da observação da tabela \ref{tab:compTecnica}, basta calcularmos o \textbf{TCF} (Fator de Complexidade Técnica) utilizando a segiinte fórmula:

	\begin{center}
		\textbf{TCF} = 0,6 + ( 0,01 * \textit{TFactor} )

		\textbf{TCF} = 0,6 + (0,01 * 31,5)

		\textbf{TCF = 0,915}
	\end{center}

	Para finalizar, basta apenas analisar os Fatores Ambientais que influenciam no desenvolvimento do sistema, esta análise está disposta na tabela \ref{tab:fatAmbiental}.

	\begin{table}[H]
		\centering
		\caption{Fatores Ambientais}
		\label{tab:fatAmbiental}
		\begin{tabular}{|c|c|c|c|c|}
			\hline
			\textbf{Fator} & \textbf{Descrição}                                                                                           & \textbf{Peso} & \textbf{Valor} & \textbf{Total} \\ \hline
			T1             & \begin{tabular}[c]{@{}c@{}}Familiaridade com o processo de \\ desenvolvimento de software.\end{tabular}      & 1,5           & 4              & \textbf{6}     \\ \hline
			T2             & Experiência na aplicação.                                                                                    & 0,5           & 0              & \textbf{0}     \\ \hline
			T3             & \begin{tabular}[c]{@{}c@{}}Experiência com OO, na linguagem e \\ na técnica de desenvolvimento.\end{tabular} & 1             & 3              & \textbf{3}     \\ \hline
			T4             & Capacidade do Líder de Projeto.                                                                              & 0,5           & 2              & \textbf{1}     \\ \hline
			T5             & Motivação                                                                                                    & 1             & 4              & \textbf{4}     \\ \hline
			T6             & Requisitos estáveis.                                                                                         & 2             & 3              & \textbf{6}     \\ \hline
			T7             & Trabalhadores com dedicação parcial.                                                                         & -1            & 5              & \textbf{-5}    \\ \hline
			T8             & \begin{tabular}[c]{@{}c@{}}Dificuldade da Linguagem \\ de Programação\end{tabular}                           & -1            & 5              & \textbf{-5}    \\ \hline
			\multicolumn{4}{|c|}{\textbf{TOTAL}}                                                                                                                           & \textbf{10}    \\ \hline
		\end{tabular}
	\end{table}

	Utilizando as tabelas \ref{tab:compTecnica} e \ref{tab:fatAmbiental} , chegamos ao resultado do \textit{EFactor} total:
	\begin{center}
		\textbf{EF} = 1,4 + (-0,03 x \textit{EFactor} )

		\textbf{EF} = 1,4 + (-0,03 x 10)

		\textbf{EF} = 1,1
	\end{center}

	O valor final de tudo, o \textbf{UCP} é calculado de acordo com a fórmula:
	\begin{center}
		\textbf{UCP} = UUCP * TCF * EF

		\textbf{UCP} = 30 * 0,915 * 31

		\textbf{UCP} = 31
	\end{center}


	Ou seja, o sistema que será desenvolvido possui 31 Pontos de Casos de Uso Ajustados. Como \cite{karner1993resource} sugere que cada Ponto de Caso de Uso necessita de 20 homens/hora para a conclusão do mesmo, para concluir todo o sistema serão necessários 604 homens/hora.

	\subsubsection{Infraestrutura}

		\begin{table}[H]
\centering
\caption{Tabela de Preços}
\begin{tabular}{|l|l|l|l|}
\hline
                      & Quantidade/Unidades & Preço de mês/Compra & Total    \\ \hline
Waveshare OV5647 Night Vision   & 15       & R\$ 123,34         & R\$ 1850,10 \\ \hline
Seagate Archive 8TB & 4        & R\$ 984,90           & R\$ 3939,60   \\ \hline
Sistema Nobreak       & 1        & R\$ 396,90           & R\$ 396,90      \\ \hline
Hardware              & 1        & R\$ 5271,86         & R\$ 5271,86    \\ \hline
Walk talk Cobra Cxr925 & 5 & R\$ 415,99 & R\$ 2079,95 \\
\hline
Engenheiro de Software & 4 & R\$ 3000,00 & R\$ 12.000,00 \\
\hline
Operador & 4 & R\$ 2000,00 & R\$ 8000,00 \\
\hline
Container & 1 & R\$ 6000,00 & R\$ 6000,00 \\
\hline
\multicolumn{4}{|l|}{Total Gasto com Empregadores}{R\$ 11000,00}                                     \\ \hline
\multicolumn{4}{|l|}{Total Gasto com Equipamentos}{R\$ 19511,41}                                     \\ \hline
\multicolumn{4}{|l|}{Total Gasto}{R\$ 30511,41}                                     \\ \hline
\end{tabular}
\label{tab:precosComponentes}
\end{table}
A tabela \ref{tab:precosComponentes} demonstra os custos do HD, sistema noBreak, walk talk e hardware necessários para suportar esse poderoso HD. Como explicado anteriormente o sistema nobreak é utilizado caso ocorra algum problema que possa ser suspenso o uso de energia, como chuvas ou problemas internos/externos. O HD será utilizado para armazenar as imagens de ações suspeitas caso seja detectada alguma ação da mesma, as configurações de hardware para suportar tal HD e o sistema de comunicação entre o operador na estação de solo e os seguranças que patrularam a area do campus. Além de conter os gastos totais com os engenheiros de software responsaveis por projetar o sistema, o operador das câmeras e o container onde ficara a estação solo.
% section subprojeto_da_esta_o_de_solo (end)

\section{Consumo Energético} % (fold)
\label{sec:consumo_energ_tico}
Inicialmente, foram analisadas as propostas para o fornecimento energético do Sistema Integrado de Monitoramento, foram elas, eólica, solar, e também a possibilidade do sistema ser ligado à rede concessionária, ou seja, um sistema \textit{on grid}.

A tomada de decisão, levou em conta fatores relevantes que poderiam afetar o bom desempenho do equipamento, estes foram postos em pauta e devidamente justificados no Ponto de Controle 01 (PC01). Com base nos pontos positivos e negativos analisados de cada proposta, a escolha que apresentou maior viabilidade para agregar ao projeto foi o sistema ligado na rede, assim como a proposta que atende a maioria das casas no Brasil.

Contudo, ainda assim serão dispostos nesta seção os dados referentes à viabilidade econômica de cada situação analisada.

\subsection{Energia Solar}

Painéis solares fotovoltaicos são dispositivos usualmente utilizados para conversão de energia proveniente do sol, em energia elétrica. Estruturalmente, painéis solares são compostos por células solares, capazes de criar uma DDP (Diferença de Potencial) através do fluxo de corrente elétrica.

Para a avaliação da viabilidade econômica no uso de painéis fotovoltaicos para alimentação do sistema SUM, foi utilizado um simulador disponível no site \href{http://www.portalsolar.com.br/calculo-solar}{Portal Solar}(figura \ref{img:Simulador Solar}), que a partir de dados sobre Estado, cidade mais próxima e consumo mensal em kWh, pôde nos fornecer uma ficha técnica do sistema gerador.

\begin{figure}[H]
	\centering
	\caption{Entrada de dados para simulação de dados ~\cite{simulador}.}
	\includegraphics[width=0.8\textwidth]{figuras/simuladorDeCustos}
	\label{img:Simulador Solar}
\end{figure}

Na figura \ref{img:Gerador} são apresentados os dados referentes à capacidade em kWp, produção anual de energia, área mínima ocupada pelo sistema fotovoltaico, peso médio por metro quadrado e quantidade de placas. Os valores foram aproximados para o preço médio de instalação de painéis fotovoltaicos de julho de 2015. O consumo de todo o sistema SUM será de 4500 kWh/mês, e os custos para suprir esta demanda variam de R\$ 205.592,11 a  R\$ 268.851,21. Teriam de ser utilizadas 127 placas de 250 Watts.

\begin{figure}[H]
	\centering
	\caption{Ficha técnica do sistema gerador fotovoltaico do SUM ~\cite{simulador}.}
	\includegraphics[width=0.6\textwidth]{figuras/fichaTecnicaSistemaGeradorPc3}
	\label{img:Gerador}
\end{figure}

O simulador também fornece a geração mensal de energia (figura \ref{img:Geracao de energia solar}).

\begin{figure}[htp]
	\centering
	\caption{Geração de energia solar durante o ano ~\cite{simulador}.}
	\includegraphics[width=0.6\textwidth]{figuras/consumoSolar}
	\label{img:Geracao de energia solar}
\end{figure}

\subsubsection{Custos de Manutenção}

Os custos de manutenção e operação dos painéis fotovoltaicos são relativamente baixos, quando falamos de poucas unidades geradoras. Na simulação feita para atender a demanda do conjunto, o número elevado de placas elevou também os custos de manutenção.

O grupo fez uma consultoria com a empresa \href{http://www.smartly.com.br}{Smartly}, localizada no SIA, em Brasília-DF e apresentou um orçamento de manutenção, custeado em 20 R\$ /kWh (20 reais por kWh) gerado.

\subsection{Energia Eólica}

A energia eólica  tem como definição ser uma energia obtida a partir da conversão da energia cinética do vento, em energia elétrica .Uma energia a qual usufrui de uma fonte renovável e limpa.

Considerada uma das mais conhecidas energias renováveis, é importante ressaltar que isso aconteceu pelo crescimento desse mercado em todo o mundo\cite{renovaveis}.

Devido ao caráter inovador, e a fim de atender a demanda energética das estruturas do balão, a energia eólica chamou a atenção, e foi uma das analisadas para o suprir as necessidades do SUM. A UnB- Faculdade do Gama (FGA) local o qual se intenciona instalar o balão, apresenta algumas condições favoráveis para esse tipo de energia, como por exemplo: fortes ventos durante grandes períodos do ano, clima estável e com pouca frequência de alterações climáticas significativas as quais poderiam afetar o funcionamento do sistema.

A busca de um equilíbrio entre o custo inicial de implantação, e bons custos de operação, são os objetivos de qualquer escolha, e não seria diferente com o sistema de alimentação do balão.

Instalações eólicas possuem um elevado custo de implantação inicial, mas baixo custo de operação e manutenção.

O cálculo de implantação de aerogeradores para atender a demanda de todo o conjunto do SUM, leva em conta a demanda energética, que é relativamente alta, no que se refere a sistemas de geração eólica. Fatores como manutenção, local de implantação e danos ao equipamento também devem ser levados em conta \cite{eolica}.

Para atender a demanda energética do SUM com aerogeradores seriam necessários aproximadamente cinco aerogeradores com capacidade produtiva de 1000 Watts, para suprir os 4500KWh de demanda do conjunto. Devemos lembrar que um sistema eólico não opera em produção máxima o tempo todo, e que para garantir a segurança e estabilidade do sistema, um sistema de baterias também deve ser implementado para reduzir os riscos de operação.

Para 5 aerogeradores do modelo Vawt 1000W, com poder de geração de 1000 Watts, o projeto teria como custo de 11.500,00 reais, um adicional de instalação e um banco de baterias no valor de 18.000,00 reais, totalizando 29.500,00 de custo inicial simplificado\cite{aerogerador}.

Com base nos dados e posicionamentos supracitados, podemos perceber que a energia eólica, devido a instabilidade operacional, e dependência dos fatores climáticos, não pode ser considerada uma boa escolha para um sistema de segurança, além de possuir um custo inicial relativamente alto.

\subsection{Energia On Grid}

O sistema ligado on grid, é aquele que trabalha em conjunto com a rede elétrica que é fornecida pela concessionária responsável pela distribuição na região.

A composição tarifária da concessionária de distribuição é composta por duas parcelas: Parcela A e B, a parcela A, é composta pelos custos nos quais são considerados não-gerenciáveis, ou seja, aqueles controlados pela Agência Reguladora, também estão nessa parcela a compra de energia, os encargos setoriais e os encargos de transmissão. Já os custos da Parcela B são os considerados gerenciáveis, os quais, são administrados pela concessionária. Ela é composta por despesas de operação e manutenção, despesas de capital, entre outros.

De acordo com a CEB, as tarifas de energia são dividias em dois grupos: Grupo A, que são os clientes de alta tensão, igual ou superior a 2,3kV, e Grupo B, que são os de baixa tensão, inferior a 2,3kV.

Os custos de implantação de um sistema OnGrid, depende de estruturas como os cabos e o quadro de distribuição, os quais apresentam os custos de 500,40 reais para 600 metros de cabo (88,40R\$/m), e 1500,00 reais para o QGBT (6 Switches). Valores estipulados de acordo com a média de mercado dos produtos, totalizando 2,222,40 reais para fins estruturais.

Nas condições de uso do SUM, atendendo a todas as necessidades, o consumo energético é de aproximadamente 4.500kWh/mês, se enquadrando assim no perfil da CEB de consumidores de alta demanda, tendo um ajuste no preço do kWh, chegando aos 0,4306 centavos por kW consumido\cite{tarifas}.

De acordo com o apresentado, podemos fazer uma análise simplificada dos custos estimados de operação, e pontos onde a economia de energia poderia fortificar a proposta de uma ligação On Grid.

Conforme os dados supracitados, conseguimos analisar o grau de viabilidade de cada escolha, baseado nos seus custos de investimento inicial, aplicação  em conformidade com a posição geográfica da UnB-FGA e custos de operação.
\\

\begin{tabular}{|l|l|l|}
\hline
	\textbf{Tipo de Energia} & \textbf{Custo kWh (R\$)} & \textbf{Custo Estrutura (R\$)}\\
\hline
	\textbf{Solar} & 2,25 & 205.592,00\\
\hline
	\textbf{Eólica} & 99,64 & 29.500,00\\
\hline
	\textbf{OnGrid} & 0,43 & 2.000,00\\
\hline
\end{tabular}\\

A Energia Solar apresenta um alto custo de implantação, inviabilidade  pelo espaço ocupado pelas placas a serem instaladas, as quais devem ser instaladas em áreas expostas e que apresentam alto índice de luminosidade e baixo custo de manutenção, que é um fator muito relevante quando estamos falando de custos de operação.

A utilização da Energia Eólica, usando como fonte geradora um Aerogerador, apresenta um relativo baixo custo inicial de implantação e estruturas, elevado grau de eficiência, levando em consideração as características de vento na região do Gama-DF, contudo, não é um sistema que oferece a estabilidade e confiança de fornecimento energético adequado para alimentar um sistema de monitoramento, onde a confiabilidade é um dos pontos cruciais.

A proposta de fazer uma ligação simplificada, ligada no quadro de recebimento da concessionária fornecedora de energia , a CEB, no caso de Brasília, apresenta baixo custo inicial de implantação, existe a disponibilidade de pontos de alta e baixa tensão na região, custo de manutenção é quase nulo,  e os custos de operação são regulados em parte pelo Governo, e em parte pela CEB, levando em conta a disponibilidade hídrica.

A confiabilidade e estabilidade de um fornecimento energético é extremamente importante para um sistema onde a segurança é o objetivo principal.

A escolha de alimentar o SUM, por um sistema On Grid foi embasada nos seguinte fatores: menor custo de implantação, menor custo de operação e maior confiabilidade que os outras opções apresentadas.

O fornecimento energético do SUM, será feito semelhante à uma residência normal, como mostrado na figura \ref{img:Energetico}.

\begin{figure}[H]
	\centering
	\caption{Diagrama Energético - Balão Cativo.}
	\includegraphics[width=0.6\textwidth]{figuras/Energetico}
	\label{img:Energetico}
\end{figure}

 \subsection{QGBT (Quadro Geral de Baixa Tensão):}
 Painéis que acomodam o equipamentos para a proteção, seccionamento e manobra de energia elétrica. Os painéis podem variar de tamanho, desde residenciais até painéis de grandes de indústrias, edificações comerciais, hospitais, entre outros. O dimensionamento do QGBT é de grande importância para manter a integridade e a segurança de todo o sistema.

\subsection{Banco de Baterias:}

As baterias elétricas têm sido utilizadas principalmente voltada a elementos acumuladores em sistemas de alimenta\c{c}ão ininterruptos. O banco de baterias é o conjunto delas ligadas em paralelo ou série-paralelo, visando atender à necessidade do sistema em situações emergenciais. O dimensionamento correto do banco de baterias irá garantir a continuidade das operações do sistema por um determinado período de tempo. O banco de baterias pode ser observado na imagem \ref{img:baterias}.

\begin{figure}[H]
	\centering
	\caption{Banco de Baterias.}
	\includegraphics[width=0.6\textwidth]{figuras/baterias}
	\label{img:baterias}
\end{figure}

\subsubsection{Dimensionamento do Banco de Baterias:}

Funciona como um sistema emergencial, quando a energia fornecida pela rede falha, ou não é o suficiente para o bom funcionamento do conjunto. O objetivo é manter funcionando apenas as estruturas vitais, para que a segurança não seja prejudicada. No caso, manteremos apenas os monitores, um computador, e a estrutura do balão, resultando num consumo de 2015,55 kWh, para ser suprido pelas baterias por um período máximo de 6 horas de autonomia.

Com auxílio da calculadora Digitek, baterias estacionárias, com 12 Volts e 70 ampéres, um inversor com eficiência próxima de 80\%, conseguimos a disposição apresentada na imagem \ref{img:calculo}.

\begin{figure}[H]
	\centering
	\caption[Cálculo de autonomia de Banco de baterias]{Cálculo de autonomia de Banco de baterias~\cite{digitek}}
	\includegraphics[width=0.5\textwidth]{figuras/calculo}
	\label{img:calculo}
\end{figure}


Um banco de 18 unidades de bateria de chumbo (Pb), 12 Volts e 70A, ligadas em paralelo, vão atender à necessidade do sistema por um período de 6 horas.

Em face do exposto, e fazendo uma breve análise do que foi apresentado, temos todas as estruturas que consumirão energia dispostas em forma de esquemáticos indicando seus respectivos consumos, dimensionamento dos fios de acordo com a NBR 5410 da ABNT, dimensionamento do banco de bateria para situações emergenciais e análise do quanto será demandado de energia para o funcionamento completo do SUM.

\subsection{Elementos consumidores do SUM}

Abaixo, temos um resumo esquemático simplificado das estruturas que são as potenciais consumidoras do conjunto, sendo assim, utilizadas como base de cálculo para o dimensionamento do banco de baterias, e dos fios de transmissão.

Para isso, separamos em dois blocos independentes, a Sala de Monitoramento (SM) e o Sistema do Balão (SB)  para o desenvolvimento dos cálculos, considerando o fornecimento da Companhia Energética de Brasília (CEB), com 220 Volts, 60Hz.~\cite{ceb}

O cálculo do consumo mensal de cada equipamento, tanto da sala de monitoramento quanto do sistema do balão, foi dado pelo produto da potência do equipamento em watt, pela quantidade de horas que ele é utilizado ao dia e o número de dias do mês, isso sendo dividido por 1000. Os equipamentos serão utilizados 24h por dia, e para a realização dos cálculos, o mês foi considerado de 30 dias.

Quando feito o somatório dos valores do consumo mensal de cada equipamento, é obtido o consumo mensal da sala de monitoramento e também o consumo mensal do sistema do balão.

$Consumo_{total} =  \sum{\frac{P_w \cdot 24h \cdot 30d}{1000}}$

 \subsubsection{Sala de Monitoramento:}

 A sala de monitoramento, apresenta em seu âmbito os equipamentos de observação e processamento dos dados, como monitores, backup, com funcionamento ininterrupto, já que estamos tratando de um sistema de segurança, operando na vigência 24/7.

Como podemos imaginar, a sala de monitoramento será a maior consumidora de energia do conjunto, por apresentar dispositivos que tornem o ambiente favorável ao trabalho de monitoramento, como: monitores, iluminação, temperatura adequada para operação dos equipamentos, entre outros equipamentos auxiliares. O dimensionamento da sala de deve levar em conta as necessidades mínimas para que em situações emergenciais o sistema continue operando sem prejuízos.

A Sala de Monitoramento, apresenta dois monitores Samsung 42’’, dois computadores de alta performance, um aparelho de ar condicionado LG de 12.000 BTUs, e oito luminárias tubulares de LED (\textit{Light Emission Diode}).

Na imagem \ref{img:salaMonitoramento} é possivel observar um esquemático com os resultados dos cálculos do consumo energético dos equipamentos e da sala.

\begin{figure}[H]
	\centering
	\caption{Sala de monitoramento}
	\includegraphics[width=0.6\textwidth]{figuras/salaMonitoramento}
	\label{img:salaMonitoramento}
\end{figure}


\subsubsection{Sistema do Balão}

O Sistema do Balão, apresenta menor consumo, e também menos itens, temos a payload, composta por sensores, eletrônica e as câmeras, e também o sistema de guinchos, utilizados no recolhimento do balão.
Na imagem \ref{img:sistemaBalao} pode-se obter um esquemático com os resultados dos cálculos do consumo energético dos equipamentos e do sistema.

\begin{figure}[H]
	\centering
	\caption{Sistema do balão}
	\includegraphics[width=0.6\textwidth]{figuras/sistemaBalao}
	\label{img:sistemaBalao}
\end{figure}


Para a determinação da espessura dos fios, utilizamos a Norma Brasileira de Regulamentação 5410/2004, referente a instalações elétricas de baixa tensão. Conforme a tabela abaixo, para as estruturas luminosas, serão utilizados fio de 1.5mm de espessura, para as tomadas de energia e dispositivos para conectar os equipamentos na Sala de Monitoramento, serão utilizados fios de 2.5mm de espessura. Para a alimentação do balão, por ser uma estrutura de baixa demanda energética, será utilizado um fio de 0.5mm de espessura, conforme determina a NBR. A bitola do fio pode ser observada na imagem \ref{img:BitoladoFio}.

\begin{figure}[H]
	\centering
	\caption{Bitola do Fio}
	\includegraphics[width=0.6\textwidth]{figuras/BitoladoFio}
	\label{img:BitoladoFio}
\end{figure}


% section consumo_energ_tico (end)

\section{Integração da Solução} % (fold)
\label{sec:integra_o_da_solu_o}
Para a melhor visualização do funcionamento integrado do SUM, apresenta-se uma situação hipotética de atividade suspeita na (Figura \ref{img:integracao}). A figura descreve a sequência de operação do SUM, desde o reconhecimento da atividade suspeita até o acionamento da segurança.

\begin{figure}[H]
	\centering
	\caption{Ilustração de situação de risco e funcionamento do SUM.}
	\includegraphics[width=0.8\textwidth]{figuras/integracao}
	\label{img:integracao}
\end{figure}

A imagem está enumerada de acordo a seguinte sequência de cenas:

\begin{enumerate}
	\item Um indivíduo estaciona próximo ao prédio e, juntamente com seu filho, caminha em direção à entrada deste. Ao chegarem perto da entrada, uma pessoa começa a se aproximar de seu carro. Ela espera até que entrem no prédio e então permanece ali durante o período de 5 minutos.

	\item As câmeras, contidas nos balões próximos ao carro, captarão estas imagens que, em seguida, serão passadas por um processamento de imagens, executados pelo sistema eletrônico presente no \textit{Payload} (suspenso pelo balão). Nesta fase de processamento de sinais, as câmeras direcionarão suas imagens para o Raspberry PI (microcontrolador responsável pelo processamento de imagens), através do seu conector específico para câmera.
	Este conterá um algoritmo que realizará a compressão dos vídeos enviados, no padrão H264, um padrão de codificação de vídeo de última geração que utiliza um quadro de referência para comparação e codifica apenas os pixels que foram modificados. Após realizar esse processo de compressão, o Raspberry, que estará conectado ao Arduino, receberá os dados dos sensores e a interpretação feita pelo algoritmo implementado no microcontrolador. Caso a interpretação do algoritmo informe determinada anomalia no sistema e exija que esse seja estabilizado, o Raspberry comunicará ao Galileo, via comunicação serial, que realize a estabilização do sistema.

	\item Com as imagens e os dados dos sensores obtidos, o Raspberry transmitirá essas informações em tempo real para o balão mais próximo da central, a comunicação entre eles será sem fio, montando uma rede intranet.

	\item O balão, que estará recebendo essas informações, transmitirá para a central (container) através de um cabo de \textit{Ethernet} e o computador que receber tais informações realizará todo o procedimento necessário com as imagens.
	
	\item Com estas imagens e os dados dos sensores transmitidos, será realizado o reconhecimento da situação em questão e será verificado o funcionamento do sistema. A estação de solo possuirá um conjunto de monitores e um monitor central. Após a transmissão das informações, serão emitidos alertas no monitor central, onde o operador irá averiguar a situação e, caso seja necessário, informar a segurança do Campus via rádio. O segurança do Campus irá até o local e verificará a situação, tendo como função informar às autoridades responsáveis pelo Campus.
\end{enumerate}

% section integra_o_da_solu_o (end)
