Inicialmente, foram analisadas as propostas para o fornecimento energético do Sistema Integrado de Monitoramento, foram elas, eólica, solar, e também a possibilidade do sistema ser ligado à rede concessionária, ou seja, um sistema \textit{on grid}.

A tomada de decisão, levou em conta fatores relevantes que poderiam afetar o bom desempenho do equipamento, estes foram postos em pauta e devidamente justificados no Ponto de Controle 01 (PC01). Com base nos pontos positivos e negativos analisados de cada proposta, a escolha que apresentou maior viabilidade para agregar ao projeto foi o sistema ligado na rede, assim como a proposta que atende a maioria das casas no Brasil.

Contudo, ainda assim serão dispostos nesta seção os dados referentes à viabilidade econômica de cada situação analisada.

\subsection{Energia Solar}

Painéis solares fotovoltaicos são dispositivos usualmente utilizados para conversão de energia proveniente do sol, em energia elétrica. Estruturalmente, painéis solares são compostos por células solares, capazes de criar uma DDP (Diferença de Potencial) através do fluxo de corrente elétrica.

Para a avaliação da viabilidade econômica no uso de painéis fotovoltaicos para alimentação do sistema SUM, foi utilizado um simulador disponível no site \href{http://www.portalsolar.com.br/calculo-solar}{Portal Solar}(figura \ref{img:Simulador Solar}), que a partir de dados sobre Estado, cidade mais próxima e consumo mensal em kWh, pôde nos fornecer uma ficha técnica do sistema gerador.

\begin{figure}[H]
	\centering
	\caption{Entrada de dados para simulação de dados ~\cite{simulador}.}
	\includegraphics[width=0.8\textwidth]{figuras/simuladorDeCustos}
	\label{img:Simulador Solar}
\end{figure}

Na figura \ref{img:Gerador} são apresentados os dados referentes à capacidade em kWp, produção anual de energia, área mínima ocupada pelo sistema fotovoltaico, peso médio por metro quadrado e quantidade de placas. Os valores foram aproximados para o preço médio de instalação de painéis fotovoltaicos de julho de 2015. O consumo de todo o sistema SUM será de 4500 kWh/mês, e os custos para suprir esta demanda variam de R\$ 205.592,11 a  R\$ 268.851,21. Teriam de ser utilizadas 127 placas de 250 Watts.

\begin{figure}[H]
	\centering
	\caption{Ficha técnica do sistema gerador fotovoltaico do SUM ~\cite{simulador}.}
	\includegraphics[width=0.6\textwidth]{figuras/fichaTecnicaSistemaGeradorPc3}
	\label{img:Gerador}
\end{figure}

O simulador também fornece a geração mensal de energia (figura \ref{img:Geracao de energia solar}).

\begin{figure}[htp]
	\centering
	\caption{Geração de energia solar durante o ano ~\cite{simulador}.}
	\includegraphics[width=0.6\textwidth]{figuras/consumoSolar}
	\label{img:Geracao de energia solar}
\end{figure}

\subsubsection{Custos de Manutenção}

Os custos de manutenção e operação dos painéis fotovoltaicos são relativamente baixos, quando falamos de poucas unidades geradoras. Na simulação feita para atender a demanda do conjunto, o número elevado de placas elevou também os custos de manutenção.

O grupo fez uma consultoria com a empresa \href{http://www.smartly.com.br}{Smartly}, localizada no SIA, em Brasília-DF e apresentou um orçamento de manutenção, custeado em 20 R\$ /kWh (20 reais por kWh) gerado.

\subsection{Energia Eólica}

A energia eólica  tem como definição ser uma energia obtida a partir da conversão da energia cinética do vento, em energia elétrica .Uma energia a qual usufrui de uma fonte renovável e limpa.

Considerada uma das mais conhecidas energias renováveis, é importante ressaltar que isso aconteceu pelo crescimento desse mercado em todo o mundo\cite{renovaveis}.

Devido ao caráter inovador, e a fim de atender a demanda energética das estruturas do balão, a energia eólica chamou a atenção, e foi uma das analisadas para o suprir as necessidades do SUM. A UnB- Faculdade do Gama (FGA) local o qual se intenciona instalar o balão, apresenta algumas condições favoráveis para esse tipo de energia, como por exemplo: fortes ventos durante grandes períodos do ano, clima estável e com pouca frequência de alterações climáticas significativas as quais poderiam afetar o funcionamento do sistema.

A busca de um equilíbrio entre o custo inicial de implantação, e bons custos de operação, são os objetivos de qualquer escolha, e não seria diferente com o sistema de alimentação do balão.

Instalações eólicas possuem um elevado custo de implantação inicial, mas baixo custo de operação e manutenção.

O cálculo de implantação de aerogeradores para atender a demanda de todo o conjunto do SUM, leva em conta a demanda energética, que é relativamente alta, no que se refere a sistemas de geração eólica. Fatores como manutenção, local de implantação e danos ao equipamento também devem ser levados em conta \cite{eolica}.

Para atender a demanda energética do SUM com aerogeradores seriam necessários aproximadamente cinco aerogeradores com capacidade produtiva de 1000 Watts, para suprir os 4500KWh de demanda do conjunto. Devemos lembrar que um sistema eólico não opera em produção máxima o tempo todo, e que para garantir a segurança e estabilidade do sistema, um sistema de baterias também deve ser implementado para reduzir os riscos de operação.

Para 5 aerogeradores do modelo Vawt 1000W, com poder de geração de 1000 Watts, o projeto teria como custo de 11.500,00 reais, um adicional de instalação e um banco de baterias no valor de 18.000,00 reais, totalizando 29.500,00 de custo inicial simplificado\cite{aerogerador}.

Com base nos dados e posicionamentos supracitados, podemos perceber que a energia eólica, devido a instabilidade operacional, e dependência dos fatores climáticos, não pode ser considerada uma boa escolha para um sistema de segurança, além de possuir um custo inicial relativamente alto.

\subsection{Energia On Grid}

O sistema ligado on grid, é aquele que trabalha em conjunto com a rede elétrica que é fornecida pela concessionária responsável pela distribuição na região.

A composição tarifária da concessionária de distribuição é composta por duas parcelas: Parcela A e B, a parcela A, é composta pelos custos nos quais são considerados não-gerenciáveis, ou seja, aqueles controlados pela Agência Reguladora, também estão nessa parcela a compra de energia, os encargos setoriais e os encargos de transmissão. Já os custos da Parcela B são os considerados gerenciáveis, os quais, são administrados pela concessionária. Ela é composta por despesas de operação e manutenção, despesas de capital, entre outros.

De acordo com a CEB, as tarifas de energia são dividias em dois grupos: Grupo A, que são os clientes de alta tensão, igual ou superior a 2,3kV, e Grupo B, que são os de baixa tensão, inferior a 2,3kV.

Os custos de implantação de um sistema OnGrid, depende de estruturas como os cabos e o quadro de distribuição, os quais apresentam os custos de 500,40 reais para 600 metros de cabo (88,40R\$/m), e 1500,00 reais para o QGBT (6 Switches). Valores estipulados de acordo com a média de mercado dos produtos, totalizando 2,222,40 reais para fins estruturais.

Nas condições de uso do SUM, atendendo a todas as necessidades, o consumo energético é de aproximadamente 4.500kWh/mês, se enquadrando assim no perfil da CEB de consumidores de alta demanda, tendo um ajuste no preço do kWh, chegando aos 0,4306 centavos por kW consumido\cite{tarifas}.

De acordo com o apresentado, podemos fazer uma análise simplificada dos custos estimados de operação, e pontos onde a economia de energia poderia fortificar a proposta de uma ligação On Grid.

Conforme os dados supracitados, conseguimos analisar o grau de viabilidade de cada escolha, baseado nos seus custos de investimento inicial, aplicação  em conformidade com a posição geográfica da UnB-FGA e custos de operação.
\\

\begin{tabular}{|l|l|l|}
\hline
	\textbf{Tipo de Energia} & \textbf{Custo kWh (R\$)} & \textbf{Custo Estrutura (R\$)}\\
\hline
	\textbf{Solar} & 2,25 & 205.592,00\\
\hline
	\textbf{Eólica} & 99,64 & 29.500,00\\
\hline
	\textbf{OnGrid} & 0,43 & 2.000,00\\
\hline
\end{tabular}\\

A Energia Solar apresenta um alto custo de implantação, inviabilidade  pelo espaço ocupado pelas placas a serem instaladas, as quais devem ser instaladas em áreas expostas e que apresentam alto índice de luminosidade e baixo custo de manutenção, que é um fator muito relevante quando estamos falando de custos de operação.

A utilização da Energia Eólica, usando como fonte geradora um Aerogerador, apresenta um relativo baixo custo inicial de implantação e estruturas, elevado grau de eficiência, levando em consideração as características de vento na região do Gama-DF, contudo, não é um sistema que oferece a estabilidade e confiança de fornecimento energético adequado para alimentar um sistema de monitoramento, onde a confiabilidade é um dos pontos cruciais.

A proposta de fazer uma ligação simplificada, ligada no quadro de recebimento da concessionária fornecedora de energia , a CEB, no caso de Brasília, apresenta baixo custo inicial de implantação, existe a disponibilidade de pontos de alta e baixa tensão na região, custo de manutenção é quase nulo,  e os custos de operação são regulados em parte pelo Governo, e em parte pela CEB, levando em conta a disponibilidade hídrica.

A confiabilidade e estabilidade de um fornecimento energético é extremamente importante para um sistema onde a segurança é o objetivo principal.

A escolha de alimentar o SUM, por um sistema On Grid foi embasada nos seguinte fatores: menor custo de implantação, menor custo de operação e maior confiabilidade que os outras opções apresentadas.

O fornecimento energético do SUM, será feito semelhante à uma residência normal, como mostrado na figura \ref{img:Energetico}.

\begin{figure}[H]
	\centering
	\caption{Diagrama Energético - Balão Cativo.}
	\includegraphics[width=0.6\textwidth]{figuras/Energetico}
	\label{img:Energetico}
\end{figure}

 \subsection{QGBT (Quadro Geral de Baixa Tensão):}
 Painéis que acomodam o equipamentos para a proteção, seccionamento e manobra de energia elétrica. Os painéis podem variar de tamanho, desde residenciais até painéis de grandes de indústrias, edificações comerciais, hospitais, entre outros. O dimensionamento do QGBT é de grande importância para manter a integridade e a segurança de todo o sistema.

\subsection{Banco de Baterias:}

As baterias elétricas têm sido utilizadas principalmente voltada a elementos acumuladores em sistemas de alimenta\c{c}ão ininterruptos. O banco de baterias é o conjunto delas ligadas em paralelo ou série-paralelo, visando atender à necessidade do sistema em situações emergenciais. O dimensionamento correto do banco de baterias irá garantir a continuidade das operações do sistema por um determinado período de tempo. O banco de baterias pode ser observado na imagem \ref{img:baterias}.

\begin{figure}[H]
	\centering
	\caption{Banco de Baterias.}
	\includegraphics[width=0.6\textwidth]{figuras/baterias}
	\label{img:baterias}
\end{figure}

\subsubsection{Dimensionamento do Banco de Baterias:}

Funciona como um sistema emergencial, quando a energia fornecida pela rede falha, ou não é o suficiente para o bom funcionamento do conjunto. O objetivo é manter funcionando apenas as estruturas vitais, para que a segurança não seja prejudicada. No caso, manteremos apenas os monitores, um computador, e a estrutura do balão, resultando num consumo de 2015,55 kWh, para ser suprido pelas baterias por um período máximo de 6 horas de autonomia.

Com auxílio da calculadora Digitek, baterias estacionárias, com 12 Volts e 70 ampéres, um inversor com eficiência próxima de 80\%, conseguimos a disposição apresentada na imagem \ref{img:calculo}.

\begin{figure}[H]
	\centering
	\caption[Cálculo de autonomia de Banco de baterias]{Cálculo de autonomia de Banco de baterias~\cite{digitek}}
	\includegraphics[width=0.5\textwidth]{figuras/calculo}
	\label{img:calculo}
\end{figure}


Um banco de 18 unidades de bateria de chumbo (Pb), 12 Volts e 70A, ligadas em paralelo, vão atender à necessidade do sistema por um período de 6 horas.

Em face do exposto, e fazendo uma breve análise do que foi apresentado, temos todas as estruturas que consumirão energia dispostas em forma de esquemáticos indicando seus respectivos consumos, dimensionamento dos fios de acordo com a NBR 5410 da ABNT, dimensionamento do banco de bateria para situações emergenciais e análise do quanto será demandado de energia para o funcionamento completo do SUM.

\subsection{Elementos consumidores do SUM}

Abaixo, temos um resumo esquemático simplificado das estruturas que são as potenciais consumidoras do conjunto, sendo assim, utilizadas como base de cálculo para o dimensionamento do banco de baterias, e dos fios de transmissão.

Para isso, separamos em dois blocos independentes, a Sala de Monitoramento (SM) e o Sistema do Balão (SB)  para o desenvolvimento dos cálculos, considerando o fornecimento da Companhia Energética de Brasília (CEB), com 220 Volts, 60Hz.~\cite{ceb}

O cálculo do consumo mensal de cada equipamento, tanto da sala de monitoramento quanto do sistema do balão, foi dado pelo produto da potência do equipamento em watt, pela quantidade de horas que ele é utilizado ao dia e o número de dias do mês, isso sendo dividido por 1000. Os equipamentos serão utilizados 24h por dia, e para a realização dos cálculos, o mês foi considerado de 30 dias.

Quando feito o somatório dos valores do consumo mensal de cada equipamento, é obtido o consumo mensal da sala de monitoramento e também o consumo mensal do sistema do balão.

$Consumo_{total} =  \sum{\frac{P_w \cdot 24h \cdot 30d}{1000}}$

 \subsubsection{Sala de Monitoramento:}

 A sala de monitoramento, apresenta em seu âmbito os equipamentos de observação e processamento dos dados, como monitores, backup, com funcionamento ininterrupto, já que estamos tratando de um sistema de segurança, operando na vigência 24/7.

Como podemos imaginar, a sala de monitoramento será a maior consumidora de energia do conjunto, por apresentar dispositivos que tornem o ambiente favorável ao trabalho de monitoramento, como: monitores, iluminação, temperatura adequada para operação dos equipamentos, entre outros equipamentos auxiliares. O dimensionamento da sala de deve levar em conta as necessidades mínimas para que em situações emergenciais o sistema continue operando sem prejuízos.

A Sala de Monitoramento, apresenta dois monitores Samsung 42’’, dois computadores de alta performance, um aparelho de ar condicionado LG de 12.000 BTUs, e oito luminárias tubulares de LED (\textit{Light Emission Diode}).

Na imagem \ref{img:salaMonitoramento} é possivel observar um esquemático com os resultados dos cálculos do consumo energético dos equipamentos e da sala.

\begin{figure}[H]
	\centering
	\caption{Sala de monitoramento}
	\includegraphics[width=0.6\textwidth]{figuras/salaMonitoramento}
	\label{img:salaMonitoramento}
\end{figure}


\subsubsection{Sistema do Balão}

O Sistema do Balão, apresenta menor consumo, e também menos itens, temos a payload, composta por sensores, eletrônica e as câmeras, e também o sistema de guinchos, utilizados no recolhimento do balão.
Na imagem \ref{img:sistemaBalao} pode-se obter um esquemático com os resultados dos cálculos do consumo energético dos equipamentos e do sistema.

\begin{figure}[H]
	\centering
	\caption{Sistema do balão}
	\includegraphics[width=0.6\textwidth]{figuras/sistemaBalao}
	\label{img:sistemaBalao}
\end{figure}


Para a determinação da espessura dos fios, utilizamos a Norma Brasileira de Regulamentação 5410/2004, referente a instalações elétricas de baixa tensão. Conforme a tabela abaixo, para as estruturas luminosas, serão utilizados fio de 1.5mm de espessura, para as tomadas de energia e dispositivos para conectar os equipamentos na Sala de Monitoramento, serão utilizados fios de 2.5mm de espessura. Para a alimentação do balão, por ser uma estrutura de baixa demanda energética, será utilizado um fio de 0.5mm de espessura, conforme determina a NBR. A bitola do fio pode ser observada na imagem \ref{img:BitoladoFio}.

\begin{figure}[H]
	\centering
	\caption{Bitola do Fio}
	\includegraphics[width=0.6\textwidth]{figuras/BitoladoFio}
	\label{img:BitoladoFio}
\end{figure}
