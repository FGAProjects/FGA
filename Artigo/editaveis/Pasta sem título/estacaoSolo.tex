Esta seção engloba todas as características dos componentes e funcionamento dos mesmos dentro da Estação Solo. Especificando todo o funcionamento do projeto, desde obtenção, processamento, identificação de risco, armazenamento e o processo de interação com a segurança do Campus.

\subsection{Armazenamento dos Dados}

	Neste projeto será utilizada a câmera Waveshare OV5647 Night Vision. É uma câmera já voltada para sistemas de vigilância, muito utilizada em escritórios e shoppings.

	Serão utilizadas, no total, 15 câmeras do modelo  Waveshare OV5647, e para conseguir armazenar os videos gravados utilizaremos o HD Seagate Archive 8TB. Como o sistema funcionará  24x7, ou seja, 24 horas por 7 dias da semana, o servidor irá passar as imagens caso a situação seja considerada de risco, as imagens irão para o HD a uma taxa de 1024 Kbps, em um mês (considerando um mês como 30 dias), será gasto um total de 4.63TB.

\begin{table}[H]
	\centering
	\caption[Especificações da Seagate Archive 8TB]{Especificações da Seagate$^{\textregistered}$ Archive 8TB~\cite{seagate}}
	\begin{tabular}{|l|l|}
		\hline
		Capacidade                  & 8TB            \\ \hline
		Modelo                      & ST8000AS0002    \\ \hline
		Interface                   & SATA de 6 GB/s \\ \hline
		Velocidade da rotação       & 5900 RPM       \\ \hline
		Cache                       & 128 MB          \\ \hline
		Impacto máximo de operação  & 80 Gs          \\ \hline
		Tipo de armazenamento       & HDD            \\ \hline
		Comprimento                 & 147.00 mm      \\ \hline
		Largura                     & 101.85 mm      \\ \hline
		Altura                      & 26.1 mm        \\ \hline
		Potência média de operação  & 7.500 W        \\ \hline
		Taxa de transferência       & 600 MB/s       \\ \hline
	\end{tabular}
	\label{tab:my-label}
\end{table}

\begin{table}[htp]
\centering
\caption{Configuração de Hardware}
\begin{tabular}{|p{5cm}|p{10cm}|}
\hline
Processador              & Intel Core i7 - 4700K                                                   \\ \hline
Placa-mãe                & ASRock Z87Killer                                                        \\ \hline
Memoria                  & 16 GB G. Skill Spiner (DDR 3 - 1600/PC3 - 12800), configurada a 1600MHz \\ \hline
Placa de vídeo           & GeForce GT 630 1GB                                                      \\ \hline
Resolução de vídeo       & 1920x1080                                                               \\ \hline
Fonte de alimentação     & Corsair CX500M                                                          \\ \hline
Unidade de inicialização & Kingston HyperX 3k 480 GB                                               \\ \hline
\end{tabular}
\label{tab:configHardware}
\end{table}

As Tabelas \ref{tab:my-label} e \ref{tab:configHardware} são mostradas para apresentar as especificações do HD Seagate Archive 8TB e as configurações necessarias de hardware do computador para suportar esse poderoso HD.

Em tempos de muitas chuvas, ocorre uma grande variação de energia, devido às descargas
elétricas de raios. Para que não se tenha o problema de o sistema parar de funcionar por falta de
energia, e pela variação de energia, não chegar a queimar o sistema ou danificar o sistema,
será utilizado um equipamento que armazenar energia por algum tempo.

O equipamento utilizado para o sistema de energia nobreak, será o \textbf{Nobreak Organizador e
Fonte para 16 câmeras}, da tecnologia ONAT. Com este equipamento, o armazenamento de
dados terá em média 4 horas de autonomia, ou seja, caso por algum motivo a luz acabe o
sistema terá em média 4 horas funcionando perfeitamente \cite{nobreak}.

O Seagate para gravação e backup de imagens deverá ser alimentado pelo sistema de energia
(nobreak), de forma a possibilitar a operação em caso de falta de energia elétrica.

\subsection{Interação com a segurança do Campus}

O sistema de comunicação será dado de maneira manual, ou seja, terá uma pessoa na estação de solo que será responsável por analisar os monitores de vigilância que  informam as áreas de possíveis situações de risco mediante a pontuação preestabelecida no sistema. E caso seja necessário, o operador irá alertar um segurança para que ele possa averiguar tal situação. O sistema será uma ferramenta para o operador, auxiliando e facilitando o monitoramento do estacionamento.

Essa comunicação será dada via voz, utilizando um rádio comunicador, ou walk talk. Este meio de comunicação é bem utilizado em sistemas de vigilância de escritórios, shoppings, em construções civis ou em operações de policiais e bombeiros.

Os Walkie Talkies tem um alcance relativamente alto, exatamente o necessário para suprir a carência de sinal de celular presente na área da FGA, e este equipamento terá um alcance de  56km. Neste sistema de monitoramento será utilizado o rádio comunicador walk talk Cobra Cxr925 56km.

A tabela \ref{table:walk} serve para mostrar as especificações do meio de comunicação entre o operador da estação solo e os seguranças que ficarão responsaveis por andar pelo campus.
\begin{table}[H]
\centering
\caption[Especificações e preço do Walk Talk Cobra Cxr925 56km]{Especificações e preço do Walk Talk Cobra Cxr925 56km~\cite{walk}}
\begin{tabular}{|l|l|}
\hline
Peso        & 68g                          \\ \hline
Alcance     & 56 km                        \\ \hline
Dimensões   & 177.50mm x 49.00mm x 33.00mm \\ \hline
Frequência  & 22 canais                    \\ \hline
Alimentação & 110 V                        \\ \hline
\end{tabular}
\label{table:walk}
\end{table}

\subsection{Processamento dos dados}

Para o armazenamento se tornar eficiente e confiável, além de resiliente, será implementado o
padrão RAID(\textbf{Redundant Array of Independent Disks}) em seu nível um também conhecido por
Mirror.

Todas as informações vindas processadas para armazenamento serão copiadas simultaneamente em dois HDs, reduzindo a performance porém mantendo assim a segurança, pois caso haja algum problema técnico em um dos armazenamentos, não haverá nenhuma perda de informação. Neste projeto serão utilizados 4 HD Seagate Archive 8TB para o armazenamento das imagens do monitoramento, formando 2 pares de HDs, que serão redundantes \cite{raid}.

Cada par será capaz de armazenar informações por 30 dias, uma vez cheio as informações passarão a ser armazenadas no par ocioso, formando assim um total de 60 dias de armazenamento de informação, ao final deste período, informações armazenadas serão a ser eliminadas, sendo assim necessário realizar cópias para outros dispositivos caso seja necessário o uso em um período posterior.O tempo em média para recuperação das imagens será de uma a duas horas.

\subsubsection{Tolerância a Falhas}

Um software bem projetado corretamente desde a sua elaboração, não necessita de técnicas de tolerância para software, mesmo que ainda não seja possível garantir na pratica que todo programa estarão corretos \cite{webertolerancia}.

Nesse projeto para que se não tenha problemas em armazenar os dados que forem necessários ou com problemas de ataques externos, será utilizado um dos métodos de tolerância a falhas, e esse método é chamado de \textbf{Diversidade (ou programação de n-versões)}.

\begin{itemize}
	\item Diversidade (ou programação n-versões)
\end{itemize}
\par


Diversidade, também chamada programação diversitária, é uma técnica de redundância usada para obter tolerância a falhas em software. A partir de um problema a ser solucionado são implementadas diversas soluções alternativas, sendo a resposta do sistema determinada por votação.

Esse método consiste em projetar o software N vezes, sendo cada versão independente das demais, e caso o sistema perceba perceba algo de diferente no tempo de execução, será realizado uma votação para solução dos problemas. A votação é feita em tempo de execução e é analisadas todas as versões do software,sendo essas N versões feita por equipes de programadores diferentes, o que faz com que o erro seja manisfestado de maneira diferente em cada versão. A votação pegara cada erro e irá escolher o erro especificado que garanta mais integridade ao sistema e menos perca ou até perca total de dados do sistema.

\textbf{Vantagens}
\par
A vantagem é usar esse método, é de manter uma maior confiabilidade no sistema e na segurança de seus dados.

\textbf{Desvantagens}
\par
Desenvolvedores costumam ter as mesmas práticas de programação, não garantindo a independência das versões.Outra desvantagem é o custo, pois tem se o custo com as equipes de desenvolvimento do software e manutenção do mesmo, a complexidade de sincronização das versões.

\textbf{Situação Hipotética}
\par
Quatro equipes de desenvolvedores são escolhidos para projetar uma versão do software. Utilizando a situação ideal em que todas as equipes não tenham as mesmas práticas de desenvolvimento, são criadas as quatro versões são criadas. Em um momento o sistema detecta uma atividade diferente da que foi projetado e é iniciada a verificação e votação das quatro versões do sistema.A votação diagnosticou que o erro se manisfestou de maneira em que seria menos recomendado usar a versão quatro, mas na versão dois o erro é mais fácil de ser corrigido com menos ou talvez sem dano algum ao sistema. Enquanto o erro na versão quatro é corrigido para que seja mais efetivo caso ocorra outro problema, as demais versões estarão funcionando perfeitamente.
\\
\\

\subsection{Operacionabilidade do Sistema}

O diagrama de Arquitetura do sistema pode ser observado na imagem \ref{img:Operacionabilidade do Sistema}.

\begin{figure}[]
	\centering
	\caption{Diagrama de Arquitetura: Operacionalibilidade do Sistema}
	\includegraphics[width=1\textwidth]{figuras/OperacionabilidadedoSistema}
	\label{img:Operacionabilidade do Sistema}
\end{figure}

O sistema funcionará da seguinte maneira: no balão ocorrerá o processamento e tratamento das imagens recebidas (Processamento das câmeras). Após serem tratadas, estas imagens serão armazenadas em um HD da estação de solo(Gerenciamento de dados).

Em outro processo paralelo, estas imagens  irão fornecer os dados para que o sistema execute suas funcionalidades, como o gerenciamento das imagens, a identificação e pontuação das situações de risco (Processamento). E a saída do sistema será por meio de uma interface que interagirá com o operador da estação de solo.

\subsection{Processo de Monitoramento}
O processo de monitoramento do Sistema Unificado de Monitoramento (SUM), relativo ao modus operandi dos agentes do sistema, é ilustrado pelo diagrama \ref{img:Processo de Monitoramento}.

\begin{figure}[]
\centering
\caption{Processo de Monitoramento}
\includegraphics[width=1.0\textwidth]{figuras/Processodemonitorament}
\label{img:Processo de Monitoramento}
\end{figure}
As atividades contempladas neste processo estão descritas abaixo:
\\

\textbf{Identificar comportamento ou atividade suspeita} - O processo é iniciado a partir da identificação de comportamento ou atividade suspeita por parte do balão de monitoramento, que leva em consideração fatores de risco, tabela \ref{tab:criteriosRisco}, que determinam se uma atividade é normal, duvidosa ou suspeita.
\\

\textbf{Emitir alerta à central de Monitoramento} - Nesta atividade o sistema do balão, após ter identificado uma atividade suspeita nas dependências do estacionamento, emiti um sinal de alerta à Central de monitoramento contatando o operador do sistema.
\\

\textbf{Avaliar possível atividade criminosa} - Nesta atividade o operador do sistema, após ter sido notificado pelo balão sobre uma atividade suspeita, acompanhará através das câmeras de vídeo do balão, em tempo real, a ação do suspeito, avaliando se esta é uma atividade criminosa.
\\

\textbf{Desativar alerta do balão de monitoramento} - Nesta atividade o operador do sistema, após ter recebido e avaliado o alerta de atividade suspeita emitido pelo balão e concluído que este não retrata uma atividade criminosa, desativará o alerta do balão para aquela ação em específico, encerrando o processo.
\\

\textbf{Alertar a equipe de Segurança} - Nesta atividade o operador do sistema, após ter concluído que o alerta do balão se trata de fato de uma atividade criminosa, irá contactar a equipe de Segurança em campo emitindo um alerta via rádio.
\\

\textbf{Averiguar a situação do local} - Nesta atividade a equipe de Segurança, após receber um alerta do operador do sistema sobre uma atividade criminosa, irá averiguar a situação , no local informado, para avaliar a possibilidade de intervenção e/ou impedimento da ação criminosa.
\\

\textbf{Tomar medidas cabíveis} - Nesta atividade a equipe de Segurança, após concluir que há possibilidade de intervenção e/ou impedimento da ação criminosa, tomará as medidas cabíveis para que o infrator seja detido, encerrando o processo.
\\

\textbf{Contactar as autoridades locais} - Nesta atividade a equipe de Segurança, após concluir que não há possibilidade de intervenção e/ou impedimento da ação criminosa por quaisquer razões, irá contactar as Autoridades locais responsáveis para que estes tomem as medidas cabíveis à situação.
\\
\par
\textbf{Requisitos do Sistema de Monitoramento}

A rastreabilidade dos requisitos do sistema pode ser observada na tabela \ref{tab:Requisitos do Sistema de Monitoramento}.

\begin{table}[H]
\centering
\caption{Requisitos do Sistema de Monitoramento}
\begin{tabular}{|l|l|}
\hline
\textbf{Sigla}        & \textbf{Descricao} \\ \hline
NEC     & Necessidade                      \\ \hline
CAR   	& Característica				   \\ \hline
UC  	& Casos de uso                     \\ \hline
RNF 	& Requisito Não-Funcional          \\ \hline

\end{tabular}
\label{tab:Requisitos do Sistema de Monitoramento}
\end{table}

\textbf{Necessidades}
\par
A partir do processo de monitoramento descrito no tópico acima abstraiu-se as necessidades dos agentes do sistema.

As necessidades são:
\begin{itemize}
	\item NEC01 - O operador do sistema precisa receber um alerta do sistema de monitoramento quando houver uma atividade suspeita ocorrendo no estacionamento.
	\item NEC02 - O operador do sistema precisa visualizar em tempo real a ação do suspeito através das câmeras do balão de monitoramento.
	\item NEC03 - O operador do sistema precisa conseguir aproximar a imagem de forma a ser possível enxergar detalhes da ação do suspeito.
	\item NEC04 - O operador do sistema precisa visualizar áreas específicas do estacionamento para acompanhar o que ocorre em cada setor.
\end{itemize}

\textbf{Características}
\par
As características deviradas a partir das necessidades dos agentes do sistema são:
\begin{itemize}
	\item CAR01 - O sistema irá prover alertas de atividade suspeita na interface do usuário.
	\item CAR02 - O sistema deve permitir que um alerta de atividade suspeita seja desativado.
	\item CAR03 - O sistema deve permitir o acesso às câmeras de vídeo do balão de monitoramento.
\end{itemize}

\subsection{Diagrama de Caso de Uso}

O diagrama de casos de uso do sistema apresenta em modo gráfico as funcionalidades e os Atores do sistema, o mesmo pode ser observado na imagem \ref{img:Casos de Uso}.
\begin{figure}[H]
	\centering
	\caption{Diagrama de Caso de Uso}
	\includegraphics[width=1\textwidth]{figuras/casodeUso}
	\label{img:Casos de Uso}
\end{figure}

\begin{itemize}
	\item UC01 - Captar e processar imagens
		\begin{enumerate}
			\item Descrição

				Este caso de uso se refere ao balão capturar e enviar as imagens para a estação de solo em um formato processável.

			\item Pré-condições

				Não possui pré condições.

			\item Pós-condições
				\begin{itemize}
					\item A estação de solo recebe as imagens em formato armazenável.
				\end{itemize}

			\item Atores
				\begin{itemize}
					\item Balão.
				\end{itemize}


			\item Fluxo básico
				\begin{enumerate}
					\item O balão captura as imagens;
					\item O balão processa as imagens transformando em formato legível e armazenável;
					\item O balão envia as imagens à estação de solo.
					\label{item:balaoEnvia}
				\end{enumerate}

			\item Fluxo alternativo A
				\begin{enumerate}
					\item No passo \ref{item:balaoEnvia};
					\item O balão não consegue enviar as imagens;
					\item O fluxo volta para o passo \ref{item:balaoEnvia}.
				\end{enumerate}
		\end{enumerate}

		\item UC02 - Gerenciamento de imagens
			\begin{enumerate}
				\item Descrição

					Este caso de uso se refere ao sistema fazer processamentos para encontrar atividades de risco.

				\item Pré-condições
					\begin{itemize}
						\item O balão enviou as imagens.
					\end{itemize}

				\item Pós-condições
					\begin{itemize}
						\item Imagens processadas e encontradas possíveis atividades de risco.
					\end{itemize}

				\item Atores
					\begin{itemize}
						\item estação de solo.
					\end{itemize}


				\item Fluxo básico
					\begin{enumerate}
						\item A estação de solo recebe as imagens do balão;
						\item O sistema encontra e conta todas as atividades de risco na imagem;
						\item O sistema marca todas as atividades;
						\item Fim do fluxo.

					\end{enumerate}
			\end{enumerate}

			\item UC03 - Informar grau de risco da atividade.
				\begin{enumerate}
					\item Descrição

						Este caso de uso se refere a pontuar cada atividade de risco encontrada pelo caso de uso UC02.

					\item Pré-condições
						\begin{itemize}
							\item As atividades suspeitas já foram encontradas.
						\end{itemize}

					\item Pós-condições
						\begin{itemize}
							\item As atividades suspeitas estão pontuadas.
						\end{itemize}

					\item Atores
						\begin{itemize}
							\item estação de solo.
						\end{itemize}


					\item Fluxo básico
						\begin{enumerate}
							\item A estação de solo recebe as atividades suspeitas;
							\item O sistema identifica em qual das condutas de risco a atividade se encontra;
							\item O sistema cria uma nota para a atividade;
							\item Fim do fluxo.

						\end{enumerate}
				\end{enumerate}

		\item UC04 - Identificar situação suspeita.
			\begin{enumerate}
				\item Descrição

					Este caso de uso especifica a ação do sistema em disparar alertas de possíveis roubos para o operador do sistema e selecionar as cores para emoldurar as imagens de acordo com seu ranking.

				\item Pré-condições
					\begin{itemize}
						\item O operador do sistema tem que estar logado no sistema.
					\end{itemize}

				\item Pós-condições
					\begin{itemize}
						\item O operador tem informações o suficiente para tomar a decisão de caracterizar um roubo ou não.
					\end{itemize}

				\item Atores
					\begin{itemize}
						\item Operador do sistema;
						\item estação de solo.
					\end{itemize}


				\item Fluxo básico
					\begin{enumerate}
						\item O sistema detecta uma atividade suspeita;
						\item O sistema caractegoriza a imagem de acordo com o \textit{ranking};
						\label{item:imagemAcordoRanking}
						\item O sistema ordena as imagens de acordo com sua probabilidade definida no \textit{ranking};
						\item O sistema apresenta as imagens com a moldura da cor específica para o ranking de acordo com a tabela \ref{tab:coresRisco};

						\begin{table}[]
							\centering
							\caption{Cores das molduras das imagens}
								\begin{tabular}{|c|c|}
								\hline
									Risco & Cor \\ \hline
									0 & {\color[HTML]{000000} Cor da moldura} \\ \hline
									1 & {\color[HTML]{9ACD32} Cor da moldura} \\ \hline
									2 & {\color[HTML]{FFFACD} Cor da moldura} \\ \hline
									3 & {\color[HTML]{EEE8AA} Cor da moldura} \\ \hline
									4 & {\color[HTML]{BDB76B} Cor da moldura} \\ \hline
									5 & {\color[HTML]{DAA520} Cor da moldura} \\ \hline
									6 & {\color[HTML]{FFA07A} Cor da moldura} \\ \hline
									7 & {\color[HTML]{A0522D} Cor da moldura} \\ \hline
									8 & {\color[HTML]{CD5C5C} Cor da moldura} \\ \hline
									9 & {\color[HTML]{CD3333} Cor da moldura} \\ \hline
									10 & {\color[HTML]{B22222} Cor da moldura} \\ \hline
									11 & {\color[HTML]{EE3B3B} Cor da moldura} \\ \hline
									12 & {\color[HTML]{FF3030} Cor da moldura} \\ \hline
									13 & {\color[HTML]{FF4500} Cor da moldura} \\ \hline
									14 & {\color[HTML]{FF0000} Cor da moldura} \\ \hline
								\end{tabular}
								\label{tab:coresRisco}
						\end{table}

						\item O sistema mostra o maior ranking em primeiro lugar.
						\label{item:maiorRanking}
					\end{enumerate}

				\item Fluxo alternativo A
					\begin{enumerate}
						\item O sistema está no passo \ref{item:imagemAcordoRanking};
						\item O sistema encontra um erro ao avaliar os pontos para fazer o ranking;
						\item O sistema da a imagem o ranking 5 e coloca nela o aviso ``Não foi possível analisar essa imagem'';
						\item O sistema retorna ao passo \ref{item:maiorRanking} do fluxo principal.
					\end{enumerate}
			\end{enumerate}

				\item UC05 - Observar áreas que merecem mais atenção
					\begin{enumerate}
						\item Descrição

							Este caso de uso se refere ao ato do operador do sistema poder escolher qual imagem quer ver no momento
						\item Pré-condições
							\begin{itemize}
								\item As imagens estão pontuadas de acordo com sua criticidade.
							\end{itemize}

						\item Pós-condições

						O operador pode acionar a segurança.

						\item Atores
							\begin{itemize}
								\item estação de solo;
								\item Operador do sistema.
							\end{itemize}


						\item Fluxo básico
							\begin{enumerate}
								\item O sistema apresenta ao operador diversas imagens;
								\item O operador do sistema seleciona uma imagem;
								\item A imagem fica maior e o operador tem controle da camêra que está gerando aquela imagem;
								\item Fim do fluxo.
							\end{enumerate}
					\end{enumerate}
		\end{itemize}


\subsection{Identificação de Risco} % (fold)
\label{sub:identifica_o_de_risco}


	O sistema SUM, como sabemos, será operado por um operador que terá como responsabilidade observar possíveis casos de roubos a carros. A decisão final sobre a possibilidade de ser um roubo real ou não, cabe ao operador, que terá apoio do sistema para chegar a conclusão final.

	Como o estacionamento da Universidade de Brasília - Campus Gama recebe um número muito grande de carros, é impossível responsabilizar apenas um operador para observar todos os carros ao mesmo tempo, verificando as possibilidades de possíveis roubos ocorrendo, inclusive, em paralelo.

	Para solucionar este problema, o sistema SUM apoiará o operador na escolha de casos suspeitos a serem observados. Ou seja, o sistema apresentará ao operador todos os casos de possíveis roubos ocorrendo no momento, especificando os casos mais importantes e menos importantes.

	A ordenação das imagens mais importantes será feita a partir da utilização de cores na tonalidade avermelhada. As imagens mais críticas serão pintadas de tons avermelhados. Quanto mais crítico, mais próximo do vermelho.

	Utilizando o sistema, o operador saberá exatamente quais imagens merecem atenção e até quais imagens merecem mais atenção que outras imagens, dependendo da quantificação do risco, que é feita pelo sistema. Esta quantificação é feita a partir da observação de critérios que identifiquem um possível caso de roubo a carro.

\subsubsection{Quantificação do Risco}

	Com o objetivo de selecionar as imagens mais importantes a serem analisadas pelo operador, o sistema SUM deverá realizar uma quantificação de critérios que levem a definição de um possível caso de roubo a carro. Estes critérios foram obtidos após a análise de inúmeras imagens que registraram casos de roubo a carros em estacionamentos universitários.

	Os critérios possuem pesos para quantificação, dependendo do quão crítico é o critério analisado. A ponderação dos critérios pode ser observada na tabela \ref{tab:criteriosRisco}:

	\begin{table}[H]
		\centering
		\caption{Identificação dos Critérios de Risco}
		\begin{tabular}{|c|c|c|}
			\hline
			\rowcolor[HTML]{C0C0C0}
			{\color[HTML]{00009B} \textbf{Critérios}}                                   & {\color[HTML]{00009B} \textbf{Descrição}}                                                                                                    & {\color[HTML]{00009B} \textbf{Peso}} \\ \hline
			Proximidade                                                                 & \begin{tabular}[c]{@{}c@{}}Distância de 2m, ou menos, \\ entre um suspeito e o carro \\ analisado.\end{tabular}                              & 1                                    \\ \hline
			\begin{tabular}[c]{@{}c@{}}Permanência próximo \\ ao carro.\end{tabular}    & \begin{tabular}[c]{@{}c@{}}Tempo em que o suspeito \\ permanece ao lado do carro \\ analisado ultrapassa os 30 segundos.\end{tabular}        & 2                                    \\ \hline
			\begin{tabular}[c]{@{}c@{}}Contato físico com a \\ porta.\end{tabular}      & \begin{tabular}[c]{@{}c@{}}O suspeito mantem contato físico com \\ a porta por mais de 10 segundos.\end{tabular}                             & 3                                    \\ \hline
			\begin{tabular}[c]{@{}c@{}}Contato físico com o \\ Porta-Malas\end{tabular} & \begin{tabular}[c]{@{}c@{}}O suspeito mantem contato físico com o \\ porta-malas do carro analisado por mais de \\ 20 segundos.\end{tabular} & 3                                    \\ \hline
			Alarme                                                                      & O alarme do carro analisado está disparando.                                                                                                 & 5                                    \\ \hline
		\end{tabular}
		\label{tab:criteriosRisco}
\end{table}

	O nível de criticidade da imagem analisada será definido a partir do somatório dos critérios observados na imagem.

	Em momento algum o sistema chegará a conclusão de que é um roubo em execução ou não, ele apenas apontará imagens que se enquadram em um possível caso de roubo a carros. A identificação das imagens mais importantes será feita a partir da geração de um Ranking de possíveis casos. Este Ranking será gerado a partir da somatória dos critérios identificados em cada caso.

	O Ranking de imagens será apresentado ao operador na forma de um “mosaico” de imagens, que receberão tons de amarelo a vermelho, dependendo de sua importância no momento. O operador poderá selecionar a imagem para poder controlar a câmera e visualizar a imagem da forma que desejar, verificando se o caso se refere a um caso de roubo ou apenas um engano.

	Para captação destes critérios, o sistema deverá possuir sensores de calor e proximidade, alem das imagens obtidas pelas câmeras.


	\subsection{Viabilidade Econômica}

	Para analisar a viabilidade econômica da Estação Solo, é necessária a identificação e quantificação de todos os gastos que deverão ser realizados para desenvolvimento, manutenção e operação do sistema desenvolvido. Os gastos identificados estão distribuidos em Desenvovimento do Sistema de Software, Infraestrutura e Operação do Sistema. Para a análise a identificação dos gastos relacionados ao Desenvolvimento e Manutenção do sistema, foi utilizada a técnica de Análise de Pontos de Casos de Uso.


	\subsubsection{Análise por Pontos de Casos de Uso}

			A Análise por Pontos de Casos de Uso é um método de Estimativa Funcional de Software que usa os Casos de Uso, criados por Gustav Karner \cite{karner1993resource} em 1993. Esta análise tem como objetivo adaptar a Estimativa por Pontos de Função (APF). Foi criado com o objetivo de viabilizar a estimativa do tamanho funcional do sistema ainda na fase de levantamento dos Casos de Uso, utilizando os próprios documentos gerados nesta fase como \textit{input} para o cálculo da estimativa, como apresenta Herval Freire \cite{freire2003calculando}.

			Segundo \cite{pontosUC}, o processo de contagem utilizando o método de Análise por Pontos de Casos de Uso, que engloba 7 atividades que estão dispostas a seguir:

		\begin{itemize}
			\item Relacionar os Atores do Sistema, classificando-os de acordo com sua complexidade.
			\item Contar os Casos de Uso e atribuir o grau de complexidade de cada um.
			\item Somar o resultado das duas atividades anteriores.
			\item Determinar o fator de complexidade técnica.
			\item Determinar o fator de complexidade ambiental.
			\item Calcular os Pontos de Casos de Uso ajustados.
			\item Calcular a estimativa de horas necessárias para conclusão do sistema.
		\end{itemize}

		Karner \cite{karner1993resource} sugere a utilização de 20 (vinte) homens/hora para cada Ponto de Caso de Uso identificado. Dessa forma, é fácil obter a quantidade de horas necessárias para conclusão do sistema, assim como, o valor aproximado para desenvolvimento do sistema. Para minimizar a mergem de erro, serão contabilizados os Fators Técnicos e Ambientais, que colaboram para a estimativa da complexidade de desenvolvimento do software, aumentando ou não o tempo/custo do projeto.

		Primeiramente deve-se identificar e pontuar os Atores do sistema, como pode ser observado na tabela \ref{tab:atoresSistema}.

		\begin{table}[H]
			\centering
			\caption{Identificação dos Atores do Sistema}
			\label{tab:atoresSistema}
			\begin{tabular}{|c|c|c|c|}
				\hline
				\textbf{Ator} & \textbf{Descrição}                                                                                                                                                                             & \textbf{Complexidade} & \textbf{Peso}                                           \\ \hline
				Balão         & \begin{tabular}[c]{@{}c@{}}Ator responsável por obter todos os dados \\ utilizados para identificação de situações de risco.\end{tabular}                                                      & Simples               & 1                                                       \\ \hline
				Estação Solo  & \begin{tabular}[c]{@{}c@{}}Ator responsável por processar e apresentar os dados \\ obtidos pelo Balão ao Operador. Além de identificar e \\ apontar possíveis situações de risco.\end{tabular} & Simples               & 1                                                       \\ \hline
				Operador      & \begin{tabular}[c]{@{}c@{}}Ator responsável por analisar as situações de risco, \\ verificando a criticidade das mesmas.\end{tabular}                                                          & Complexa              & 3                                                       \\ \hline
				\rowcolor[HTML]{C0C0C0}
				\multicolumn{3}{|c|}{\cellcolor[HTML]{C0C0C0}\textbf{TOTAL}}                                                                                                                                                                           & \multicolumn{1}{l|}{\cellcolor[HTML]{C0C0C0}\textbf{  5}} \\ \hline
			\end{tabular}
		\end{table}

		Após a análise dos Atores, deve-se contar os Casos de Uso do sistema, listando-os e quantificando o número de transições do Caso de Uso e, com isso, obtendo a complexidade do mesmo, como pode ser observado na tabela \ref{tab:idUC}.

		\begin{table}[H]
			\centering
			\caption{Identificação dos Casos de Uso do Sistema}
			\label{tab:idUC}
			\begin{tabular}{|c|c|c|c|c|}
				\hline
				\textbf{Caso de Uso} & \textbf{Descrição}                                                                  & \textbf{\begin{tabular}[c]{@{}c@{}}Nº. de \\ Transações\end{tabular}} & \textbf{Complexidade} & \textbf{Peso} \\ \hline
				UC01                 & \begin{tabular}[c]{@{}c@{}}Captar e processar \\ imagens.\end{tabular}              & 1                                                                     & Simples               & 5             \\ \hline
				UC02                 & \begin{tabular}[c]{@{}c@{}}Gerenciamento de \\ Imagens.\end{tabular}                & 1                                                                     & Simples               & 5             \\ \hline
				UC03                 & \begin{tabular}[c]{@{}c@{}}Informar grau de risco \\ da atividade.\end{tabular}     & 1                                                                     & Simples               & 5             \\ \hline
				UC04                 & \begin{tabular}[c]{@{}c@{}}Identificar situação \\ suspeita.\end{tabular}           & 2                                                                     & Simples               & 5             \\ \hline
				UC05                 & \begin{tabular}[c]{@{}c@{}}Observar áreas que merecem \\ mais atenção.\end{tabular} & 1                                                                     & Simples               & 5             \\ \hline
				\rowcolor[HTML]{C0C0C0}
				\multicolumn{4}{|c|}{\cellcolor[HTML]{C0C0C0}\textbf{TOTAL}}                                                                                                                                               & 25            \\ \hline
			\end{tabular}
		\end{table}

		Com a obtenção das complexidades dos Atores e dos Casos de Uso do sistema, devemos obter o UUCP (Pontos de Casos de Uso) não ajustados, seguindo a seguinte fórmula:

	\begin{center}

		\textbf{UUCP}= UAW + UUCW

		\textbf{UUCP}= 5 + 25

		\textbf{UUCP}= 30
	\end{center}

	Após a obtenção do \textbf{UUCP} (Pontos de Casos de Uso Não ajustados), devemos analisar a complexidade técnica do sistema, que está disposta na tabela \ref{tab:compTecnica}.

	\begin{table}[H]
		\centering
		\caption{Análise da Complexidade Técnica}
		\label{tab:compTecnica}
		\begin{tabular}{|c|c|c|c|c|}
			\hline
			\textbf{Fator} & \textbf{Descrição}                           & \textbf{Peso} & \textbf{Valor} & \textbf{Total} \\ \hline
			T1             & Sistemas Distribuídos.                       & 2             & 4              & \textbf{8}     \\ \hline
			T2             & Tempo de resposta/desempenho.                & 1             & 3              & \textbf{3}     \\ \hline
			T3             & Eficiência do usuário final (on-line)        & 1             & 4              & \textbf{4}     \\ \hline
			T4             & Processamento interno complexo.              & 1             & 4              & \textbf{4}     \\ \hline
			T5             & Reusabilidade do código em outras aplicações & 1             & 0              & \textbf{0}     \\ \hline
			T6             & Facilidade de instalação                     & 0,5           & 3              & \textbf{1,5}   \\ \hline
			T7             & Usabilidade (facilidade operacional)         & 0,5           & 4              & \textbf{2}     \\ \hline
			T8             & Portabilidade                                & 2             & 0              & \textbf{0}     \\ \hline
			T9             & Facilidade de manutenção                     & 1             & 3              & \textbf{3}     \\ \hline
			T10            & Acessos simultâneos (concorrência)           & 1             & 3              & \textbf{3}     \\ \hline
			T11            & Aspectos especiais de segurança              & 1             & 3              & \textbf{3}     \\ \hline
			T12            & Acesso direto para terceiros                 & 1             & 0              & \textbf{0}     \\ \hline
			T13            & Facilidades especiais de treinamento         & 1             & 0              & \textbf{0}     \\ \hline
			\multicolumn{4}{|c|}{\textbf{TOTAL}}                                                           & \textbf{31,5}  \\ \hline
		\end{tabular}
	\end{table}

	A partir da observação da tabela \ref{tab:compTecnica}, basta calcularmos o \textbf{TCF} (Fator de Complexidade Técnica) utilizando a segiinte fórmula:

	\begin{center}
		\textbf{TCF} = 0,6 + ( 0,01 * \textit{TFactor} )

		\textbf{TCF} = 0,6 + (0,01 * 31,5)

		\textbf{TCF = 0,915}
	\end{center}

	Para finalizar, basta apenas analisar os Fatores Ambientais que influenciam no desenvolvimento do sistema, esta análise está disposta na tabela \ref{tab:fatAmbiental}.

	\begin{table}[H]
		\centering
		\caption{Fatores Ambientais}
		\label{tab:fatAmbiental}
		\begin{tabular}{|c|c|c|c|c|}
			\hline
			\textbf{Fator} & \textbf{Descrição}                                                                                           & \textbf{Peso} & \textbf{Valor} & \textbf{Total} \\ \hline
			T1             & \begin{tabular}[c]{@{}c@{}}Familiaridade com o processo de \\ desenvolvimento de software.\end{tabular}      & 1,5           & 4              & \textbf{6}     \\ \hline
			T2             & Experiência na aplicação.                                                                                    & 0,5           & 0              & \textbf{0}     \\ \hline
			T3             & \begin{tabular}[c]{@{}c@{}}Experiência com OO, na linguagem e \\ na técnica de desenvolvimento.\end{tabular} & 1             & 3              & \textbf{3}     \\ \hline
			T4             & Capacidade do Líder de Projeto.                                                                              & 0,5           & 2              & \textbf{1}     \\ \hline
			T5             & Motivação                                                                                                    & 1             & 4              & \textbf{4}     \\ \hline
			T6             & Requisitos estáveis.                                                                                         & 2             & 3              & \textbf{6}     \\ \hline
			T7             & Trabalhadores com dedicação parcial.                                                                         & -1            & 5              & \textbf{-5}    \\ \hline
			T8             & \begin{tabular}[c]{@{}c@{}}Dificuldade da Linguagem \\ de Programação\end{tabular}                           & -1            & 5              & \textbf{-5}    \\ \hline
			\multicolumn{4}{|c|}{\textbf{TOTAL}}                                                                                                                           & \textbf{10}    \\ \hline
		\end{tabular}
	\end{table}

	Utilizando as tabelas \ref{tab:compTecnica} e \ref{tab:fatAmbiental} , chegamos ao resultado do \textit{EFactor} total:
	\begin{center}
		\textbf{EF} = 1,4 + (-0,03 x \textit{EFactor} )

		\textbf{EF} = 1,4 + (-0,03 x 10)

		\textbf{EF} = 1,1
	\end{center}

	O valor final de tudo, o \textbf{UCP} é calculado de acordo com a fórmula:
	\begin{center}
		\textbf{UCP} = UUCP * TCF * EF

		\textbf{UCP} = 30 * 0,915 * 31

		\textbf{UCP} = 31
	\end{center}


	Ou seja, o sistema que será desenvolvido possui 31 Pontos de Casos de Uso Ajustados. Como \cite{karner1993resource} sugere que cada Ponto de Caso de Uso necessita de 20 homens/hora para a conclusão do mesmo, para concluir todo o sistema serão necessários 604 homens/hora.

	\subsubsection{Infraestrutura}

		\begin{table}[H]
\centering
\caption{Tabela de Preços}
\begin{tabular}{|l|l|l|l|}
\hline
                      & Quantidade/Unidades & Preço de mês/Compra & Total    \\ \hline
Waveshare OV5647 Night Vision   & 15       & R\$ 123,34         & R\$ 1850,10 \\ \hline
Seagate Archive 8TB & 4        & R\$ 984,90           & R\$ 3939,60   \\ \hline
Sistema Nobreak       & 1        & R\$ 396,90           & R\$ 396,90      \\ \hline
Hardware              & 1        & R\$ 5271,86         & R\$ 5271,86    \\ \hline
Walk talk Cobra Cxr925 & 5 & R\$ 415,99 & R\$ 2079,95 \\
\hline
Engenheiro de Software & 4 & R\$ 3000,00 & R\$ 12.000,00 \\
\hline
Operador & 4 & R\$ 2000,00 & R\$ 8000,00 \\
\hline
Container & 1 & R\$ 6000,00 & R\$ 6000,00 \\
\hline
\multicolumn{4}{|l|}{Total Gasto com Empregadores}{R\$ 11000,00}                                     \\ \hline
\multicolumn{4}{|l|}{Total Gasto com Equipamentos}{R\$ 19511,41}                                     \\ \hline
\multicolumn{4}{|l|}{Total Gasto}{R\$ 30511,41}                                     \\ \hline
\end{tabular}
\label{tab:precosComponentes}
\end{table}
A tabela \ref{tab:precosComponentes} demonstra os custos do HD, sistema noBreak, walk talk e hardware necessários para suportar esse poderoso HD. Como explicado anteriormente o sistema nobreak é utilizado caso ocorra algum problema que possa ser suspenso o uso de energia, como chuvas ou problemas internos/externos. O HD será utilizado para armazenar as imagens de ações suspeitas caso seja detectada alguma ação da mesma, as configurações de hardware para suportar tal HD e o sistema de comunicação entre o operador na estação de solo e os seguranças que patrularam a area do campus. Além de conter os gastos totais com os engenheiros de software responsaveis por projetar o sistema, o operador das câmeras e o container onde ficara a estação solo.