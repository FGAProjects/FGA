Para a melhor visualização do funcionamento integrado do SUM, apresenta-se uma situação hipotética de atividade suspeita na (Figura \ref{img:integracao}). A figura descreve a sequência de operação do SUM, desde o reconhecimento da atividade suspeita até o acionamento da segurança.

\begin{figure}[H]
	\centering
	\caption{Ilustração de situação de risco e funcionamento do SUM.}
	\includegraphics[width=0.8\textwidth]{figuras/integracao}
	\label{img:integracao}
\end{figure}

A imagem está enumerada de acordo a seguinte sequência de cenas:

\begin{enumerate}
	\item Um indivíduo estaciona próximo ao prédio e, juntamente com seu filho, caminha em direção à entrada deste. Ao chegarem perto da entrada, uma pessoa começa a se aproximar de seu carro. Ela espera até que entrem no prédio e então permanece ali durante o período de 5 minutos.

	\item As câmeras, contidas nos balões próximos ao carro, captarão estas imagens que, em seguida, serão passadas por um processamento de imagens, executados pelo sistema eletrônico presente no \textit{Payload} (suspenso pelo balão). Nesta fase de processamento de sinais, as câmeras direcionarão suas imagens para o Raspberry PI (microcontrolador responsável pelo processamento de imagens), através do seu conector específico para câmera.
	Este conterá um algoritmo que realizará a compressão dos vídeos enviados, no padrão H264, um padrão de codificação de vídeo de última geração que utiliza um quadro de referência para comparação e codifica apenas os pixels que foram modificados. Após realizar esse processo de compressão, o Raspberry, que estará conectado ao Arduino, receberá os dados dos sensores e a interpretação feita pelo algoritmo implementado no microcontrolador. Caso a interpretação do algoritmo informe determinada anomalia no sistema e exija que esse seja estabilizado, o Raspberry comunicará ao Galileo, via comunicação serial, que realize a estabilização do sistema.

	\item Com as imagens e os dados dos sensores obtidos, o Raspberry transmitirá essas informações em tempo real para o balão mais próximo da central, a comunicação entre eles será sem fio, montando uma rede intranet.

	\item O balão, que estará recebendo essas informações, transmitirá para a central (container) através de um cabo de \textit{Ethernet} e o computador que receber tais informações realizará todo o procedimento necessário com as imagens.
	
	\item Com estas imagens e os dados dos sensores transmitidos, será realizado o reconhecimento da situação em questão e será verificado o funcionamento do sistema. A estação de solo possuirá um conjunto de monitores e um monitor central. Após a transmissão das informações, serão emitidos alertas no monitor central, onde o operador irá averiguar a situação e, caso seja necessário, informar a segurança do Campus via rádio. O segurança do Campus irá até o local e verificará a situação, tendo como função informar às autoridades responsáveis pelo Campus.
\end{enumerate}
