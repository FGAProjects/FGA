Este trabalho buscou averiguar se é possível a simulação de atividades humanas por meio da tecnologia de Inteligência Artificial. A conscientização de tecnologias que tentam aproximar decisões humanas a máquinas já são empregadas em nosso cotidiano, como carros com navegação personalizada, sistemas embutidos de inteligência artificial em \textit{smartphones}, como no caso da \textit{SIRI} e \textit{Android} para as empresas internacionais \text{Apple} e \textit{Google} respectivamente.

Para a confecção deste trabalho foi aplicada a técnica de Revisão Sistemática de Literatura (RSL) para direcionar um método de pesquisa e assim organizar as publicações conforme o tema, em bases de dados selecionadas, sendo feita a partir de palavras-chave e \textit{string} de busca.

As publicações foram escolhidas a partir dos critérios de inclusão e o conjunto de publicações que não foram selecionadas, foram excluídas a partir do critério de exclusão. As publicações foram categorizadas em duas categorias: Tomada de Decisões e Inteligência Artificial. A maior parte dos artigos selecionados, foram publicados nos últimos 20 anos. Com a realização da pesquisa deste trabalho, foi possível alcançar resultados favoráveis aos objetivos desejados. Foi constatado que existem estudos na área com potenciais para o desenvolvimento de sistemas artificiais que interajam de forma mais humana. Esses sistemas poderão simular sentimentos e ações humanas, e se adaptar conforme o conjunto social em que esteja envolvido.

Após a análise dos dados deste trabalho, é verificado que existe a possibilidade de desenvolvimento para novos trabalhos, envolvendo o desenvolvimento da Inteligência Artificial e tecnologias emergentes. Para um melhor desenvolvimento dessa área, pesquisadores de IA poderiam estudar mais como funciona o cérebro humano e a partir disso aplicar esses estudos na tecnologia, o que poderia levar a resultados mais próximos do desejado.