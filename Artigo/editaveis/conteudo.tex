
\newpage
\pdfbookmark[0]{\contentsname}{toc}
\tableofcontents*
\cleardoublepage
\newpage


\hrule	% Linha horizontal
\begin{center} % Minipage Centralizado
\begin{minipage}[c]{12.5cm} % Largura
Autor: \imprimirautor

Tema: \imprimirtitulo 
\imprimirlocal, 
\imprimirorientador\\
\hspace{0.5cm}
\parbox[t]{\textwidth}{\imprimirinstituicao
\imprimirdata.}\\
\hspace{0.5cm}
\end{minipage}
\end{center}
\hrule

\vspace{1cm}

\SingleSpacing
\noindent
{\textbf{LEARNING ARTIFICIAL INTELLIGENCE THROUGH ARTIFICIAL MEMORIES}}
\indent
\small

Memories and attitudes contribute the conscious and the human character from diverse happenings, be they good or not. However, both contribute to learning. In the case of memories, the conscious directs the attitudes, demanding less brain processing. Artificial intelligence uses this same principle of events for its development, that is, the use of artificial memories theoretically could simulate human attitudes.

\noindent
 
\textbf{Keywords}: Intelligence,Artificial,Memories,Humans,Technology,Modern,World,
\\Consciousness,Through,Brain.

\SingleSpacing
\noindent
{\textbf{APRENDIZADO DE INTELIGÊNCIA ARTIFICIAL ATRAVÉS DE MEMÓRIAS ARTIFICIAIS}}
\indent
\small

Memórias e atitudes contribuem para a construção do consciente e o caráter humano a partir de diversos acontecimentos, sejam eles bons ou não, ambos contribuindo para o aprendizado. No caso das memórias, o consciente direciona as atitudes, demandando menos processamento do cérebro. A inteligência artificial usa desse mesmo princípio de acontecimentos para o seu desenvolvimento, ou seja, a utilização de memórias artificiais teoricamente poderiam simular atitudes humanas.

\noindent
 
\textbf{Palavras-Chave}: Inteligência,Artificial,Memórias,Humanas,Tecnologia,Moderno,Mundo,
\\Consciência,Através,Cérebro.

\vspace{6cm}

\chapter{Introdução} 
“A inteligência artificial (IA), trouxe várias aplicações importantes para níveis de desempenho quase humanos” \cite{anthes}.
As pessoas mesmo involuntariamente possuem atos, estes que são aprendidos a partir de algum fato do seu cotidiano, ou por experiências passadas. Esses atos são armazenados em pequenas informações mnemônicas, o que pode direcionar na tomada de decisões \cite{buzsaki}. Estes autores afirmam que para a tomada de decisões ideais é necessário um mecanismo que integre as populações neurais distribuídas. Segundo \cite{eichenbaum}, essas decisões são tomadas a partir de memórias do hipocampo, que é uma estrutura que representa a memória para eventos passados.



\chapter{Revisão Sistemática}
Para confecção deste artigo, foi utilizada a técnica de Revisão Sistemática de Literatura(RSL). Esta possui o objetivo de realizar um trabalho de pesquisa e a partir deste trabalho, ser capaz de interpretar e analisar estudos sobre determinado assunto \cite{brereton}. Dyba, Kitchenham,  Jorgensen(2005), propuseram um processo que resume de forma sucinta como aplicar esta técnica.


\vspace{6cm}
\chapter{Viabilidade Econômica}

A viabilidade econômica de qualquer projeto deve levar em consideração os custos de implantação, operação, manutenção, localização e pessoal. A proposta do SUM, consiste em um sistema de segurança, onde os equipamentos possuem preços elevados, os sistemas emergenciais e afins devem ser de boa qualidade, o que refletem diretamente no valor total do conjunto.

A análise da viabilidade para o projeto em si, levou em consideração a soma dos custos de todas as áreas envolvidas, e posteriormente, dispostas em comparação com outras propostas para o monitoramento da área do estacionamento da UnB-FGA.

É importante ressaltar que o SUM possui um custo de implementação e operação mais elevados que as outras propostas devido à suas disposições e possibilidades, como a identificação e alerta à equipe de segurança. As outras propostas consistem apenas no monitoramento e gravação dos fatos, não dispondo de tecnologia suficiente para identificar e alertar de maneira autônoma uma atividade de risco ao patrimônio, contudo, apresenta um caráter ostensivo, o qual estatísticamente reduz os índices de furto e criminalidade na região.

Após observamos os custos analiticamente, levando em consideração as atribuições, vantagens e desvantagens de cada proposta, a equipe de desenvolvimento do Sistema Unificado de Monitoramento conclui que a implementação do Sistema na UnB-FGA é inviável, devido ao valores elevados para tornar o projeto operacional e dificuldades burocráticas quando estamos nos referindo à treinamento e deslocamento de uma equipe terceirizada para operar as câmeras. Contudo, a implementação de sistemas de segurança mais simples, atenderia de maneira eficaz na redução dos índices de furtos e atividades ilícitas nas proximidades do campus.

A proposta de implementar balões de vigilância, aparenta maior funcionalidade e eficiência quando usados de maneira sazonial, como em jogos olímpicos, shows, eventos em espaços abertos e semelhantes, o que de maneira direta, reduzem os custos de maneira drástica. Quando citamos a implementação de maneira fixa e permanente, os custos totais tornam o projeto inviável.

Todos os valores relacionados aos custos para implantação e sustentação do sistema SUM estão dispostos nas tabelas \ref{tab:custosSoftware}, \ref{tab:custosEletronica}, \ref{tab:CustosEstrutura}, \ref{tab:custosEnergia} e \ref{tab:custosTotais}.

\begin{table}[]
\centering
\caption{Custos Estação de Software}
\label{tab:custosSoftware}
\begin{tabular}{|c|c|c|c|}
\hline
\rowcolor[HTML]{FFFFFF} 
{\color[HTML]{000000} \textbf{Item}} & \textbf{Quantidade} & \textbf{Preço} & \textbf{Total}        \\ \hline
Engenheiro de Software               & 4                   & R\$3.000,00    & R\$12.000,00          \\ \hline
Operador                             & 4                   & R\$2.000,00    & R\$8.000,00           \\ \hline
Waveshare OV5647 Night Vision        & 15                  & R\$123,34      & R\$1.850,10           \\ \hline
Seagate Archive 8TB                  & 4                   & R\$ 984.90     & R\$ 3939.60           \\ \hline
Sistema Nobreak                      & 1                   & R\$ 396.90     & R\$ 396.90            \\ \hline
Hardware                             & 1                   & R\$ 5271.86    & R\$ 5271.86           \\ \hline
Walk talk Cobra Cxr925               & 5                   & R\$415,99      & R\$2.079,95           \\ \hline
Container                            & 1                   & R\$6.000,00    & R\$6.000,00           \\ \hline
\rowcolor[HTML]{C0C0C0} 
\multicolumn{3}{|c|}{\cellcolor[HTML]{C0C0C0}\textbf{Total}}                & \textbf{R\$29.930,05} \\ \hline
\end{tabular}
\end{table}

\begin{table}[]
\centering
\caption{Custos Eletrônica Embarcada}
\label{tab:custosEletronica}
\begin{tabular}{|c|c|c|c|}
\hline
\rowcolor[HTML]{FFFFFF} 
{\color[HTML]{000000} \textbf{Item}} & \textbf{Quantidade} & \textbf{Preço} & \textbf{Total}        \\ \hline
Arduino uno                          & 5                   & R\$ 42,99      & R\$ 214,95            \\ \hline
Galileo                              & 5                   & R\$ 184,92     & R\$ 924,60            \\ \hline
Rasberry                             & 8                   & R\$ 127,28     & R\$ 1.018,24          \\ \hline
sensor ADXL345                       & 5                   & R\$ 17,00      & R\$ 85,00             \\ \hline
Sensor BMP180                        & 5                   & R\$ 9,11       & R\$ 45,55             \\ \hline
L3G4200D                             & 5                   & R\$ 23,32      & R\$ 116,60            \\ \hline
HMC5883L                             & 5                   & R\$ 9,15       & R\$ 45,75             \\ \hline
Sensor DHT11                         & 5                   & R\$ 10,28      & R\$ 51,40             \\ \hline
Câmera                               & 8                   & R\$ 102,62     & R\$ 820,96            \\ \hline
Aweek 850 nm                         & 8                   & R\$ 96,11      & R\$ 768,88            \\ \hline
Antena Wiriless Omni                 & 8                   & R\$ 27,00      & R\$ 216,00            \\ \hline
Placa de Fenolite                    & 5                   & R\$ 2,82       & R\$ 14,10             \\ \hline
Sensor LM35                          & 5                   & R\$ 7,50       & R\$ 37,50             \\ \hline
Fios e Cabos                         & 5                   & R\$ 40,000     & R\$ 200,00            \\ \hline
Motor de passo                       & 5                   & R\$ 359,000    & R\$ 1.795,00          \\ \hline
SparkFun AutoDriver                  & 5                   & R\$ 134,560    & R\$ 672,80            \\ \hline
\rowcolor[HTML]{C0C0C0} 
\textbf{Total}                       &                     &                & \textbf{R\$ 7.027,33} \\ \hline
\end{tabular}
\end{table}


\begin{table}[]
\centering
\caption{Custos da Estrutura}
\label{tab:CustosEstrutura}
\begin{tabular}{|c|c|c|c|}
\hline
\rowcolor[HTML]{FFFFFF} 
{\color[HTML]{000000} \textbf{Item}} & \textbf{Quantidade} & \textbf{Preço} & \textbf{Total}          \\ \hline
Poste                                & 4                   & R\$ 1.389,00   & R\$ 5.556,00            \\ \hline
Chumbador                            & 16                  & R\$ 21,41      & R\$ 342,56              \\ \hline
Cabo                                 & 100                 & R\$ 5,00       & R\$ 500,00              \\ \hline
Gás                                  & 5                   & R\$ 15.960,00  & R\$ 79.800,00           \\ \hline
Motor                                & 10                  & R\$ 1.700,00   & R\$ 17.000,00           \\ \hline
Envelope                             & 5                   & R\$ 1.100,00   & R\$ 5.500,00            \\ \hline
Suporte para o motor                 & 10                  & R\$1.994,92    & R\$19.949,20            \\ \hline
\rowcolor[HTML]{C0C0C0} 
\multicolumn{3}{|c|}{\cellcolor[HTML]{C0C0C0}\textbf{Total}}                & \textbf{R\$ 128.647,76} \\ \hline
\end{tabular}
\end{table}

\begin{table}[]
\centering
\caption{Custos da Energia}
\label{tab:custosEnergia}
\begin{tabular}{|c|c|c|c|}
\hline
\rowcolor[HTML]{FFFFFF} 
{\color[HTML]{000000} \textbf{Item}} & \textbf{Quantidade} & \textbf{Preço} & \textbf{Total}        \\ \hline
Bateria                              & 18                  & R\$ 75,00      & R\$ 1.350,00          \\ \hline
\rowcolor[HTML]{C0C0C0} 
\multicolumn{3}{|c|}{\cellcolor[HTML]{C0C0C0}\textbf{Total}}                & \textbf{R\$ 1.350,00} \\ \hline
\end{tabular}
\end{table}

As possibilidades alternativas a implantação do sistema SUM estão dispostas nas tabelas \ref{tab:poss1} e \ref{tab:poss2}.

\begin{table}[]
\centering
\caption{Possibilidade 1}
\label{tab:poss1}
\begin{tabular}{|c|c|c|c|}
\hline
\rowcolor[HTML]{FFFFFF} 
{\color[HTML]{000000} \textbf{Possibilidade 1}} & \textbf{Quantidade} & \textbf{Preço} & \textbf{Total}    \\ \hline
Funcionarios de segurança                       & 6                   & 2060,33        & 12361,98          \\ \hline
\rowcolor[HTML]{C0C0C0} 
\multicolumn{3}{|c|}{\cellcolor[HTML]{C0C0C0}\textbf{Total}}                           & \textbf{12361,98} \\ \hline
\end{tabular}
\end{table}


\begin{table}[]
\centering
\caption{Possibilidade 2}
\label{tab:poss2}
\begin{tabular}{|c|c|c|c|}
\hline
\rowcolor[HTML]{FFFFFF} 
{\color[HTML]{000000} \textbf{Possibilidade 2}} & \textbf{Quantidade} & \textbf{Preço} & \textbf{Total}    \\ \hline
Câmeras                                         & 12                  & 102,62         & 1231,44           \\ \hline
Postes                                          & 6                   & R\$ 5.556,36   & 33338,16          \\ \hline
Funcionários                                    & 4                   & 2060,33        & 8241,32           \\ \hline
\rowcolor[HTML]{C0C0C0} 
\multicolumn{3}{|c|}{\cellcolor[HTML]{C0C0C0}\textbf{Total}}                           & \textbf{42810,92} \\ \hline
\end{tabular}
\end{table}

Os custos relacionados a manutenção do sistema estão dispostos na tabela \ref{tab:custoManutencao}.

\begin{table}[]
\centering
\caption{Custos de Manutenção}
\label{tab:custoManutencao}
\begin{tabular}{|c|c|}
\hline
\rowcolor[HTML]{C0C0C0} 
\textbf{Manutenção}                                                               & \textbf{Preço} \\ \hline
Energia on Gride                                                                  & R\$ 2.000      \\ \hline
Operador                                                                          & R\$ 8000       \\ \hline
Reposição de Gás                                                                  & R\$ 5000       \\ \hline
\begin{tabular}[c]{@{}c@{}}manutenção de equipamentos \\ eletrônicos\end{tabular} & R\$ 2457,2     \\ \hline
Eletricista                                                                       & R\$ 1227,5     \\ \hline
Engenheiro para trabalhos gerais                                                  & R\$ 2473,47    \\ \hline
TOTAL                                                                             & R\$ 18.685     \\ \hline
\end{tabular}
\end{table}


\begin{table}[]
\centering
\caption{Custos Totais}
\label{tab:custosTotais}
\begin{tabular}{|c|c|}
\hline
\rowcolor[HTML]{C0C0C0} 
\textbf{Área}        & \textbf{Custos Totais} \\ \hline
Estação de Software  & R\$29.930,05           \\ \hline
Eletrônica Embarcada & R\$ 7.027,33           \\ \hline
Estrutura            & R\$ 128.647,76         \\ \hline
Energia              & R\$ 1.350,00           \\ \hline
\rowcolor[HTML]{F56B00} 
TOTAL                & R\$166.955,14          \\ \hline
\end{tabular}
\end{table}

\vspace{6cm}
\chapter{Conclusões}
Este trabalho buscou averiguar se é possível a simulação de atividades humanas por meio da tecnologia de Inteligência Artificial. A conscientização de tecnologias que tentam aproximar decisões humanas a máquinas já são empregadas em nosso cotidiano, como carros com navegação personalizada, sistemas embutidos de inteligência artificial em \textit{smartphones}, como no caso da \textit{SIRI} e \textit{Android} para as empresas internacionais \text{Apple} e \textit{Google} respectivamente.

Para a confecção deste trabalho foi aplicada a técnica de Revisão Sistemática de Literatura (RSL) para direcionar um método de pesquisa e assim organizar as publicações conforme o tema, em bases de dados selecionadas, sendo feita a partir de palavras-chave e \textit{string} de busca.

As publicações foram escolhidas a partir dos critérios de inclusão e o conjunto de publicações que não foram selecionadas, foram excluídas a partir do critério de exclusão. As publicações foram categorizadas em duas categorias: Tomada de Decisões e Inteligência Artificial. A maior parte dos artigos selecionados, foram publicados nos últimos 20 anos. Com a realização da pesquisa deste trabalho, foi possível alcançar resultados favoráveis aos objetivos desejados. Foi constatado que existem estudos na área com potenciais para o desenvolvimento de sistemas artificiais que interajam de forma mais humana. Esses sistemas poderão simular sentimentos e ações humanas, e se adaptar conforme o conjunto social em que esteja envolvido.

Após a análise dos dados deste trabalho, é verificado que existe a possibilidade de desenvolvimento para novos trabalhos, envolvendo o desenvolvimento da Inteligência Artificial e tecnologias emergentes. Para um melhor desenvolvimento dessa área, pesquisadores de IA poderiam estudar mais como funciona o cérebro humano e a partir disso aplicar esses estudos na tecnologia, o que poderia levar a resultados mais próximos do desejado.
