Esta atividade é o ínicio da segunda fase do processo da Figura \ref{img:processo}, sendo a fase de execução. Nesta fase é a que utiliza toda a bagagem da primeira fase do processo e a executa, com o propósito de alcançar o objetivo deste trabalho.

\section{Execução da Busca}

Na execução desta atividade, foi utilizadas as \textit{Strings} de busca, definidas no tópico \ref{sec:planejamento}, e aplicadas nas bases de dados.

Com o intuito de selecionar os artigos que melhor se adeque ao tema, foram utilizados os critérios de inclusão. Os artigos resultantes da pesquisa utilizando a \textit{String} de busca e que atendiam aos critérios de inclusão, foram armazenados para uma posterior análise crítica deste trabalho.

Após feita a seleção das publicações, foram extraídos dados e informações que pudessem classificar as iniciativas e aumentar o entendimento do tópico em questão. As áreas mais recorrentes foram categorizadas em duas áreas: Inteligência artificial e Tomadas de Decisões. A escolha dessas duas categorias se deu ao fato de ser necessário entender primeiro como se a distribuição neural do cérebro humano e como essa distribuição pode auxiliar a inteligência artificial.

\begin{table}[h]
\centering
\begin{tabular}{ | l | l | p{5cm} | l | p{10cm} |}
\hline
\rowcolor[HTML]{F8FF00} 
{\color[HTML]{000000} \textbf{Categoria}} & \textbf{Ano} & \textbf{Título} & \textbf{Autores}        \\ \hline
{\cellcolor[HTML]{34FF34}}\textbf{Tomada de Decisão}               & 2013                   & Synchronization of Medial Temporal Lobe and Prefrontal Rhythms in Human Decision Making    & Guitart-Masip, Marc;Barnes          \\ \hline
 & 2007 & The medial temporal lobe and recognition memory & Eichenbaum H, Yonelinas AP\\ \hline
 & 2015 & Composite collective decision-making & Czaczkes, Tomer J ; Benjamin \\ \hline
 {\cellcolor[HTML]{34FF34}}\textbf{Inteligência Artificial} & 2017 & Artificial intelligence poised to ride a new wave & Anthes Gary \\ \hline
 & 2010 & Artificial Learning in Artificial Memories & Burger, John Robert \\ \hline
 & 2012 & Interdisciplinary Reviews: Computational Statistics & Tecuci, Gheorghe,Wiley \\ \hline
\end{tabular}
\caption{Categorização das Áreas de Pesquisa}
Fonte: Autor
\label{tab:categorizacao}
\end{table}