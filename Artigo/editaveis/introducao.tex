“A inteligência artificial (IA), trouxe várias aplicações importantes para níveis de desempenho quase humanos” \cite{anthes}.
As pessoas mesmo involuntariamente possuem atos, estes que são aprendidos a partir de algum fato do seu cotidiano, ou por experiências passadas. Esses atos são armazenadas em pequenas informações mnemônicas, o que pode direcionar na tomada de decisões \cite{buzsaki}. Estes autores afirmam que  para a tomada de decisões ideais é necessário um mecanismo que integre  as populações neurais distribuídas. Segundo \cite{eichenbaum}, essas decisões são tomadas a partir de memórias do hipocampo, que é uma estrutura que representa a memória para eventos passados.

