O principal objetivo deste trabalho foi a investigação a possibilidade de simular atitudes humanas, com a utilização da inteligência artificial.

\cite{anthes} ressalta que a inteligência artificial (IA) trouxe várias aplicações importantes para níveis de desempenho quase humanos. Para isso, é imprescindível que se entenda como os humanos procedem. A partir desses procedimentos é possível desenvolver sistemas que busquem imitar a inteligência humana.

As pessoas, mesmo involuntariamente, executam procedimentos e atividades que foram aprendidas a partir de algum fato do seu cotidiano ou de experiências passadas. \cite{buzsaki}, afirmam que as atividades são armazenadas como informações mnemônicas, o que pode direcionar a tomada de decisões.

Os resultados encontrados na elaboração deste trabalho, foi que existe a possibilidade de criar sistemas que simulem atitudes humanas a partir de aprendizado artificial. A inteligência artificial utiliza procedimentos de repetição como  aprendizado. A forma como a tecnologia aprende é muito aproximada a como os seres humanos aprendem, a partir de repetição de atividades rotineiras.

Foi utilizado a técnica de revisão sistemática de literatura (RSL) na confecção deste trabalho. Este trabalho foi organizado da seguinte maneira: 

\begin{itemize}
	\item No tópico \ref{sec:revisaosistematica} foi feita a modelagem do processo da técnica de revisão sistemática de literatura (RSL).
		\begin{itemize}
		\item O subtópico \ref{subsec:objetivos} foram descritos os objetivos do trabalho, bem como as questões de pesquisa.
		\end{itemize}
	\item No tópico \ref{sec:planejamento} foi descrito o planejamento sobre o trabalho. As palavras chave a serem utilizadas nas bases de dados. A partir dessas palavras foram feitas \textit{Strings} de busca e descritas as bases de dados escolhidas.
		\begin{itemize}
		\item No subtópico \ref{subsec:inclusao} foram descritos os critérios de inclusão, com o objetivo de selecionar as publicações que não fogem do tema.
		\item No subtópico \ref{subsec:exclusao} foram descritos os critérios de exclusão, com o objetivo de excluir as publicações que fogem do tema.
		\end{itemize}
	\item O tópico \ref{sec:execucao} foi feita a execução da revisão sistemática
		\begin{itemize}
		\item O subtópico \ref{subsec:busca} foram feitas categorizações, baseado nas áreas mais recorrentes das pesquisas, utilizando a String de busca.
		\end{itemize}.
	\item O tópico \ref{sec:resultados} foi realizada o resultado da técnica de revisão sistemática de literatura (RSL).
		\begin{itemize}
		\item O subtópico \ref{subsec:decisoes} descreve o resultado da pesquisa sobre a categoria tomada de atitudes.
		\item O subtópico \ref{subsec:artificial} descreve o resultado da pesquisa sobre a categoria inteligência artificial.
		\end{itemize}.
		\item O último tópico deste trabalho são feitas as considerações finais.
\end{itemize}