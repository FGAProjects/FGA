Este tópico aborda sobre a pesquisa realizada, pesquisa em bases de busca utilizadas na realização da pesquisa, além das palavras-chave e a \textit{String} de busca.

Com a intenção de encontrar uma maior variedade de artigos e revistas publicadas, foi escolhido o idioma inglês. Este idioma, por se tratar de um idioma conhecido no mundo inteiro e ser o principal idioma para publicações internacionais, o que auxiliou na procura de artigos relevantes para este trabalho.

Após ser definido o objetivo do trabalho e as questões de pesquisa que foram levantadas a partir deste, foram escolhidas palavras-chave que ajudasse na construção de strings de busca. Após estes tópicos serem contemplados, foi escolhido o idioma que abrange um público internacional maior, que é o idioma inglês. A letra “P” foi utilizada para abreviação da palavra “Palavra”.

Palavras-chave identificadas:

\begin{itemize}
	\item P01. Intelligence
	\item P02. Artificial
	\item P03. Memories
	\item P04. Humans
    \item P05. Technology
    \item P06. Modern
    \item P07. World
    \item P08. Human
    \item P09. Consciousness
    \item P10. Through
    \item P11. Brain
\end{itemize}

Utilizando os tópicos anteriores de objetivos, questões de pesquisa, idioma escolhido e as palavras-chave, estas últimas serão utilizadas na próxima fase do processo da Figura 1. Estas palavras serão utilizadas como base para a especificação da \textit{String} de busca. As siglas “SB” a seguir, se referem as palavras “\textit{String} de busca”.

\begin{itemize}
    \item SB01:Intelligence artificial and modern world;
	\item SB02: Human Consciousness and Intelligence Artificial;
	\item SB03: Artificial Intelligence Through Memories;
    \item SB04: Artificial Learning and Intelligence Artificial.
\end{itemize}

As bases de dados escolhidas para a elaboração deste trabalho, foram escolhidas pelas suas relevâncias de pesquisa e extensão. Utilizando a sigla de BD, que é um acrônimo para base de dados, as seguintes bases de dados foram usadas para aplicação da técnica de Revisão Sistemática de Literatura (RSL).

\begin{itemize}
    \item BD1. Qualis CAPES;
    \item BD2. Google Academic.
\end{itemize} 

\section{Critérios de Inclusão}

Após serem definidas as \textit{Strings} de busca e inseridas nas bases de dados, foram adotados alguns critérios para a escolha dos trabalhos já publicados. A sigla “CI” é a abreviação das palavras Critérios de Inclusão.

\begin{itemize}
    \item CI01. A publicação deve estar em inglês;
    \item CI02. A publicação deve estar disponível nas bases para download ou disponível para leitura online;
    \item CI03. A publicação deve necessariamente estar de acordo com o tema e os objetivos propostos.
\end{itemize}

\subsection{Critérios de Exclusão}

Com a intenção de não ser aceito todos os artigos resultantes a partir da \textit{String} de busca, foram adotados alguns critérios para uma melhor escolha dos artigos aceitos. A sigla “CE” é a abreviação das palavras Critérios de Exclusão. 

\begin{itemize}
    \item CE01. A publicação não pode fugir do objetivo proposto.
\end{itemize}
