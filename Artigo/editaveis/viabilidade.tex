
A viabilidade econômica de qualquer projeto deve levar em consideração os custos de implantação, operação, manutenção, localização e pessoal. A proposta do SUM, consiste em um sistema de segurança, onde os equipamentos possuem preços elevados, os sistemas emergenciais e afins devem ser de boa qualidade, o que refletem diretamente no valor total do conjunto.

A análise da viabilidade para o projeto em si, levou em consideração a soma dos custos de todas as áreas envolvidas, e posteriormente, dispostas em comparação com outras propostas para o monitoramento da área do estacionamento da UnB-FGA.

É importante ressaltar que o SUM possui um custo de implementação e operação mais elevados que as outras propostas devido à suas disposições e possibilidades, como a identificação e alerta à equipe de segurança. As outras propostas consistem apenas no monitoramento e gravação dos fatos, não dispondo de tecnologia suficiente para identificar e alertar de maneira autônoma uma atividade de risco ao patrimônio, contudo, apresenta um caráter ostensivo, o qual estatísticamente reduz os índices de furto e criminalidade na região.

Após observamos os custos analiticamente, levando em consideração as atribuições, vantagens e desvantagens de cada proposta, a equipe de desenvolvimento do Sistema Unificado de Monitoramento conclui que a implementação do Sistema na UnB-FGA é inviável, devido ao valores elevados para tornar o projeto operacional e dificuldades burocráticas quando estamos nos referindo à treinamento e deslocamento de uma equipe terceirizada para operar as câmeras. Contudo, a implementação de sistemas de segurança mais simples, atenderia de maneira eficaz na redução dos índices de furtos e atividades ilícitas nas proximidades do campus.

A proposta de implementar balões de vigilância, aparenta maior funcionalidade e eficiência quando usados de maneira sazonial, como em jogos olímpicos, shows, eventos em espaços abertos e semelhantes, o que de maneira direta, reduzem os custos de maneira drástica. Quando citamos a implementação de maneira fixa e permanente, os custos totais tornam o projeto inviável.

Todos os valores relacionados aos custos para implantação e sustentação do sistema SUM estão dispostos nas tabelas \ref{tab:custosSoftware}, \ref{tab:custosEletronica}, \ref{tab:CustosEstrutura}, \ref{tab:custosEnergia} e \ref{tab:custosTotais}.

\begin{table}[]
\centering
\caption{Custos Estação de Software}
\label{tab:custosSoftware}
\begin{tabular}{|c|c|c|c|}
\hline
\rowcolor[HTML]{FFFFFF} 
{\color[HTML]{000000} \textbf{Item}} & \textbf{Quantidade} & \textbf{Preço} & \textbf{Total}        \\ \hline
Engenheiro de Software               & 4                   & R\$3.000,00    & R\$12.000,00          \\ \hline
Operador                             & 4                   & R\$2.000,00    & R\$8.000,00           \\ \hline
Waveshare OV5647 Night Vision        & 15                  & R\$123,34      & R\$1.850,10           \\ \hline
Seagate Archive 8TB                  & 4                   & R\$ 984.90     & R\$ 3939.60           \\ \hline
Sistema Nobreak                      & 1                   & R\$ 396.90     & R\$ 396.90            \\ \hline
Hardware                             & 1                   & R\$ 5271.86    & R\$ 5271.86           \\ \hline
Walk talk Cobra Cxr925               & 5                   & R\$415,99      & R\$2.079,95           \\ \hline
Container                            & 1                   & R\$6.000,00    & R\$6.000,00           \\ \hline
\rowcolor[HTML]{C0C0C0} 
\multicolumn{3}{|c|}{\cellcolor[HTML]{C0C0C0}\textbf{Total}}                & \textbf{R\$29.930,05} \\ \hline
\end{tabular}
\end{table}

\begin{table}[]
\centering
\caption{Custos Eletrônica Embarcada}
\label{tab:custosEletronica}
\begin{tabular}{|c|c|c|c|}
\hline
\rowcolor[HTML]{FFFFFF} 
{\color[HTML]{000000} \textbf{Item}} & \textbf{Quantidade} & \textbf{Preço} & \textbf{Total}        \\ \hline
Arduino uno                          & 5                   & R\$ 42,99      & R\$ 214,95            \\ \hline
Galileo                              & 5                   & R\$ 184,92     & R\$ 924,60            \\ \hline
Rasberry                             & 8                   & R\$ 127,28     & R\$ 1.018,24          \\ \hline
sensor ADXL345                       & 5                   & R\$ 17,00      & R\$ 85,00             \\ \hline
Sensor BMP180                        & 5                   & R\$ 9,11       & R\$ 45,55             \\ \hline
L3G4200D                             & 5                   & R\$ 23,32      & R\$ 116,60            \\ \hline
HMC5883L                             & 5                   & R\$ 9,15       & R\$ 45,75             \\ \hline
Sensor DHT11                         & 5                   & R\$ 10,28      & R\$ 51,40             \\ \hline
Câmera                               & 8                   & R\$ 102,62     & R\$ 820,96            \\ \hline
Aweek 850 nm                         & 8                   & R\$ 96,11      & R\$ 768,88            \\ \hline
Antena Wiriless Omni                 & 8                   & R\$ 27,00      & R\$ 216,00            \\ \hline
Placa de Fenolite                    & 5                   & R\$ 2,82       & R\$ 14,10             \\ \hline
Sensor LM35                          & 5                   & R\$ 7,50       & R\$ 37,50             \\ \hline
Fios e Cabos                         & 5                   & R\$ 40,000     & R\$ 200,00            \\ \hline
Motor de passo                       & 5                   & R\$ 359,000    & R\$ 1.795,00          \\ \hline
SparkFun AutoDriver                  & 5                   & R\$ 134,560    & R\$ 672,80            \\ \hline
\rowcolor[HTML]{C0C0C0} 
\textbf{Total}                       &                     &                & \textbf{R\$ 7.027,33} \\ \hline
\end{tabular}
\end{table}


\begin{table}[]
\centering
\caption{Custos da Estrutura}
\label{tab:CustosEstrutura}
\begin{tabular}{|c|c|c|c|}
\hline
\rowcolor[HTML]{FFFFFF} 
{\color[HTML]{000000} \textbf{Item}} & \textbf{Quantidade} & \textbf{Preço} & \textbf{Total}          \\ \hline
Poste                                & 4                   & R\$ 1.389,00   & R\$ 5.556,00            \\ \hline
Chumbador                            & 16                  & R\$ 21,41      & R\$ 342,56              \\ \hline
Cabo                                 & 100                 & R\$ 5,00       & R\$ 500,00              \\ \hline
Gás                                  & 5                   & R\$ 15.960,00  & R\$ 79.800,00           \\ \hline
Motor                                & 10                  & R\$ 1.700,00   & R\$ 17.000,00           \\ \hline
Envelope                             & 5                   & R\$ 1.100,00   & R\$ 5.500,00            \\ \hline
Suporte para o motor                 & 10                  & R\$1.994,92    & R\$19.949,20            \\ \hline
\rowcolor[HTML]{C0C0C0} 
\multicolumn{3}{|c|}{\cellcolor[HTML]{C0C0C0}\textbf{Total}}                & \textbf{R\$ 128.647,76} \\ \hline
\end{tabular}
\end{table}

\begin{table}[]
\centering
\caption{Custos da Energia}
\label{tab:custosEnergia}
\begin{tabular}{|c|c|c|c|}
\hline
\rowcolor[HTML]{FFFFFF} 
{\color[HTML]{000000} \textbf{Item}} & \textbf{Quantidade} & \textbf{Preço} & \textbf{Total}        \\ \hline
Bateria                              & 18                  & R\$ 75,00      & R\$ 1.350,00          \\ \hline
\rowcolor[HTML]{C0C0C0} 
\multicolumn{3}{|c|}{\cellcolor[HTML]{C0C0C0}\textbf{Total}}                & \textbf{R\$ 1.350,00} \\ \hline
\end{tabular}
\end{table}

As possibilidades alternativas a implantação do sistema SUM estão dispostas nas tabelas \ref{tab:poss1} e \ref{tab:poss2}.

\begin{table}[]
\centering
\caption{Possibilidade 1}
\label{tab:poss1}
\begin{tabular}{|c|c|c|c|}
\hline
\rowcolor[HTML]{FFFFFF} 
{\color[HTML]{000000} \textbf{Possibilidade 1}} & \textbf{Quantidade} & \textbf{Preço} & \textbf{Total}    \\ \hline
Funcionarios de segurança                       & 6                   & 2060,33        & 12361,98          \\ \hline
\rowcolor[HTML]{C0C0C0} 
\multicolumn{3}{|c|}{\cellcolor[HTML]{C0C0C0}\textbf{Total}}                           & \textbf{12361,98} \\ \hline
\end{tabular}
\end{table}


\begin{table}[]
\centering
\caption{Possibilidade 2}
\label{tab:poss2}
\begin{tabular}{|c|c|c|c|}
\hline
\rowcolor[HTML]{FFFFFF} 
{\color[HTML]{000000} \textbf{Possibilidade 2}} & \textbf{Quantidade} & \textbf{Preço} & \textbf{Total}    \\ \hline
Câmeras                                         & 12                  & 102,62         & 1231,44           \\ \hline
Postes                                          & 6                   & R\$ 5.556,36   & 33338,16          \\ \hline
Funcionários                                    & 4                   & 2060,33        & 8241,32           \\ \hline
\rowcolor[HTML]{C0C0C0} 
\multicolumn{3}{|c|}{\cellcolor[HTML]{C0C0C0}\textbf{Total}}                           & \textbf{42810,92} \\ \hline
\end{tabular}
\end{table}

Os custos relacionados a manutenção do sistema estão dispostos na tabela \ref{tab:custoManutencao}.

\begin{table}[]
\centering
\caption{Custos de Manutenção}
\label{tab:custoManutencao}
\begin{tabular}{|c|c|}
\hline
\rowcolor[HTML]{C0C0C0} 
\textbf{Manutenção}                                                               & \textbf{Preço} \\ \hline
Energia on Gride                                                                  & R\$ 2.000      \\ \hline
Operador                                                                          & R\$ 8000       \\ \hline
Reposição de Gás                                                                  & R\$ 5000       \\ \hline
\begin{tabular}[c]{@{}c@{}}manutenção de equipamentos \\ eletrônicos\end{tabular} & R\$ 2457,2     \\ \hline
Eletricista                                                                       & R\$ 1227,5     \\ \hline
Engenheiro para trabalhos gerais                                                  & R\$ 2473,47    \\ \hline
TOTAL                                                                             & R\$ 18.685     \\ \hline
\end{tabular}
\end{table}


\begin{table}[]
\centering
\caption{Custos Totais}
\label{tab:custosTotais}
\begin{tabular}{|c|c|}
\hline
\rowcolor[HTML]{C0C0C0} 
\textbf{Área}        & \textbf{Custos Totais} \\ \hline
Estação de Software  & R\$29.930,05           \\ \hline
Eletrônica Embarcada & R\$ 7.027,33           \\ \hline
Estrutura            & R\$ 128.647,76         \\ \hline
Energia              & R\$ 1.350,00           \\ \hline
\rowcolor[HTML]{F56B00} 
TOTAL                & R\$166.955,14          \\ \hline
\end{tabular}
\end{table}